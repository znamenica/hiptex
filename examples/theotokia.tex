\documentclass[14pt,twoside]{extreport}
\usepackage[T2A]{fontenc}
\usepackage[cp1251]{inputenc}
\usepackage[a5paper,xdvi,civil=antiqua]{izhitsabook}
\usepackage{titletoc}

\tolerance 900

\renewcommand{\*}{~~\raise3pt\hbox{\footnotesize*}}

\titlecontents{section}[0em]{}{}{}{\titlerule*[0.5pc]{.}{\civil\footnotesize\contentspage}}
\titlecontents{subsection}[0em]{\small}{}{}{\titlerule*[0.5pc]{.}{\civil\footnotesize\contentspage}}

\addtocontents{toc}{\protect\setcounter{tocdepth}{2}\ignorespaces}

% ==========================================================================================

\begin{document}

\stdcrosstitle{
  \MakeUppercase{Бг~орw'дичны _о=сми` гласw'въ} \\
  \vskip 0.5\baselineskip
    {\large по{_е'}мыя, _е=гда` _е='сть Сла'ва ст~о'му въ мине'и}}

\maketitle

\stdsecondpage

% ==========================================================================================

\hdrcrosspage
\section[Бг~орw'дичны воскре'сны]
{\MakeUppercase{Бг~орw'дичны воскре'сны}}

\subsubsection{\MakeUppercase{на _о='смь гласw'въ}}

\subsection[Гла'съ а~]{Гла'съ а~, бг~оро'диченъ:}

\bukv{В}семi'рную сла'ву,\* w\т человjь^къ прозя'бшую,\* и= вл\дку
ро'ждшую,\* небе'сную две'рь воспои'мъ мр~i'ю дв~у,\*
безпло'тныхъ пjь'снь, и= вjь'рныхъ о_у=добре'нiе:\* сiя` бо
ja=ви'ся нб~о, и= хра'мъ бж~ества`:\* сiя` прегражде'нiе вражды`
разруши'вши,\* ми'ръ введе`, и= цр\ствiе w\тве'рзе,\* сiю`
о_у='бw и=му'ще вjь'ры о_у=твержде'нiе,\* побо'рника и='мамы и=зъ
нея` ро'ж\-д\-ша\-го\-ся гд\са.\* дерза'йте о_у=бw, дерза'йте лю'дiе
бж~iи:\* и='бо то'й победи'тъ враги`,\* ja='кw всеси'ленъ.

\subsubsection{На стiхо'внjь, бг~оро'диченъ:}

\bukv{С}е` и=спо'лнися и=са'iино прорече'нiе,\* дв~а бо родила` _e=си`,
\* и= по рождествjь` ja='кw пре'жде рождества` пребыла` _e=си`:
\* бг~ъ бо бjь` рожде'йся,\* тjь'мже и= _e=ст_eства`
новопресjьче`.\* но _w бг~о\-ма'\-ти!\* мол_e'нiя твои'хъ
рабw'въ,\* въ твое'мъ хра'мjь приноси^мая тебjь` не пре'зри:
\* но ja='кw бл~го\-у\-тро'б\-на\-го твои'ма рука'ма нося'щи,\* на
твоя^ рабы^ о_у=милосе'рдися,\* и= моли` спасти'ся душа'мъ
на'шымъ.

\delimpict

\subsection[Гла'съ в~]{Гла'съ в~, бг~оро'диченъ:}

\bukv{П}ре'йде сjь'нь зако'нная, бл~года'ти прише'дши:\* ja='коже
бо купина` не сгара'ше w=паля'ема:\* та'кw дв~а родила` _e=си`,
\* и= дв~а пребыла` _e=си`.\* вмjь'сто столпа` w='гненнаго,
\* пра'ведное возсiя` сл~нце:\* вмjь'сто мwv"се'я,\*
хр\сто'съ, спасе'нiе ду'шъ на'шихъ.

\subsubsection{На стiхо'внjь, бг~оро'диченъ:}

\bukv{_W} чудесе` но'вагw всjь'хъ дре'внихъ чуде'съ!\* кто' бо
позна` мт~рь безъ му'жа ро'ждшую,\* и= на руку` нося'щую, всю`
тва'рь содержа'щаго;\* бж~iе _e='сть и=зволе'нiе, ро'ждшееся.
\* _e=го'же ja='кw мл\днца, преч\стая, твои'ма рука'ма носи'вшая,
\* и= мт~рне дерзнове'нiе къ нему` и=му'щая,\* не преста'й
моля'щи w= чту'щихъ тя`,\* о_у=ще'дрити и= сп~сти` ду'шы на'шя.

\delimpict

\subsection[Гла'съ г~]{Гла'съ г~, бг~оро'диченъ:}

\bukv{К}а'кw не диви'мся бг~ому'жному рождеству` твоему`, преч\стна'я;\*
и=скуше'нiя бо му'жескагw не прiе'мши, всенепоро'чная,\* родила`
_e=си` без\ъ _о=ц~а` сн~а пло'тiю,\* пре'жде вjь^къ w\т _о=ц~а`
рожде'ннаго без\ъ мт~ре,\* ника'коже претепjь'вшаго и=змjьне'нiя,
и=ли` смjьше'нiя, и=ли` раздjьле'нiя,\* но _о=бою` существу` сво'йство
цjь'ло сохра'ншаго.\* тjь'мже, мт~и, дв~о вл\дчце,\* того` моли`
спасти'ся душа'мъ,\* правосла'внw бц\ду и=сповjь'дающихъ тя`.

\subsubsection{На стiхо'внjь, бг~оро'диченъ:}

\bukv{Б}ез\ъ сjь'мене w\т бж~е'ственнагw дх~а,\* во'лею же _о='ч~ею зачала`
_e=си` сн~а бж~iя,\* w\т _о=ц~а` без\ъ мт~ре пре'жде вjь^къ су'ща:\*
на'съ же ра'ди, и=зъ тебе` без\ъ _о=ц~а` бы'вша,\* пло'тiю родила`
_e=си`,\* и= мл\днца млеко'мъ пита'ла _e=си`.\* тjь'мже не преста'й
моли'ти,\* и=зба'витися w\т бjь'дъ душа'мъ на'шымъ.

\delimpict

\subsection[Гла'съ д~]{Гла'съ д~, бг~оро'диченъ:}

\bukv{И='}же тебе` ра'ди бг~о{_о}ц~ъ пр\оро'къ дв~дъ\* пjь'сненнw w=
тебjь` провозгласи`,\* вели^чiя тебjь` со\-тво'р\-ше\-му:\* предста` цр~и'ца
w=десну'ю тебе`. тя' бо мт~рь, хода'таицу живота` показа`,\* без\ъ
_о=ц~а` и=з\ъ тебе` вочеловjь'читися бл~говоли'вый бг~ъ:\* да сво'й
па'ки w=бнови'тъ w='бразъ, и=стлjь'вшiй страстьми`,\* и= заблу'ждшее
горохи'щное w=брjь'тъ _о=вча`,\* на ра'мо воспрiи'мъ, ко _о=ц~у`
принесе'тъ, и= своему` хотjь'нiю,\* съ небе'сными совокупи'тъ
си'лами,\* и= спасе'тъ бц\де, мi'ръ,\* хр\сто'съ и=мjь'яй ве'лiю и=
бога'тую мл\сть.

\subsubsection{На стiхо'внjь, бг~оро'диченъ:}

\bukv{П}ри'зри на мол_e'нiя твои'хъ ра^бъ, все\-не\-по\-ро'ч\-ная,\*
о_у=толя'ющи лю^тая на ны` воста^нiя,\* вся'кiя ско'рби на'съ
и=змjьня'ющи:\* тя' бо _e=ди'ну, тве'рдое и= и=звjь'стное
о_у=твержде'нiе и='мамы:\* и= твое` предста'тельство стяжа'хомъ,\* да
не постыди'мся, вл\дчце, тя` призыва'ющiи,\* потщи'ся на
о_у=моле'нiе,\* тебjь` вjь'рнw вопiю'щихъ:\* ра'\-дуй\-ся, вл\дчце,\*
всjь'хъ по'моще, ра'досте и= покро'ве,\* и= спасе'нiе ду'шъ на'шихъ.

\delimpict

\subsection[Гла'съ _е~]{Гла'съ _е~, бг~оро'диченъ:}

\bukv{В}ъ чермнjь'мъ мо'ри,\* неискусобра'чныя невjь'сты _о='бразъ
написа'ся и=ногда`:\* та'мw мwv"се'й, раздjьли'тель воды`:\* здjь' же
гаврiи'лъ, служи'тель чудесе`.\* тогда` глубину` ше'ствова немо'креннw
i=и~ль:\* ны'нjь же хр\ста` роди` безсjь'меннw дв~а.\* мо'ре по
проше'ствiи i=и~левjь, пребы'сть непрохо'дно:\* непоро'чная по
рождествjь` _e=мману'илевjь, пребы'сть нетлjь'нна.\* сы'й, и= пре'жде
сы'й,\* ja=вле'йся ja='кw чл~вjь'къ,\* бж~е, поми'луй на'съ.

\subsubsection{На стiхо'внjь, бг~оро'диченъ:}

\bukv{Х}ра'мъ и= две'рь _e=си`,\* пала'та и= пр\сто'лъ цр~е'въ,\* дв~о
всечестна'я,\* _e='юже и=зба'витель мо'й, хр\сто'съ гд\сь, во тмjь`
спя'щымъ ja=ви'ся,\* сл~нце пра'вды, просвjьти'ти хотя`,\* ja=`же
созда' по _о='бразу своему` руко'ю свое'ю.\* тjь'мже, всепjь'тая,
ja='кw мт~рне дерзнове'нiе къ нему` стяжа'вшая,\* непреста'ннw моли`
спасти'ся душа'мъ на'шымъ.

\delimpict

\subsection[Гла'съ s~]{Гла'съ s~, бг~оро'диченъ:}

\bukv{К}то` тебе` не о_у=блажи'тъ прест~а'я дв~о;\* кто' ли не
воспое'тъ твоегw` преч\стагw рождества`;\* безлjь'тнw бо w\т _о=ц~а`
возсiя'вый сн~ъ _e=диноро'дный,\* то'йже w\т тебе` ч\стыя про'йде,\*
неизрече'ннw во\-пло'щ\-ся,\* _e=стество'мъ бг~ъ сы'й,\* и= _e=стество'мъ
бы'въ чл~вjь'къ на'съ ра'ди,\* не во двою` лицу^ раздjьля'емый,\* но
во двою` _e=стеству^\* несли'тнw познава'емый.\* того` моли`, ч\стая,
всебл~же'нная,\* поми'ловатися душа'мъ на'шымъ.

\subsubsection{На стiхо'внjь, бг~оро'диченъ:}

\bukv{Т}воре'цъ и= и=зба'витель мо'й, преч\стая, хр\сто'съ гд\сь\*
и=зъ твои'хъ ложе'снъ проше'дъ,\* въ мя` w=болкi'йся,\* пе'рвыя
кля'твы а=да'ма свободи`.\* тjь'мже ти`, всеч\стая,\* ja='кw бж~iи
мт~ри же и= дв~jь,\* вои'стинну вопiе'мъ немо'лчнw:\* ра'дуйся
а='гг~льски, ра'дуйся вл\дчце,\* предста'тельство и= покро'ве,\* и=
спасе'нiе ду'шъ на'шихъ.

\delimpict

\subsection[Гла'съ з~]{Гла'съ з~, бг~оро'диченъ:}

\bukv{М}т~и о_у='бw позна'лася _e=си`,\* па'че _e=стества`, бц\де,\*
пребыла' же _e=си` дв~а,\* па'че сло'ва и= ра'зума:\* и= чудесе`
рождества` твоегw сказа'ти я=зы'къ не мо'жетъ.\* пресла'вну бо су'щу
зача'тiю, ч\стая,\* непости'женъ _e='сть w='бразъ рожде'нiя:\*
и=дjь'же бо хо'щетъ бг~ъ, побjьжда'ется _e=стества` чи'нъ.\* тjь'мже
тя` вси`, мт~рь бж~iю вjь'дуще,\* мо'лимтися прилjь'жнw,\* моли`
спасти'ся ду'шамъ на'шымъ.

\subsubsection{На стiхо'внjь, бг~оро'диченъ:}

\bukv{П}од\ъ кро'въ тво'й, вл\дчце, вси` земноро'днiи прибjьга'юще,
вопiе'мъ ти`:\* бц\де о_у=пова'нiе на'ше,\* и=зба'ви ны` w\т
безмjь'рныхъ прегрjьше'нiй,\* и= спаси` ду'шы на'шя.

\delimpict

\subsection[Гла'съ и~]{Гла'съ и~, бг~оро'диченъ:}

\bukv{Ц}р~ь нб\сный за человjьколю'бiе на земли` ja=ви'ся,\* и= съ
человjь^ки поживе`:\* w\т дв~ы бо ч\стыя пло'ть прiе'мый,\* и= и=зъ
нея` проше'дый съ вос\-прi\-я'\-тi\-емъ:\* _e=ди'нъ _e='сть сн~ъ, сугу'бъ
_e=стество'мъ,\* но не v=поста'сiю.\* тjь'мже соверше'нна того`
бг~а,\* и= соверше'нна чл~вjь'ка вои'стинну проповjь'дающе,\*
и=сповjь'дуемъ хр\ста` бг~а на'шего:\* _e=го'же моли`, мт~и
безневjь'стная,\* поми'ловатися душа'мъ на'шымъ.

\subsubsection{На стiхо'внjь, бг~оро'диченъ:}

\bukv{Б}езневjь'стная дв~о,\* ja='же бг~а неизрече'ннw заче'нши
пло'тiю,\* мт~и бг~а вы'шнягw, твои'хъ рабw'въ мольбы^ прiими`,
всенепоро'чная,\* всjь^мъ подаю'щи w=чище'нiе прегрjьше'нiй:\* ны'нjь
на^ша мол_e'нiя прiе'млющи,\* моли` спасти'ся всjь^мъ на'мъ.

\vskip -2.5\baselineskip
\csendpictsmall
\clearpage

% ==========================================================================================

\hdrcrosspage
\section[Бг~орw'дичны _о=сми` гласw'въ]
{\MakeUppercase{Бг~орw'дичны _о=сми` гласw'въ}}

\subsubsection{по{_е'}мыя, _е=гда` _е='сть Сла'ва ст~о'му въ мине'и}

\subsubsection{И= ны'нjь, по гла'су сiя^}

\subsection[Гла'съ а~]{Въ недjь'лю ве'чера бг~оро'диченъ, гла'съ а~:}

\bukv{Н}б\сныхъ чинw'въ ра'дованiе, на земли` че\-ло\-вjь'\-кwвъ крjь'п\-кое
предста'тельство, преч\стая дв~о, спа\-си' ны, и=`же къ тебjь` прибjьга'ющыя:
jа='кw на тя` о_у=пова'нiе по бз~jь, бц\де, возложи'хомъ.

\subsubsection{Въ понедjь'льникъ, на о_у='трени бг~оро'диченъ:}

\bukv{С}т~jь'йшая ст~ы'хъ всjь'хъ си'лъ, честнjь'йшая всея` тва'ри, бц\де,
вл\дчце мi'ра, спаси' ны, сп~са ро'жд\-шая, w\т прегрjьше'нiй тмори'чныхъ и=
бjь'дъ, jа='кw бл~га'я, моли'твами твои'ми.

\subsubsection{Въ понедjь'льникъ ве'чера бг~оро'диченъ:}

\bukv{Д}в~о всепjь'тая, ю=`же къ тебjь` та'йну мwv"се'й пр\оро'ческима ви'дjь
_о=чи'ма, купину` горя'щу и= не\-w\-па\-ля'\-е\-му: бж~ества` бо _о='гнь о_у=тро'бу
твою`, ч\стая, не w=пали`. Тjь'мже мо'лимъ тя`, jа='кw мт~рь бг~а на'шегw,
ми'ръ и=спроси'ти мi'рови, и= ве'лiю ми'лость.

\subsubsection{Во вто'рникъ о_у='тра бг~оро'диченъ:}

\bukv{Б}лудни'цу и= блу'днаго а='зъ, и= разбо'йника побjьди'хъ, и= мытаря`
прегрjьше'нiй премину'хъ, и= нiнеvi'тяны. О_у=вы` мнjь`, что` бу'ду; ка'кw
о_у=бjь'гну му'ки а='зъ _о=кая'нный; ч\стая, припа'даю ти`, о_у=ще'дри мя`
ми'лостiю твое'ю, jа='коже _о='ныя сн~ъ тво'й спа'слъ _е='сть.

\subsubsection{Въ сре'ду ве'чера бг~оро'диченъ:}

\bukv{Р}а'дуйся, ра'дость пра'дjьдwвъ, а=п\столwвъ, и= му'\-че\-ни\-кwвъ весе'лiе,
и= покро'въ на'съ, дв~о, твои'хъ рабw'въ.

\subsubsection{Въ четверто'къ о_у='тра бг~оро'диченъ:}

\bukv{Р}а'дуйся, бг~оро'дице дв~о, ра'дуйся, похвало` всея` все\-ле'н\-ныя,
ра'дуйся, преч\стая мт~и бж~iя бл~го\-сло\-ве'н\-ная.

\subsubsection{Въ пято'къ ве'чера бг~оро'диченъ:}

\bukv{В}ои'стинну па'че о_у=ма` чу^дная твоя^ вели^чiя, рж\ства` твоегw`,
бг~оневjь'сто, jа=`же проповjь'даша пр\оро'цы вси`, вся^ пресла^вная, зача'тiе
и= рж\ство`, всепjь'тая, недомы'сленно, и= несказа'нно, и='мже мi'ръ спа'слъ
_е='сть, jа='кw милосе'рдъ.

\subsubsection{Въ суббw'ту о_у='тра бг~оро'диченъ:}

\bukv{Р}а'дуйся w\т на'съ, ст~а'я бц\де дв~о, чи'стый сосу'де всея`
вселе'нныя, свjьще` неугаси'мая, вмjьсти'лище не\-вмjь\-сти'\-ма\-гw, хра'ме
неwбори'мый: ра'дуйся, и=зъ нея'же роди'ся а='гнецъ бж~iй, взе'мляй грjьхи`
всегw` мi'ра.

\delimpict

\subsection[Гла'съ в~]{Въ недjь'лю ве'чера бг~ро'диченъ, гла'съ в~:}

\bukv{Р}а'дуйся, мр~i'е бц\де, хра'ме неразруши'мый, па'че же ст~ы'й, jа='коже
вопiе'тъ пр\оро'къ: свя'тъ хра'мъ тво'й, ди'венъ въ пра'вдjь.

\subsubsection{Въ понедjь'льникъ о_у='тра бг~оро'диченъ:}

\bukv{Н}а тя` о_у=пова'нiе, бц\де, возложи'хомъ ча'янiя, да не w\тпаде'мъ,
спаси` на'съ w\т бjь'дъ, помо'щнице недоумjь'_емымъ, и= сопроти'вныхъ совjь'ты
разори`: ты' бо _е=си` на'ше спасе'нiе, бл~гослове'нная.

\subsubsection{Въ понедjь'льникъ ве'чера бг~оро'диченъ:}

\bukv{Н}епроходи^мая врата` та'йнw за\-пе\-ча'т\-ст\-вw\-ван\-ная, бл~го\-сло\-ве'н\-ная бц\де
дв~о, прiими` мол_е'нiя на^ша, и= принеси` твоему` сн~у и= бг~у, да спасе'тъ
тобо'ю ду'шы на'шя.

\subsubsection{Во вто'рникъ о_у='тра бг~оро'диченъ:}

\bukv{Р}а'дуйся, мр~i'е бц\де, хра'ме неразруши'мый, па'че же ст~ы'й, jа='коже
вопiе'тъ пр\оро'къ: ст~ъ хра'мъ тво'й, ди'венъ въ пра'вдjь.

\subsubsection{Въ сре'ду ве'чера бг~оро'диченъ:}

\bukv{JА='}кw плодови'та ма'слина дв~а и=зрасти` тебе`, плода` живо'тнаго,
плодоноси'ти мi'рови ве'лiю и= бога'тую ми'лость.

\subsubsection{Въ четверто'къ о_у='тра бг~оро'диченъ:}

\bukv{В}се` о_у=пова'нiе мое` на тя` возлага'ю, мт~и бж~iя, сохрани' мя под\ъ
кро'вомъ твои'мъ.

\subsubsection{Въ пято'къ ве'чера бг~оро'диченъ:}

\bukv{С}паси` w\т бjь'дъ рабы^ твоя^, бц\де дв~о, jа='кw вси` по бз~jь къ
тебjь` прибjьга'емъ, jа='кw къ неруши'мjьй стjьнjь` и= предста'тельству.

\subsubsection{Въ суббw'ту о_у='тра бг~оро'диченъ:}

\bukv{П}рiиди'те, мт~рь свjь'та, пjь'сньми немо'лчными зову'ще, вси`
просла'вимъ, та' бо роди` спасе'нiе на'ше: и= ра'дуйся, принесе'мъ jа='кw
_е=ди'ной ро'ждшей всjь'хъ нача'льнjьйшаго, и='же пре'жде вjь^къ бг~а:
ра'дуйся, jа='же _е='vу па'дшую па'ки назда'вшая: ра'дуйся, пре\-ч\стая дв~о
неискусобра'чная.

\delimpict

\subsection[Гла'съ г~]{Въ недjь'лю ве'чера бг~оро'диченъ, гла'съ г~:}

\bukv{Б}ц\де предста'тельнице всjь'хъ моля'щихся тебjь`, тобо'ю дерза'емъ, и=
тобо'ю хва'лимся, и= къ тебjь` все` о_у=пова'нiе на'ше _е='сть, моли`
ро'ждшагося и=зъ теб_е` за непотр_е'бныя рабы^ твоя^.

\subsubsection{Въ понедjь'льникъ о_у='тра бг~оро'диченъ:}

\bukv{С}т~опервоч\стая похвало` су'щи нб\сныхъ чинw'въ, а=п\столwвъ пjь'нiе,
и= пр\оро'кwвъ сбытiе`, вл\дчце, прi\-и\-ми` моли^твы на'шя.

\subsubsection{Въ понедjь'льникъ ве'чера бг~оро'диченъ:}

\bukv{В}ъ жена'хъ ст~а'я бц\де мт~и безневjь'стная, моли`, _е=го'же родила`
_е=си`, цр~я` и= бг~а: да спасе'тъ на'съ, jа='кw чл~вjьколю'бецъ.

\subsubsection{Во вто'рникъ о_у='тра бг~оро'диченъ:}

\bukv{Б}езъ сjь'мене зачала` _е=си` w\т дх~а ст~а'гw, и= славосло'вяще
воспjьва'емъ тя`: ра'дуйся, прест~а'я дв~о.

\subsubsection{Въ сре'ду ве'чера бг~оро'диченъ:}

\bukv{В}ельми` согрjьша'юща мя`, _о=трокови'це, и=схити` вели'кою твое'ю
моли'твою пла'мене ну'жднагw, и= и=спра'ви, ч\стая, твои'ми мольба'ми, ко
спас_е'ннымъ стезя'мъ наставля'ющи мя` мт~рними твои'ми моли'твами.

\subsubsection{Въ четверто'къ о_у='тра бг~оро'диченъ:}

\bukv{Б}езъ сjь'мене зачала` _е=си` w\т дх~а ст~а'гw, и= славосло'вяще
воспjьва'емъ тя`: ра'дуйся, прест~а'я дв~о.

\subsubsection{Въ пято'къ ве'чера бг~оро'диченъ:}

\bukv{В}ъ жена'хъ ст~а'я бц\де мт~и безневjь'стная, моли`, _е=го'же родила`
_е=си`, цр~я` и= бг~а: да спасе'тъ на'съ, jа='кw чл~вjьколю'бецъ.

\subsubsection{Въ суббw'ту о_у='тра бг~оро'диченъ:}

\bukv{Б}езъ сjь'мене зачала` _е=си` w\т дх~а ст~а'гw, и= славосло'вяще
воспjьва'емъ тя`: ра'дуйся, прест~а'я дв~о.

\delimpict

\subsection[Гла'съ д~]{Въ недjь'лю ве'чера бг~оро'диченъ, гла'съ д~:}

\bukv{Р}а'дуйся, свjь'та _о='блаче: ра'дуйся, свjь'щнице свjь'тлый: ра'дуйся,
ру'чко, въ не'йже ма'нна: ра'\-дуй\-ся, же'зле а=арw'новъ: ра'дуйся, купино`
неwпали'мая: ра'дуйся, черто'же: ра'дуйся пр\сто'ле: ра'дуйся, горо` ст~а'я:
ра'дуйся, прибjь'жище: ра'дуйся, бж\ственная трапе'зо: ра'дуйся, две'ре
та'йная, ра'дуйся, всjь'хъ ра'досте.

\subsubsection{Въ понедjь'льникъ о_у='тра бг~оро'диченъ:}

\bukv{Б}ц\де, всjь'хъ цр~и'це, правосла'вныхъ похвало`, _е=ретi'чествующихъ
шата^нiя разори`, и= ли'ца и='хъ посрами`, не кла'няющихся, ниже` чту'щихъ,
пре\-ч\стая, ч\стны'й тво'й _о='бразъ.

\subsubsection{Въ понедjь'льникъ ве'чера бг~оро'диченъ:}

\bukv{И=}зба'ви на'съ w\т ну'ждъ на'шихъ, мт~и хр\ста` бг~а, ро'ждшая всjь'хъ
творца`, да вси` зове'мъ ти`: ра'дуйся, _е=ди'но предста'тельство ду'шъ
на'шихъ.

\subsubsection{Во вто'рникъ о_у='тра бг~оро'диченъ:}

\bukv{W\т} всjь'хъ бjь'дъ рабы^ твоя^ сохраня'й, благослове'нная бц\де: да тя`
сла'вимъ наде'жду ду'шъ на'шихъ.

\subsubsection{Въ сре'ду ве'чера бг~оро'диченъ:}

\bukv{И=}му'ще тя`, бц\де, о_у=пова'нiе` и= предста'тельство, вра'жiихъ
навjь^тъ не о_у=бои'мся, jа='кw спаса'еши ду'шы на'шя.

\subsubsection{Въ четверто'къ о_у='тра бг~оро'диченъ:}

\bukv{Т}я` стjь'ну стяжа'хомъ, бц\де преч\стая, и= благоути'шное приста'нище,
и= о_у=твержде'нiе. Тjь'мже молю'ся и='же въ житiи` w=бурева'емь: w=корми` и=
спаси' мя.

\subsubsection{Въ пято'къ ве'чера бг~оро'диченъ:}

\bukv{С}вjьще` неугаси'мая, пр\сто'ле пра'ведный, пре\-ч\стая вл\дчце, моли`
спасти'ся душа'мъ на'шымъ.

\subsubsection{Въ суббw'ту о_у='тра бг~оро'диченъ:}

\bukv{Е=}ди'ная ч\стая и= преч\стая дв~о, jа='же бг~а безсjь'меннw ро'ждшая,
моли` спасти'ся душа'мъ на'\-шымъ.

\delimpict

\subsection[Гла'съ _е~]{Въ недjь'лю ве'чера бг~оро'диченъ, гла'съ _е~:}

\bukv{С}тра'шно и= пресла'вно, и= ве'лiе та'инство, невмjьсти'мый во чре'вjь
вмjьсти'ся, и= мт~и по рж\ствjь` па'ки пребы'сть дв~а: бг~а бо роди` и=зъ нея`
воплоще'нна. тому` возопiи'мъ, тому` пjь'снь рце'мъ, со а='гг~лы воспjьва'юще:
ст~ъ _е=си`, хр\сте` бж~е, и='же на'съ ра'ди вочл~вjь'чивыйся, сла'ва тебjь`.

\subsubsection{Въ понедjь'льникъ о_у='тра бг~оро'диченъ:}

\bukv{W=}бра'дованная, хода'тайствуй твои'ми мо\-ли'т\-ва\-ми, и= и=спроси` ду\-ша'мъ
на'шымъ мно'жество щедро'тъ, и= w=чище'нiе мно'гихъ пре\-грjь\-ше'\-нiй, мо'\-лим\-ся.

\subsubsection{Въ понедjь'льникъ ве'чера бг~оро'диченъ:}

\bukv{О_у=}толи` болjь^зни многовоздыха'ющiя души` мо\-ея`, о_у=толи'вшая вся'ку
сле'зу w\т лица` земли`: ты' бо человjь'кwвъ болjь^зни w\тго'ниши, и=
грjь'шныхъ скw'рби разруша'еши. Тебе' бо вси` стяжа'хомъ наде'жду и=
о_у=твержде'нiе, прест~а'я мт~и дв~о.

\subsubsection{Во вто'рникъ о_у='тра бг~оро'диченъ:}

\bukv{W=}бра'дованная, хода'тайствуй твои'ми мо\-ли'т\-ва\-ми, и= и=спроси` ду\-ша'мъ
на'шымъ мно'жество щедро'тъ, и= w=чище'нiе мно'гихъ пре\-грjь\-ше'\-нiй, мо'\-лим\-ся.

\subsubsection{Въ сре'ду ве'чера бг~оро'диченъ:}

\bukv{Б}л~жи'мъ тя`, бц\де дв~о, и= сла'вимъ тя`, вjь'рнiи, по до'лгу, гра'дъ
непоколеби'мый, стjь'ну неwбори'мую, тве'рдую предста'тельницу и= прибjь'жище
ду'шъ на'шихъ.

\subsubsection{Въ четверто'къ о_у='тра бг~оро'диченъ:}

\bukv{Б}л~жи'мъ тя`, бц\де дв~о, jа='кw и=зъ теб_е` возсiя` сл~нце пра'вды
хр\сто'съ, и=мjь'яй ве'лiю ми'лость.

\subsubsection{Въ пято'къ ве'чера бг~оро'диченъ:}

\bukv{Т}ебjь` мо'лимся jа='кw б~жiи мт~ри: бл~гослове'нная, моли` w= спасе'нiи
ду'шъ на'шихъ.

\subsubsection{Въ суббw'ту о_у='тра бг~оро'диченъ:}

\bukv{_W}ле _о=кая'нная душе`! кi'й w\твjь'тъ и='маши рещи` судiи` во _о='нъ
ча'съ, _е=гда` пр\сто'ли поста'вятся на судjь`, и= судiя` прiи'детъ w\т нб~съ,
соше'дъ со тма'ми а='гг~льскими; _е=гда` ся'детъ на суди'щи, прю` сотвори'ти
съ рабы` непотре'бными, подо'бными мнjь`, что` w\твjьща'ти и='маши; что' же
принести` тогда`; пои'стиннjь ничто'же, о_у='мъ и= тjь'ло w=скверни'вши
твое`. Тjь'мже припади` къ дв~jь, и= зови` непреста'ннw, пода'ти тебjь`
бога'тнw грjьхw'въ проще'нiе.

\delimpict

\subsection[Гла'съ s~]{Въ недjь'лю ве'чера бг~оро'диченъ, гла'съ s~:}

\bukv{А=}рха'гг~льски воспои'мъ, вjь'рнiи, нб\сный черто'гъ, и= две'рь
запеча'танну вои'стинну: ра'дуйся, _е=я'же ра'ди w\трасте` на'мъ сп~съ всjь'хъ
хр\сто'съ, жизнода'вецъ и= бг~ъ: низложи`, вл\дчце, мучи'тели, без\-бw'ж\-ныя
враги` на'шя, руко'ю твое'ю, преч\стая, о_у=пова'нiе хр\стiа'нъ.

\subsubsection{Въ понедjь'льникъ о_у='тра бг~оро'диченъ:}

\bukv{А=}рха'гг~льское сло'во прiя'ла _е=си`, и= херувi'мскiй пр\сто'лъ
показа'лся _е=си`, и= на w=б\ъя'тiяхъ тво\-и'хъ носи'ла _е=си`, бц\де, наде'жду
ду'шъ на'шихъ.

\subsubsection{Въ понедjь'льникъ ве'чера бг~оро'диченъ:}

\bukv{Н}икто'же притека'яй къ тебjь` посра'мленъ w\т теб_е` и=схо'дитъ,
преч\стая бц\де дв~о: но про'ситъ бл~года'ти, и= прiе'млетъ дарова'нiе къ
поле'зному проше'нiю.

\subsubsection{Во вто'рникъ о_у='тра бг~оро'диченъ:}

\bukv{В}ели'кихъ дарова'нiй, ч\стая дв~о бг~ома'ти, ты` сподо'билася _е=си`,
jа='кw родила` _е=си` пло'тiю _е=ди'наго w\т тр\оцы хр\ста` жизнода'вца, во
спасе'нiе ду'шъ на'шихъ.

\subsubsection{Въ сре'ду ве'чера бг~оро'диченъ:}

\bukv{_О=}ко се'рдца моегw` воспуща'ю къ тебjь`, вл\дчце, не пре'зри ма'лагw
моегw` воздыха'нiя, въ ча'съ, _е=гда` су'дитъ сн~ъ тво'й мi'ру, бу'ди ми`
покро'въ и= помо'щница.

\subsubsection{Въ четверто'къ о_у='тра бг~оро'диченъ:}

\bukv{П}реложе'нiе скорбя'щихъ, премjьне'нiе боля'щихъ _е=си`, бц\де
всепjь'тая, спаси` гра'дъ и= лю'ди, бори'мыхъ о_у=мире'нiе, w=бурева'емыхъ
тишина`, _е=ди'на предста'тельнице вjь'рныхъ.

\subsubsection{Въ пято'къ ве'чера бг~оро'диченъ:}

\bukv{М}оли'твами ро'ждшiя тя`, хр\сте`, мч~никъ тво\-и'хъ и= а=п\слъ, и=
пр\орw'къ, и= ст~и'телей, прп\дбныхъ и= прв\дныхъ, и= всjь'хъ ст~ы'хъ,
о_у=со'пшыя рабы^ твоя^ о_у=поко'й.

\subsubsection{Въ суббw'ту о_у='тра бг~оро'диченъ:}

\bukv{Б}г~а и=зъ теб_е` воплоти'вшагося разумjь'хомъ, бц\де дв~о: того` моли`
w= спасе'нiи ду'шъ на'шихъ.

\delimpict

\subsection[Гла'съ з~]{Въ недjь'лю ве'чера бг~оро'диченъ, гла'съ з~:}

\bukv{Е='}же ра'дуйся, тебjь` зове'мъ со а=гг~ломъ, бг~о\-не\-вjь'\-сто, черто'гъ и=
две'рь, и= пр\сто'лъ _о='гненный нарица'юще тебе`, и= несjько'мую го'ру, и=
купину` неwпали'мую.

\subsubsection{Въ понедjь'льникъ о_у='тра бг~оро'диченъ:}

\bukv{О_у=}мири` мл~твами бц\ды жи'знь на'шу, вопiю'щихъ ти`: ми'лостиве
гд\си, сла'ва тебjь`.

\subsubsection{Въ понедjь'льникъ ве'чера бг~оро'диченъ:}

\bukv{Р}а'дуйся, сл~нца _о='блаче мы'сленнагw и= неизрече'ннагw вл\дчце:
ра'дуйся, всесвjь'тлая свjьще`: ра'дуйся, свjь'щниче всезлаты'й. Тобо'ю,
прест~а'я, _е='vа и=зба'вися w\т кля'твы. Но jа='кw и=му'щи дерзнове'нiе ко
бл~гопремjь'нному сн~у твоему` и= бг~у, мт~рнею твое'ю моли'твою не w=скудjь'й
моли'тися, преч\стая.

\subsubsection{Во вто'рникъ о_у='тра бг~оро'диченъ:}

\bukv{С}вjь'тъ хр\сте`, прозя'блъ _е=си` w\т дв~ы: и= просвjьти'лъ _е=си`
ро'дъ человjь'ческiй: гд\си, сла'ва тебjь`.

\subsubsection{Въ сре'ду ве'чера бг~оро'диченъ:}

\bukv{Е=}ди'ну по рж\ствjь` преч\стую дв~у воспои'мъ, jа='кw мт~рь бг~а
сло'ва, глаго'люще: сла'ва тебjь`.

\subsubsection{Въ четверто'къ о_у='тра бг~оро'диченъ:}

\bukv{Р}оди'лся _е=си` w\т дв~ы несказа'ннw, хр\сте`, и= просвjьти'лъ _е=си`
су'щыя во тмjь`, вопiю'щыя: гд\си, сла'ва тебjь`.

\subsubsection{Въ пято'къ ве'чера бг~оро'диченъ:}

\bukv{JА='}же _е=ди'на невмjьсти'маго прiи'мши, и= ро'ждши бг~а сло'ва
воплоще'нна, моли` спасти'ся душа'мъ на'шымъ.

\subsubsection{Въ суббw'ту о_у='тра бг~оро'диченъ:}

\bukv{М}оли`, дв~о, со а=п\слы и= му'ченики, w=брести` на судjь`
преста'вльшымся ве'лiю ми'лость.

\delimpict

\subsection[Гла'съ и~]{Въ недjь'лю ве'чера бг~оро'диченъ, гла'съ и~:}

\bukv{А=}рха'гг~ла гаврiи'ла гла'съ воспрiи'мше, рце'мъ: ра'дуйся, мт~и бж~iя,
jа='же жизнода'вца хр\ста` мi'ру ро'ждши.

\subsubsection{Въ понедjь'льникъ о_у='тра бг~оро'диченъ:}

\bukv{Н}б\сная пою'тъ тя`, w=бра'дованная мт~и безневjь'стная, и= мы`
славосло'вимъ неизслjь'дованное твое` рж\ство`, бц\де, моли` спасти'ся душа'мъ
на'шымъ.

\subsubsection{Въ понедjь'льникъ ве'чера бг~оро'диченъ:}

\bukv{Р}а'дуйся, вселе'нныя похвало`: ра'дуйся, хра'ме гд\снь: ра'дуйся, горо`
приwсjьне'нная: ра'дуйся, всjь'хъ прибjь'жище: ра'дуйся, свjь'щнице зла\-ты'й:
ра'дуйся, сла'во правосла'вныхъ ч\стна'я: ра'дуйся, мр~i'е мт~и хр\ста` бг~а:
ра'дуйся, раю`: ра'дуйся, бж\ственная трапе'зо: ра'дуйся, сjь'не: ра'дуйся,
ру'чко всезлата'я: ра'дуйся, всjь'хъ о_у=пова'нiе.

\subsubsection{Во вто'рникъ о_у='тра бг~оро'диченъ:}

\bukv{К}ро'въ тво'й, бц\де дв~о, врачество` _е='сть дх~о'вное: въ _о='нь бо
прибjьга'юще, w\т душе'вныхъ неду^гъ и=збавля'емся.

\subsubsection{Въ сре'ду ве'чера бг~оро'диченъ:}

\bukv{А='}зъ, дв~о ст~а'я бц\де, къ покро'ву твоему` прибjьга'ю, вjь'мъ,
jа='кw w=бря'щу тобо'ю спасе'нiе, мо'\-же\-ши бо, ч\стая, помощи` мнjь`.

\subsubsection{Въ четверто'къ о_у='тра бг~оро'диченъ:}

\bukv{И=}схи'ти мя`, вл\дчце, руки` sмi'я человjькоубi'йцы, хотя'ща мя`
лу\-ка'в\-ст\-вомъ поглоти'ти до конца`: сокруши` ч_е'люсти _е=гw`, молю' тя, и=
кw'зни разори`, jа='кw да и=збы'въ w\т ногте'й _е=гw`, велича'ю заступле'нiе твое`.

\subsubsection{Въ пято'къ ве'чера бг~оро'диченъ:}

\bukv{Ч}т\сая дв~о, сло'ва врата`, бг~а на'шегw мт~и, моли` спасти'ся на'мъ.

\subsubsection{Въ суббw'ту о_у='тра бг~оро'диченъ:}

\bukv{В}л\дчце, прiими` моли^твы рабw'въ твои'хъ, и= и=зба'ви на'съ w\т
вся'кiя ну'жды и= печа'ли.

\csendpict
\clearpage
\hdrcrosspage

\section[W\тпусти'тельныя воскре'сны _о=сми` гласw'въ
бг~оро'дичны]{\MakeUppercase{W\тпусти'тельныя воскре'сны _о=сми` гласw'въ
    бг~оро'дичны}}

\subsection{Гла'съ а~, бг~оро'диченъ:}

\bukv{Г}аврiи'лу вjьща'вшу тебjь`, дв~о, ра'дуйся, со гла'сомъ воплоща'шеся
всjь'хъ вл\дка въ тебjь` ст~jь'мъ кiвw'тjь, jа='коже рече` пра'ведный дв~дъ,
jа=ви'лася _е=си` ши'ршая нб~съ, поноси'вше зижди'теля твоего`: сла'ва
все'льшемуся въ тя`: сла'ва про\-ше'д\-ше\-му и=зъ теб_е`: сла'ва свободи'вшему
на'съ рождество'мъ твои'мъ.

\subsection{Гла'съ в~, бг~оро'диченъ:}

\bukv{В}ся^ па'че смы'сла, вся^ пресла^вная твоя^, бц\де, та^инства, чистотjь`
запеча'танной, и= дjь'вству храни'му, мт~и позна'лася _е=си` нело'жна, бг~а
ро'ждши и='стиннаго. того` моли`, спасти'ся душа'мъ на'шымъ.

\subsection{Гла'съ г~, бг~оро'диченъ:}

\bukv{Т}я` хода'тайствовавшую спасе'нiе ро'да на'шегw воспjьва'емъ, бц\де
дв~о: пло'тiю бо w\т теб_е` воспрiя'тою сн~ъ тво'й, и= бг~ъ на'шъ, кресто'мъ
воспрiи'мъ стра'сть, и=зба'ви на'съ w\т тли`, jа='кw чл~вjьколю'бецъ.

\subsection{Гла'съ д~, бг~оро'диченъ:}

\bukv{Е='}же w\т вjь'ка о_у=тае'нное, и= а='гг~лwмъ несвjь'домое та'инство,
тобо'ю, бц\де, су'щымъ на земли` jа=ви'ся бг~ъ, в несли'тномъ соедине'нiи
воплоща'емь, и= кре'стъ во'лею на'съ ра'ди воспрiи'мъ: и='мже воскреси'въ
первозда'ннаго, спасе` w\т сме'рти ду'шы на'шя.

\subsection{Гла'съ _е~, бг~оро'диченъ:}

\bukv{Р}а'дуйся, две'ре гд\сня непроходи'мая: ра'дуйся, стjьно` и= покро'ве
притека'ющихъ къ тебjь`. Ра'дуйся, неwбурева'емое приста'нище, и=
неискусобра'чная, ро'ждшая пло'тiю творца` твоего` и= бг~а: моля'щи не
w=скудjьва'й w= воспjьва'ющихъ, и= кла'няющихся рождеству` твоему`.

\subsection{Гла'съ s~, бг~оро'диченъ:}

\bukv{П}редповjьству'етъ гедеw'нъ зача'тiе, и= ска\-зу'\-етъ дв~дъ рождество`
твое`, бц\де: сни'де бо jа='кw до'ждь на руно` сл~во во чре'во твое`, и=
прозябла` _е=си` без\ъ сjь'мене земле` ст~а'я, мi'ра спасе'нiе, хр\ста` бг~а
на'шего, бл~года'тная.

\subsection{Гла'съ з~, бг~оро'диченъ:}

\bukv{JА='}кw на'шегw воскресе'нiя сокро'вище, на тя` надjь'ющыяся,
всепjь'тая, w\т ро'ва и= глубины` прегрjьше'нiй возведи`: ты' бо пови^нныя
грjьху` спасла` _е=си`, ро'ждшая спасе'нiе на'ше, jа='же пре'жде рождества`
дв~а, и= въ рождествjь` дв~а, и= по рождествjь` па'ки пребыва'еши дв~а.

\subsection{Гла'съ и~, бг~оро'диченъ:}

\bukv{И=}же на'съ ра'ди рожде'йся w\т дв~ы, и= распя'тiе претерпjь'въ,
бл~гi'й, и=спрове'ргiй сме'ртiю сме'рть, и= воскресе'нiе jа=вле'й jа='кw бг~ъ,
не пре'зри, jа=`же созда'лъ _е=си` руко'ю твое'ю: jа=ви` человjьколю'бiе
твое`, ми'лостиве, прiими` ро'ждшую тя` бц\ду моля'щуюся за ны`, и= спаси`,
сп~се на'шъ, лю'ди w\тча^янныя.

%\vskip 1.5\baselineskip
\csendpict
\clearpage
\thispagestyle{izhcontentspage}
\vskip -0.75\baselineskip
\tableofcontents

\end{document}
