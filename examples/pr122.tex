\documentclass[12pt,twoside,xdvi,a6paper,civil=times]{hipbook}
\usepackage{titletoc}

% ==========================================================================================
\hyphenation{прп\дбный}

\tolerance 2000

\renewcommand{\*}{\raise3pt\hbox{\footnotesize*}}

% ==========================================================================================

\titleformat{\section}[hang]
  {\small\filcenter}{}{0em}{\color{red}\bfseries }

\titlespacing*{\section}{0pt}{0.5\baselineskip}{0.0\baselineskip}

\setmarks{section}{section}

\newpagestyle{izhitsabookcivil}[\rm\small]{
  \sethead[\civil\scriptsize\usepage][\footnotesize
  \sectiontitle][]
  {}{\footnotesize
    \sectiontitle}{\civil\scriptsize\usepage}
  \headrule \setfoot{}{}{\footnotesize\nextword} }

\pagestyle{izhitsabookcivil}

% =============================================

\title{
\vskip -30 mm
 %\fbox{
      % \scalebox{0.6}{\includegraphics* [170, 450] [420, 827] {david1.ps}} 
      %\scalebox{0.3}{\includegraphics* {david.ps}} 
{\izhverylarge +} % православный крест
   %} \\
      \\
\vskip 15 mm
\large\MakeUppercase{Моли^твы и=збра^нныя}\\ 
\vskip -70mm}

% ==========================================================================================

\begin{document}
\maketitle
% ==========================================================================================

\newpage
\thispagestyle{empty} 
\phantom{abc}\vskip 1.75\baselineskip 
\begin{center}
  {\Huge+\par} 
  \vskip 0.5\baselineskip 
  %\small 
До'му твоему` подоба'етъ ст~ы'ня, гд\си, въ долготу` днi'й.
\end{center}
\newpage

% ==========================================================================================

\hdrcrosspage

\section{_Псало'мъ рк~s}

А='ще не гд\сь сози'ждетъ до'мъ, всу'е тру\-ди'\-ша\-ся зи'ждущiи: а='ще не гд\сь
со\-хра\-ни'тъ гра'дъ, всу'е бдjь` стрегi'й. Всу'е ва'мъ _e='сть о_у='треневати,
воста'нете по сjь\-дjь'\-нiи, ja=ду'щiи хлjь'бъ болjь'зни, _e=гда` да'стъ
воз\-лю'б\-л_eн\-нымъ свои^мъ со'нъ. Се` достоя'нiе гд\сне сы'нове, мзда` плода`
чре'внягw. JА='кw стрjь'лы въ руцjь` си'льнагw, та'кw сы'нове w\ттрясе'нныхъ.
Бл~же'нъ, и='же и=спо'лнитъ же\-ла'\-нiе свое` w\т ни'хъ: не постыдя'тся, _e=гда`
глаго'лютъ врагw'мъ свои^мъ во вратjь'хъ.

\section{_Псало'мъ рл~а}

Помяни`, гд\си, дв~да, и= всю` кро'тость _e=гw`: JА='кw кля'тся гд\сви,
w=бjьща'ся бг~у i=а'\-кwв\-лю: А='ще вни'ду въ селе'нiе до'му моегw`, и=ли`
взы'ду на _о='дръ посте'ли моея`: А='ще да'мъ со'нъ _о=чи'ма мои'ма, и=
вjь'ждома мои'ма дрема'нiе, и= поко'й скранiа'ма мои'ма: До'ндеже w=бря'щу
мjь'сто гд\сви, селе'нiе бг~у i=а'кwвлю. Се` слы'шахомъ я=` во _e=vфра'fjь,
w=брjьто'хомъ я=` въ поля'хъ дубра'вы: Вни'демъ въ сел_e'нiя _e=гw`,
поклони'мся на мjь'сто, и=дjь'же стоя'стjь но'зjь _e=гw`: Воскр\сни`, гд\си,
въ поко'й тво'й, ты` и= кiвw'тъ святы'ни твоея`. Сщ~е'нницы твои` w=блеку'тся
пра'вдою, и= прп\дбнiи твои` возра'дуются. Дв~да ра'ди раба` твоегw`, не
w\тврати` лице` пома'заннагw твоегw`. Кля'тся гд\сь дв~ду и='стиною, и= не
w\тве'ржется _e=я`: w\т плода` чре'ва твоегw` посажду` на пре\-сто'\-лjь твое'мъ:
А='ще сохраня'тъ сы'нове твои` за\-вjь'тъ мо'й, и= свидjь^нiя моя^ сiя^,
и=`мже научу` я=`, и= сы'нове и='хъ до вjь'ка ся'дутъ на престо'лjь твое'мъ.
JА='кw и=збра` гд\сь сiw'на, и=зво'ли и=` въ жили'ще себjь`. Се'й поко'й мо'й
во вjь'къ вjь'ка, здjь` вселю'ся, ja='кw и=зво'лихъ и=`. Лови'тву _e=гw`
бл~гословля'яй бл~гословлю`, ни'щыя _e=гw` насыщу` хлjь^бы: Сщ~е'нники _e=гw`
w=блеку` во сп~се'нiе, и= прп\дбнiи _e=гw` ра'достiю возра'дуются. Та'мw
возращу` ро'гъ дв~дови, о_у=гото'вахъ свjьти'льникъ пома'занному моему`.
Враги` _e=гw` w=блеку` студо'мъ: на не'мже процвjьте'тъ ст~ы'ня моя`.

\section{_Псало'мъ рл~в}

Се` что` добро`, и=ли` что` красно`, но _e='же жи'ти бра'тiи вку'пjь; JА='кw
мv'ро на гла\-вjь`, схо\-дя'\-щее на браду`, браду` а=арw'ню, схо\-дя'\-щее на w=ме'ты
_о=де'жды _e=гw`: JА='кw роса` а=ермw'нская сходя'щая на го'ры сiw^нскiя:
ja='кw та'мw заповjь'да гд\сь бл~гослове'нiе и= живо'тъ до вjь'ка.

\section{Стiхи'ра прест~jь'й бц\дjь держа'внjьй}

_W пресла'внагw чудесе`! нб~се` и= земли` цр~ца, w\т ст~ы'хъ сро'дникwвъ
на'шихъ о_у=моля'емая, доны'нjь зе'млю ру'сскую покрыва'етъ, и= ли'ка своегw`
и=зwбраже'нiями ми'лостивнw _w=бо\-га\-ща'\-етъ. _w вл\дчце держа'вная! не преста'ни
и= на бу'дущее врjь'мя, во о_у=твержде'нiе на руси` правосла'вiя, ми'лwсти и=
чудеса` и=злива'ти до вjь'ка. а=ми'нь.

\section[Тропа'рь ст~. влкмч. пантелеи'мону]
{Тропа'рь ст~о'му великому'ченику\\пантелеи'мону, гла'съ г~:}

Страстоте'рпче ст~ы'й и= цjьле'бниче пантелеи'моне, моли` мл\стиваго бг~а, да
прегрjьше'нiй w=ставле'нiе пода'стъ душа'мъ на'шымъ.

\section[Тропа'рь ст~. прв\д. i=wа'нну кроншта'дтскому]
{Тропа'рь ст~о'му прв\дному i=wа'нну кроншта'дтскому, гла'съ а~:}

Правосла'вныя вjь'ры побо'рниче, земли` рwс\-сi'й\-скiя печа'льниче, па'стыр_eмъ
пра'вило и= w='бразе вjь^рнымъ, покая'нiя и= жи'зни во хр\стjь`
проповjь'дниче, бж~е'ственныхъ та'инъ бл~гоговjь'йный служи'телю и=
дерзнове'нный w= лю'дехъ мл~твенниче, _о='ч~е прв\дный i=wа'нне, цjьли'телю
и= преди'вный чудотво'рче, гра'ду кроншта'дту похвало` и= цр~кве на'шея
о_у=\-кра\-ше'\-нiе, моли` всебл~га'го бг~а о_у=мири'ти мi'ръ и= сп~сти` ду'шы
на'шя.

\section[Тропа'рь ст~. бл~ж. _ксе'нiи петербу'ргстjьй]
{Тропа'рь ст~jь'й бл~же'ннjьй ма'тери _ксе'нiи петербу'ргстjьй, гла'съ з~:}

Нищету` хр\сто'ву возлюби'вши, без\-сме'рт\-ныя трапе'зы ны'нjь наслажда'ешися,
без\-у'\-мi\-емъ мни'мымъ безу'мiе мi'ра w=бличи'вши, сми\-ре'\-нiемъ кр\стнымъ си'лу
бж~iю восприня'ла _e=си`. сегw` ра'ди да'ръ чудодjь'йственныя по'\-мо\-щи
стяжа'вшая, _ксе'нiе бл~же'нная, моли` хр\ста` бг~а и=зба'витися на'мъ w\т
вся'кагw sла` покая'нiемъ.

\section[\cs Тропа'рь ст~. прп\дмч. корни'лiю]
{Тропа'рь ст~о'му прп\дбному'ченику корни'лiю пско'во-пече'рскому чудотво'рцу,
гла'съ s~:}

Пско'во-пече'рская w=би'тель \* и='здавна сла'в\-ная чудеса'ми i=кw'ны
бг~омт~рнiя, \* мнw'\-гiя и='ноки бг~ови воспита`. ту` и= прп\дбный кор\-ни'\-лiй \*
по'двигомъ до'брымъ подвиза'ся~~\* чу'д\-ную бг~омт~рь сла'вя, \* и=новjь^рныя
про\-свjь\-ща'я, \* и='нокwвъ и= мнw'гiя лю'ди сп~са'я,\,\* w=би'тель же свою` ди'внw
о_у=краша'я и= w=граж\-да'я, \* ту` и= му'ченичества вjьне'цъ \* по мно'гимъ
лjь'т_eмъ па'стырства своегw` до'б\-лест\-вен\-нjь прiя'тъ. \* тjь'мже возра'дуемся
лю'дiе \* хр\ста` бг~а и= преч\стую _e=гw` мт~рь воз\-бла\-го\-да\-ри'мъ, \* ja='кw
дарова` на'мъ прп\дбному'ченика сла'вна \* и= мл~твенника w= дш~а'хъ на'шихъ
достобл~же'нна.

\civil
\section[\civil Тропарь прп. Амвросию Оптинскому]
{Тропарь святому преподобному\\Амвросию Старцу Оптинскому}

Яко к целебному источнику, притекаем к тебе, Амвросие, отче наш, ты бо на
путь спасения нас верно наставляеши, молитвами от бед и напастей охраняеши, в
телесных и душевных скорбех утешаеши, паче же смирению, терпению и любви
научаеши, моли Человеколюбца Христа и Заступницу Усердную спастися душам
нашим.

\cs
\section[\cs Покая'нная моли'тва:]
{Покая'нная моли'тва, ю='же что'ша въ це'рквахъ въ рwссi'и во дни^ сму'ты}

Гд\си, бж~е, вседержи'телю, при'зри на на'съ грjь'шныхъ и= недосто'йныхъ
ча^дъ тво\-и'хъ, согрjьши'вшихъ предъ тобо'ю, про\-гнjь'\-вав\-шихъ бл~гость твою`,
навле'кшихъ гнjь'въ тво'й пра'ведный на ны`, па'дшихъ во глу\-би\-ну` грjьхо'вную.
ты` зри'ши, гд\си, не'\-мощь на'шу и= ско'рбь душе'вную, вjь'си рас\-тлjь'\-нiе
о_у=мw'въ и= серде'цъ на'шихъ, w=скудjь'нiе вjь'ры, w\тступле'нiе w\т
за'повjьдей твои'хъ, о_у=\-мно\-же'\-нiе нестрое'нiй семе'йныхъ, раз\ъ\-е\-ди\-н_e'\-нiя и=
раздо'ры церкw'вныя, ты зри'\-ши пе\-ча^\-ли и= скw'рби на'шя, w\т болjь'зней,
гла'\-дwвъ, потопле'нiя, запале'нiя и= меж\-до\-у\-со'б\-ныя бра'ни происходя'щыя. но
премл\стивый и= человjьколюби'вый гд\си, вразуми`, наста'ви и= поми'луй на'съ,
недосто'йныхъ. и=спра'ви жи'знь на'шу грjьхо'вную, о_у=толи` раздо'ры и=
не\-стро\-{_е}'\-нiя, собери` расточ_e'нныя, соедини` раз\-сjь^\-ян\-ныя, пода'ждь ми'ръ
странjь` на'шей и= бл~годе'нствiе, и=зба'ви ю=` w\т вся'кихъ бjь'дъ и=
несча'стiй. всест~ы'й вл\дко, просвjьти` ра'зумъ на'шъ свjь'томъ о_у=че'нiя
_e=v\гльскагw, возгрjь'й сердца` на'ша теплото'ю блг\дти твоея` и= напра'ви
я=` къ дjь'ланiю за'повjьдей твои'хъ, да просла'вится въ на'съ всест~о'е и=
пресла'вное и='мя твое`, _о=ц~а` и= сн~а и= ст~а'гw дх~а, ны'нjь и= во вjь'ки
вjькw'въ. А=ми'нь.

\section{\cs Мл~тва держа'внjьй бж~iей мт~ри}

_W мi'ра засту'пнице, мт~и всепjь'тая! со стра'хомъ, вjь'рою и= любо'вiю
припа'дающе предъ ч\стно'ю i=кw'ною твое'ю держа'вною, о_у=\-се'рд\-нw мо'лимъ
тя`: не w\тврати` лица` твоегw` w\т прибjьга'ющихъ къ тебjь`. о_у=моли`,
мл\срдная мт~и свjь'та, сн~а твоего` и= бг~а на'\-ше\-го, сладча'йшаго гд\са
i=и~са хр\ста`: да со\-хра\-ни'тъ въ ми'рjь страну` на'шу, да о_у=\-твер\-ди'тъ
держа'ву на'шу въ бл~годе'нствiи и= и=зба'витъ на'съ w\т междоусо'бныя бра'ни:
да о_у=крjьпи'тъ ст~у'ю цр~ковь на'шу пра\-во\-сла'в\-ную, и= непозы'блему
соблюде'тъ ю=` w\т не\-вjь'\-рiя, раско'ла и= _e='ресей. не и='мамы бо и=ны'я
по'мощи, ра'звjь тебе`, преч\стая дв~о: ты` _e=си` всеси'льная хр\стiа'нъ
засту'пница предъ бг~омъ, пра'ведный гнjь'въ _e=гw` о_у=мягча'ющая. и=зба'ви
всjь'хъ съ вjь'рою тебjь` моля'щихся, w\т паде'нiй грjьхо'вныхъ, w\т навjь'та
sлы'хъ человjь^къ, w\т гла'да, ско'рбей и= болjь'зней. да'руй на'мъ ду'хъ
сокруше'нiя, смире'нiе се'рдца, ч\сто\-ту` помышле'нiй, и=справле'нiе
грjьхо'вныя жи'зни и= w=ставле'нiе согрjьше'нiй на'шихъ: да вси`, бл~года'рнjь
воспjьва'юще вели^чiя твоя^, сподо'бимся небе'снагw ца'рствiя, и= та'мw со
всjь'ми ст~ы'ми просла'вимъ преч\стно'е и= великолjь'пое и='мя въ тр\оцjь
сла'вимагw бг~а: _о=ц~а`, сн~а и= ст~а'гw дх~а. А=ми'нь.

\civil
\section[\civil Молитва св. вмч. Пантелеимону]
{Молитва святому великомученику Пантеле\'имону}

О великий Христов угодниче и преславный целебниче, великомучениче
Пантеле\'имоне! Ду\-ш\'ею на Небес\`и Престолу Божию предстояй и триипостасныя
Его славы наслаждаяйся, телом же и ликом святым на земл\`и в Божественных
хр\'амех почиваяй и данною ти свыше благодатию различныя чудеса источаяй,
призри милостивным твоим оком на предстоящия люди, честней твоей иконе умильно
молящияся и просящия от тебе целебныя помощи и заступления: простр\`и ко
Господу Богу нашему теплыя твоя молитвы и испрос\`и душ\'ам нашим оставление
согрешений. С\`е бо мы, за беззакония наша не смеюще возвести очеса наша к
высоте небесней, ниж\`е вознест\`и глас молебный к Его в Божестве
неприст\'упней сла\-ве, сердцем сокрушенным и духом смиренным тебе, ход\'атая
милостива ко Владыце и молитвенника за ны, грешныя, призываем, яко ты приял
еси благодать от Него нед\'уги отгоняти и страсти исцеляти. Тебе убо просим: не
пр\'езри нас, недостойных, молящихся тебе и твоея помощи требующих. Буди нам в
печалех ут\'ешитель, в недузех лютых страждущим врач, нап\'аствуемым скорый
покровитель, очесем недугующим прозр\'ения датель, ссущим и младенцем в скорбех
готовейший предстатель и исцелитель: исходатайствуй всем вся, яже ко спасению
полезная, яко да твоими ко Господу Богу молитвами получивше благодать и
милость, прославим всех благих Источника и Дароподателя Бога, Единаго в Троице
Святей славимаго Отца и Сына и Святаго Духа ныне и присно и во веки веков.
Аминь.

\section[\civil Молитва св. прав. Иоанну Кронштадтскому]
{Молитва святому праведному\\отцу Иоанну Кронштадтскому}

О великий угодниче Христов, святый праведный отче Иоанне Кронштадтский,
пастырю дивный, скорый помощниче и милостивый пред\-ста\-те\-лю!

Вознося славословие Триединому Богу, ты молитвенно взывал:

Имя Тебе - Любовь: не отвергни меня, заблуждающагося.

Имя Тебе - Сила: укрепи меня, изнемогающаго и падающаго.

Имя Тебе - Свет: просвети душу мою, омраченную житейскими страстями.

Имя Тебе - Мир: умири мятущуюся душу мою.

Имя Тебе - Милость: не переставай миловать меня.

Ныне благодарная твоему предстательству всероссийская паства молится тебе:

Христоименитый и праведный угодниче Божий!

Любовию твоею озари нас, грешных и немощных, сподоби нас принести достойные
плоды покаяния и неосужденно причащатися Святых Христовых Таин. Силою твоею
веру в нас укрепи, в молитве поддержи, недуги и болезни исцели, от напастей,
врагов, видимых и невидимых, избави.

Светом лика твоего служителей и предстоятелей Алтаря Христова на святыя
подвиги пастырского делания подвигни, младенцем воспитание даруй, юность
настави, старость поддержи, святыни храмов и святые обители озари!

Умири, чудотворче и провидче преизряднейший, народы страны нашея, благодатию и
даром Святаго Духа избави от междоусобныя брани, расточенная собери,
прельщенныя обрати и совокупи Святей Соборней и Апостольстей Церкви.

Милостию твоею супружества в мире и единомыслии соблюди, монашествующим в
делах благих преуспеяние и благословение даруй, малодушныя утеши, страждущих
от духов нечистых свободи, в нуждах и обстояниих сущих помилуй и всех нас на
путь спасения настави.

Во Христе живый, отче наш Иоанне, приведи нас к Невечернему Свету жизни
вечныя, да сподобимся с тобою вечнаго блаженства, хваляще и превозносяще Бога
во веки веков. Аминь.

\section[\civil Молитва св. блаж. Ксении Петербургской]
{Молитва святой блаженной\\матери Ксении Петербургской}

О Святая всеблаженная мати Ксение! Под кровом Всевышняго жившая, вед\'омая и
укрепляемая Богоматерью, голод и жажду, холод и зной, поношения и гонения
претерпевшая, дар прозорливости и чудотворения от Бога получила еси и под
сенью Всемогущаго покоишися. Ныне святая Церковь, яко благоуханный цвет,
прославляет тя. Предстояще на месте твоего погребения, пред образом твоим
святым, яко живей ти, сущей с нами, молимся ти: приими прошения наша и принеси
их ко Престолу Милосерднаго Отца Небеснаго, яко дерзновение к Нему имущая,
испроси притекающим к тебе вечное спасение, на благая дела и начинания наша
щедро благословение, от всяких бед и скорбей избавление. Предстани святыми
твоими молитвами пред Всемилостивым Спасителем нашим о нас, недостойных и
грешных. Помози, святая блаженная мати Ксение, младенцы светом Святого
Крещения озарити и печатию дара Духа Святаго запечатлети, отроки и отроковицы
в вере, честности, богобоязненности воспитати и успехи в учении им даровати;
болящия и недугующия исцели, семейным любовь и согласие ниспосли,
монашествующих подвигом добрым подвизатися удостой и от поношений огради,
пастыри в крепости Духа Святаго утверди, народ и страну нашу в мире и
безмятежии сохрани, о лишенных в предсмертный час причащения Святых Христовых
Таин умоли. Ты наша надежда и упование, скорое услышание и избавление, тебе
благодарение воссылаем и с тобою славим Отца и Сына и Святаго Духа ныне и
присно и во веки веков. Аминь.

\cs
%\clearpage
\raggedbottom
\section[\cs Мл~тва ст~. прп\дмч. корни'лiю]
{Мл~тва ст~о'му прп\дбномч~нику корни'лiю пско'во-пече'рскому чудотво'рцу}

Ст~ы'й прп\дбномч~ниче корни'лiе, смире'ннw припа'даемъ къ ра'цjь моще'й
твои'хъ и= о_у=\-ми'ль\-нw мо'лимъ тя`: ми'лостивw при'зри на скw'рби на'шя
дш~_e'вная и= тjьл_e'сная и= и=зба'ву на'мъ пода'ждь: помози` на'мъ, свя'т\-че
бж~iй, w\т навjь'та sлы'хъ че\-ло\-вjь^къ, w\т ни'хже и= са'мъ ты` неви'ннw на
земли` пострада'лъ _e=си`. Защити` на'съ w\т наси'лiя дiа'вола, стра'стнw
вою'ющагw въ не'мощныхъ о_у=десjь'хъ на'шихъ: о_у=моли` гд\са бг~а и=
преч\стую _e=гw` мт~рь да\-ро\-ва'ти на'мъ ти'хое и= безгрjь'шное житiе`,
вза\-и'м\-ную бра'тскую нелицемjь'рную любо'вь и= ми'рную хр\стiа'нскую кончи'ну,
да чи'стою со'\-вjь\-стiю предста'немъ нелицемjь'рному стра'ш\-ному суди'лищу
хр\сто'ву, и= въ цр\ствiи _e=гw` просла'вимъ животворя'щую тр\оцу, _о=ц~а` и=
сн~а и= ст~а'гw дх~а, во вjь'ки вjькw'въ. А=ми'нь.

\civil

\section[\civil Молитва прп. Амвросию Оптинскому]
{Молитва преподобному Амвросию Старцу Оптинскому}

Великий Старче и Угодниче Божий, преподобне Отче наш Амвросие, Оптинская
похвал\'о и Всея Руси учителю благочестия! Славим твое во Христе смиренное
житие, имже Бог превознес\'е имя твое еще на земли тебе суща, наипаче же увенча
тя небесною честию по отшествии твоем в чертог славы вечныя. Приими ныне
моление нас, недостойных чад твоих, чтущих тя и призывающих имя твое святое,
избави нас твоим предстательством пред Престолом Божиим от всех скорбных
обстояний, душевных и телесных недугов, злых напастей, тлетворных и лукавых
искушений, низпосли Отечеству нашему от Великодаровитаго Бога мир, тишину и
благоденствие, буди непреложный покровитель Святыя Обителя сия, в нейже в
преуспеянии сам подвизался еси и угодил еси всеми в Троице славимому Богу
нашему, Емуже подобает всякая слава, честь и поклонение, Отцу и Сыну и Святому
Духу, ныне и присно и во веки веков. Аминь.

\csendpictsmall
\clearpage
%\flushbottom

\section[\civil Молитва задержания]{\large Молитва задержания}

\begin{center}
\textit{\footnotesize(из сборника молитв старца Пансофия\\ Афонского 1848 г.)}
\vskip -1\baselineskip
\end{center}

%\begin{group}
  
\vskip 0.5\baselineskip
{\footnotesize
\leftskip = 0.4\textwidth
Сила сих молитв в утаении от слуха и взора людского, в тайнодействовании ея.\par}
\vskip 0.5\baselineskip

Яко неплодную смоковницу, не посецы мене, Спасе, грешнаго, но на многая лета
пождание ми даруй, напаяя душу слезами покаяния, да плод принесу Ти,
Многомилостиве.

\subsection{\bfseries Молитва задержания}

Милосердый Господи! Ты некогда устами служителя Моисея, Иисуса, сына Навина,
задерживал целый день движение солнца и луны, доколе народ Израильский мстил
врагам своим.

Молитвой Елисея пророка некогда поразил сириян, задерживая их, и вновь исцелил
их.

Ты некогда вещал пророку Исайи: вот, Я возвращу назад на десять ступеней
солнечную тень, которая прошла по ступеням Ахазовым, и возвратилось солнце на
десять ступеней по ступеням, по которым оно сходило.

Ты некогда устами пророка Иезекииля затворил бездны, останавливал реки,
задерживал воды.

И Ты некогда постом и молитвою пророка Твоего Даниила заграждал уста львов во
рву.

И ныне задержи и замедли до благовремения все замыслы окрест стоящих мя о моем
перемещении, увольнении, смещении, изгнании.

Так и ныне разруши злые хотения и требования всех осуждающих меня, загради
уста и сердца всех клевещущих, злобствующих и рыкающих на меня и всех хулящих
и унижающих мя.

Так и ныне наведи духовную слепоту на глаза всех восстающих на мя и на врагов
моих.

Не Ты ли вещал апостолу Павлу: говори и не умолкай, ибо Я с тобою, и никто не
сделает тебе зла.

Смягчи сердца всех противоборствующих благу и достоинству Церкви Христовой.

Посему да не умолкнут уста моя для обличения нечестивых и прославления
праведных и всех дивных дел Твоих. И да исполнятся вся благая начинания наша и
желания.

К вам, праведницы и молитвенницы Божии, наши дерзновении предстателие, некогда
силою своих молитв сдерживавшие нашествия иноплеменников, подход ненавидящих,
разрушившие злые замыслы людей, заграждавшие уста львов, ныне обращаюсь я с
молитвой моей, с моим прошением.

И Ты, преподобный великий Еллий Египетский, некогда оградивший в круге
крестным знамением место поселения ученика своего, повелел ему вооружиться
именем Господним и не бояться отныне демонских искушений. Огради дом мой, в
коем я живу, в круге молитв твоих и сохрани его от огненного запаления,
воровского нападения и всякого зла и страхования.

И Ты, преподобный отче Поплие Сирийский, некогда своею непрестанною молитвою
десять дней демона державший неподвижным и не могущим идти ни днем ни ночью;
ныне окрест келии моей и дома (моего) сего удержи за оградою его вся
сопротивныя силы и всех хулящих имя Божие и презирающих мя.

И Ты, преподобная девственница Пиама, некогда силою молитвы остановившая
движение шедших погубить жителей тоя деревни, где жила, ныне приостанови все
замыслы врагов моих, хотящих изгнати мя из града сего и погубити мя: не
допускай им приближатися к дому сему, останови их силою молитвы своей:
<<Господи, Судия Вселенной, Ты которому неугодна всякая неправда, когда приидет
к Тебе молитва сия, пусть Святая Сила остановит их на том месте, где постигнет
их>>.

И Ты, блаженный Лаврентий Калужский, моли Бога о мне, как имеющий дерзновение
пред Господом предстательствовать о страждущих от козней диавольских. Моли
Бога о мне, да оградит Он меня от козней сатанинских.

И Ты, преподобный Василий Печерский, соверши свои молитвы запрещения над
нападающими на меня и отжени все козни диавольские от мене.

И вы, вси Святые земли Российской, развейте силою молитв своих обо мне все
бесовские чары, все диавольские замыслы и козни -- досадить мне и погубить меня
и достояние мое.

И Ты, великий и грозный страже, Архистратиже Михаиле, огненным мечем посекаяй
все хотения врага рода человеческаго и всех приспешников его, хотящих погубить
мя. Стой нерушимо на страже дома сего, всех живущих в нем и всего достояния
его.

И Ты, Владычице, не напрасно именуемая <<Нерушимой стеной>>, будь для всех
враждующих против меня и замышляющих пакостная творити мне, воистину некоей
преградой и нерушимой стеной, ограждающей меня от всякого зла и тяжких
обстояний.

\subsection{\bfseries Молитва преподобного Макария, аввы Египетского}

Господи, как Ты хочешь, и как знаешь -- помилуй.

Но да будет тако! Да будет тако! Да будет тако!

\cs
\subsection{Догма'тикъ, гла'съ а~:}

Всемi'рную сла'ву, w\т человjь^къ прозя'бшую, и= вл\дку ро'ждшую, небе'сную
две'рь воспои'мъ мр~i'ю дв~у, безпло'тныхъ пjь'снь, и= вjь'рныхъ
о_у=добре'нiе: сiя' бо ja=ви'ся нб~о, и= хра'мъ бж~е\-ства`: сiя` прегражде'нiе
вражды` разруши'вши, ми'ръ введе`, и= цр\ствiе w\тве'рзе, сiю` о_у='бw и=му'\-ще
вjь'ры о_у=твержде'нiе, побо'рника и='мамы и=зъ нея` ро'ждшагося гд\са.
дер\-за'й\-те о_у='бw, дер\-за'й\-те лю'дiе бж~iи: и='бо то'й победи'тъ вра\-ги`, ja='кw
всеси'ленъ.

\subsection{Млт~ва нб\снымъ си'ламъ}

\subsubsection{Тропа'рь, гла'съ д~:}

Нб\сныхъ во'инствъ а=рхiстрати'зи, мо'лимъ ва'съ при'снw, мы` недосто'йнiи, да
ва'шими мо\-ли'т\-ва\-ми, w=градите` на'съ кро'вомъ крилу` невеще'ственныя ва'шея
сла'вы, сохраня'юще на'съ припа'дающихъ прилjь'жнw, и= во\-пi\-ю'\-щихъ: w\т бjь'дъ
и=зба'вите на'съ, ja='кw чи\-но\-на\-ча'ль\-ни\-цы вы'шнихъ си'лъ.

\subsubsection{Конда'къ, гла'съ в~:}

А=рхiстрати'зи бж~iи, служи'телiе бж~е'ст\-вен\-ныя сла'вы, а='гг~лwвъ
нача^льницы и= че\-ло\-вjь'\-кwвъ наста^вницы, поле'зное на'мъ про\-си'\-те, и= ве'лiю
мл\сть, ja='кw безпло'тныхъ а=рхiстрати'зи.

\vskip -2.4\baselineskip
\csendpictsmall

\end{document}

% Тропарь святому праведному Иоанну Кронштадтскому, глас 1-й

% Православныя веры поборниче, земли Российския печальниче, пастырем правило и
% образе верным, покаяния и жизни во Христе проповедниче, Божественных Таин
% благоговейный служителю и дерзновенный о людех молитвенниче, отче праведный
% Иоанне, целителю и предивный чудотворче, граду Кронштадту похвало и церкве
% нашея украшение, моли всеблагаго Бога умирити мир и спасти души наша.

