\documentclass[12pt,a6paper,twoside,dvips,civil=antiqua,cs=izhitsa]{hipbook}
\usepackage[russian,churchslavonic]{babel}
\tolerance 500

% ==========================================================================================

\stdcrosstitle{
_ПСАЛТИРЬ \\
\vskip 5 mm
Каfi'сма s~_i}

% ==========================================================================================

\begin{document}

% ==========================================================================================

\maketitle

%\raggedbottom

% ==========================================================================================

\stdsecondpage

% ==========================================================================================

\hdrcrosspage

\selectlanguage{churchslavonic}
\cs % эту команду необходимо выполнить в этом месте, чтобы вернуть тильду и пр. в режим cs.

\section{{\Large Разу'мно да бу'детъ,}}
\section[Разу'мно да бу'детъ,]{ка'кw подоба'етъ _о=со'бь пjь'ти _псалти'рь.}
\subsection[ка'кw подоба'етъ _о=со'бь пjь'ти _псалти'рь.]{}
\vskip -0.5\baselineskip

\subsubsection{А='ще _i=ере'й, глаго'летъ:}

\bukv{Б}л~гослове'нъ бг~ъ на'шъ, всегда`, ны'нjь и= при'снw, и= во
вjь'ки вjькw'въ.

\subsubsection{А='ще ли ни`, глаго'ли о_у=миле'ннw:}

\bukv{М}л~твами ст~ы'хъ _о=т_e'цъ на'шихъ, гд\си, _i=и~се хр\сте`,
бж~е на'шъ, поми'луй на'съ, а=ми'нь.

\bukv{Ц}р~ю` нб\сный: \rem{Трист~о'е.} \bukv{О='}ч~е на'шъ:

\subsubsection{И= тропари` сiя^, гла'съ s~:}

\bukv{П}оми'луй на'съ, гд\си, поми'луй на'съ: вся'кагw бо w\твjь'та
недоумjь'юще, сiю' ти мл~тву, ja='кw вл\дцjь, грjь'шнiи прино'симъ:
поми'луй на'съ.

\rem{С__л__а'__в__а:} Ч\Стно'е пр\оро'ка твоегw`, гд\си, торжество`,
нб~о цр~ковь показа`, съ человjь'ки лику'ютъ а='гг~ли: тогw`
мл~твами хр\сте` бж~е, въ ми'рjь о_у=пра'ви живо'тъ на'шъ, да пое'мъ
ти`: а=ллилу'iа.

\rem{И=__ н__ы'__н__jь:} Мнw'гая мно'ж_eства мои'хъ бц\де,
прегрjьше'нiй, къ тебjь` прибjьго'хъ ч\стая, сп~се'нiя тре'буя:
посjьти` немощству'ющую мою` ду'шу, и= моли` сн~а твоего`,
и= бг~а на'шего, да'ти ми` w=ставле'нiе, ja=`же содjь'яхъ
лю'тыхъ, _e=ди'на бл~гослове'нная.

\vskip 0.5\baselineskip
\centerline{\bukv{Г}д\си поми'луй, \rem{м~}.}
\vskip 0.5\baselineskip

\centerline{\rem{И= поклони'ся, _e=ли'кw ти` мо'щно.}}

\subsubsection{Та'же мл~тва ст~jь'й живонача'льнjьй тр\оцjь:}

\bukv{В}сест~а'я тр\оце, бж~е и= содjь'телю всегw` м_i'ра, поспjьши`
и= напра'ви се'рдце мое` нача'ти съ ра'зумомъ, и= конча'ти дjь'лы
бл~ги'ми бг~одухнов_e'нныя сiя^ кни^ги, ja=`же ст~ы'й дх~ъ
о_у=сты^ дв~дwвы w\тры'гну, и=`хже ны'нjь хощу` глаго'лати
а='зъ недосто'йный. разумjь'я же свое` невjь'жество, припа'дая молю'ся
ти`, и= _e='же w\т тебе` по'мощи прося`: гд\си, о_у=пра'ви о_у='мъ
мо'й, и= о_у=тверди` се'рдце мое`, не w= глаго'ланiи о_у=сте'нъ
стужа'ти си`, но w= ра'зумjь глаго'лемыхъ весели'тися, и=
пригото'витися на творе'нiе до'брыхъ дjь'лъ, ja=`же
о_у=чу'ся, и= глаго'лю: да до'брыми дjь'лы просвjьще'нъ, на суди'щи
десны'я ти` страны` прича'стникъ бу'ду со всjь'ми и=збра'нными
твои'ми. и= ны'нjь вл\дко, бл~гослови`, да воздохну'въ w\т се'рдца, и=
я=зы'комъ воспою`, глаго'ля си'це:

\begin{center}
  %\vskip -0.5\baselineskip
  \bukv{П}рiиди'те, поклони'мся: \rem{три'жды.}
  %\vskip -0.5\baselineskip
\end{center}

{\small \color[named]{Red} Та'же посто'й ма'лw, до'ндеже о_у=тиша'тся
  вся^ чу^вства: тогда` сотвори` нача'ло не вско'рjь, без\ъ
  лjь'ности, со о_у=миле'нiемъ и= сокруше'ннымъ се'рдцемъ. Рцы` сiе`:
  {\color[named]{Black} \normalfont {Б}ла'го _e='сть и=сповjь'датися
    гд\сви:} ти'хw и= разу'мнw, со внима'нiемъ, а= не борзя'ся,
  ja='коже и= о_у=мо'мъ разумjьва'ти глаго'л_eмая.}

%\vskip 2.0\baselineskip
\csendpict

\clearpage
\hdrcrosspage

\section{{\Large Дв~да пр\оро'ка и= цр~я` пjь'снь}}
%\vskip -0.3\baselineskip
\section[Каf_i'сма s~_i]{Каf_i'сма s~_i:}
%\vskip -0.3\baselineskip

\subsection{_Псало'мъ дв~ду, р~f}

\bukv{Р}ече` гд\сь гд\сви моему`: сjьди` w=десну'ю мене`, до'ндеже  положу`
враги` твоя^ подно'жiе но'гъ твои'хъ. Же'злъ си'лы по'слетъ ти` гд\сь w\т
сiw'на, и= гд\сствуй посредjь` врагw'въ твои'хъ. Съ тобо'ю нача'ло въ де'нь
си'лы твоея`, во свjь'тлостехъ ст~ы'хъ твои'хъ: и=з\ъ чре'ва пре'жде денни'цы
роди'хъ тя`. Кля'тся гд\сь, и= не раска'ется: ты` i=ере'й во вjь'къ по чи'ну
мелхiседе'кову. Гд\сь w=десну'ю тебе` сокруши'лъ _е='сть въ де'нь гнjь'ва
своегw` цари`: Су'дитъ во jа=зы'цjьхъ, и=спо'лнитъ пад_е'нiя, сокруши'тъ
главы^ на земли` мно'гихъ. W\т пото'ка на пути` пiе'тъ: сегw` ра'ди вознесе'тъ
гла'ву.

\delimpict

\subsection{А=ллилу'iа, р~_i}

\bukv{И=}сповjь'мся тебjь`, гд\си, всjь'мъ се'рдцемъ мои'мъ, в совjь'тjь
пра'выхъ и= со'нмjь. В_е'лiя дjьла` гд\сня, и=зы^скана во всjь'хъ во'ляхъ
_е=гw`: И=сповjь'данiе и= великолjь'пiе дjь'ло _е=гw`: и= пра'вда _е=гw`
пребыва'етъ въ вjь'къ вjь'ка. Па'мять сотвори'лъ _е='сть чуде'съ свои'хъ:
мл\стивъ и= ще'дръ гд\сь. Пи'щу даде` боя'щымся _е=гw`: помяне'тъ въ вjь'къ
завjь'тъ сво'й: Крjь'пость дjь'лъ свои'хъ возвести` лю'демъ свои^мъ, да'ти
и=`мъ достоя'нiе jа=зы^къ. Дjьла` ру'къ _е=гw` и='стина и= су'дъ: вjь^рны вся^
за'повjьди _е=гw`, О_у=тверж_е'ны въ вjь'къ вjь'ка, сотвор_е'ны во
и='стинjь и= правотjь`. И=збавле'нiе посла` лю'демъ свои^мъ, заповjь'да въ
вjь'къ завjь'тъ сво'й: ст~о и= стра'шно и='мя _е=гw`. Нача'ло
прему'дрости стра'хъ гд\сень, ра'зумъ же бл~гъ всjь^мъ творя'щымъ и=`: хвала`
_е=гw` пребыва'етъ въ вjь'къ вjь'ка.

\delimpict

\subsection{А=ллилу'iа, а=гге'ево и= заха'рiино, _псало'мъ ра~_i}

\bukv{Б}л~же'нъ му'жъ боя'йся гд\са, въ за'повjьдехъ _е=гw` восхо'щетъ
sjьлw`. Си'льно на земли` бу'детъ сjь'мя _е=гw`, ро'дъ пра'выхъ
бл~гослови'тся: Сла'ва и= бога'тство въ дому` _е=гw`, и= пра'вда _е=гw`
пребыва'етъ въ вjь'къ вjь'ка. Возсiя` во тьмjь` свjь'тъ пра^вымъ: мл\стивъ и=
ще'дръ и= прв\днъ. Бл`гъ му'жъ ще'дря и= дая`, о_у=стро'итъ словеса` своя^ на
судjь`, jа='кw въ вjь'къ не подви'жится. Въ па'мять вjь'чную бу'детъ
прв\дникъ. W\т слу'ха sла` не о_у=бои'тся. Гото'во се'рдце _е=гw` о_у=пова'ти
на гд\са: о_у=тверди'ся се'рдце _е=гw`, не о_у=бои'тся, до'ндеже воззри'тъ на
враги` своя^. Расточи` даде` о_у=бw'гимъ: пра'вда _е=гw` пребыва'етъ во вjь'къ
вjь'ка: ро'гъ _е=гw` вознесе'тся въ сла'вjь. Грjь'шникъ о_у='зритъ и=
прогнjь'вается, зубы` свои'ми поскреже'щетъ и= раста'етъ: жела'нiе грjь'шника
поги'бнетъ.

\slava

\delimpict

\subsection{А=ллилу'iа, рв~_i}

\bukv{Х}вали'те _о='троцы гд\са, хвали'те и='мя гд\сне, Бу'ди и='мя гд\сне,
бл~гослове'но w\т ны'нjь и= до вjь'ка. W\т востw'къ с'олнца до за^падъ
хва'льно и='мя гд\сне. Высо'къ над\ъ всjь'ми jа=зы'ки гд\сь: над\ъ нб~сы`
сла'ва _е=гw`. Кто` jа='кw г\дсь бг~ъ на'шъ; на высо'кихъ живы'й, И= на
смир_е'нныя призира'яй на нб~си и= на земли`: Воздвиза'яй w\т земли` ни'ща, и=
w\т гно'ища возвыша'яй о_у=бо'га: Посади'ти _е=го` съ кня'зи, съ кня'зи люде'й
свои'хъ. Вселя'я непло'довь въ до'мъ, ма'терь w= ча'дjьхъ веселя'щуся.

\delimpict

\subsection{А=ллилу'iа, рг~_i}

\bukv{В}о и=схо'дjь i=и~левjь w\т _е=гv'пта, до'му i=а'кwвля и=з\ъ люде'й
ва^рваръ, Бы'сть i=уде'а ст~ы'ня _е=гw`, i=и~ль _о='бласть _е=гw`. Мо'ре ви'де
и= побjьже`, i=_орда'нъ возврати'ся вспя'ть: Го'ры взыгра'шася jа='кw _о=вни`,
и= хо'лми jа='кw а='гнцы _о='вчiи. Что' ти _е='сть мо'ре, jа='кw побjь'гло
_е=си`; и= тебjь` i=_орда'не, jа='кw возврати'лся _е=гw` вспя'ть; Го'ры,
jа='кw взыгра'стеся jа='кw _о=вни` и= хо'лми jа='кw а='гнцы _о='вчiи, W\т
лица` гд\сня подви'жеся земля`, w\т лица` бг~а i=а'кwвля: W=бра'щшагw ка'мень
во _е=зе'ра водна'я, и= несjько'мый во и=сто'чники вwдны'я. Не на'мъ, гд\си,
не на'мъ, но и='мени твоему` да'ждь сла'ву w= мл\сти твое'й и= и='стинjь
твое'й: Да не когда` реку'тъ jа=зы'цы: гдjь` _е='сть бг~ъ и='хъ, Бг~ъ же на'шъ
на нб~си` и= на земли`, вся^ _е=ли^ка восхотjь` сотвори`. I='дwли jа=зы^къ
сребро` и= зла'то, дjьла` ру'къ человjь'ческихъ: О_у=ста` и='мутъ, и= не
возглаго'лютъ: _о='чи и='мутъ, и= не о_у='зрятъ: О_у='ши и='мутъ и= не
о_у=слы'шатъ: нw'здри и='мутъ, и= не w=боня'ютъ: Ру'цjь и='мутъ, и= не
w=ся'жутъ: но'зjь и='мутъ, и= не по'йдутъ: не возглася'тъ горта'немъ
свои'мъ. Подо'бни и=`мъ да бу'дутъ творя'щiи я=`, и= вси` надjь'ющiися на
ня`. До'мъ i=и~левъ о_у=пова` на гд\са: помо'щникъ и= защи'титель и=`мъ
_е='сть. До'мъ а=арw'нь о_у=пова` на гд\са: помо'щникъ и= защи'титель и=`мъ
_е='сть. Боя'щiися гд\са, о_у=пова'ша на гд\са, помо'щникъ и= защи'титель
и=`мъ _е='сть. Гд\сь помяну'въ ны` бл~гослови'лъ _е='сть на'съ: бл~гослови'лъ
_е='сть до'мъ i=и~левъ, бл~гослови'лъ _е='сть до'мъ а=арw'нь: Бл~гослови'лъ
_е='сть боя'щiися гд\са, ма^лыя съ вели'кими. Да приложи'тъ гд\сь на вы`, на
вы`, и= на сы'ны ва'шя. Бл~гослове'ни вы` гд\сви, сотво'ршему нб~о и=
зе'млю. Нб~о нб~се` гд\сви, зе'млю же даде` сыновw'мъ человjь'ческимъ. Не
ме'ртвiи восхва'лятъ тя` гд\си, ниже` вси` нисходя'щiи во а='дъ: Но мы` живi'и
бл~гослови'мъ гд\са, w\т ны'нjь и= до вjь'ка.

\delimpict

\subsection{А=ллилу'iа, рд~_i}

\bukv{В}озлюби'хъ, jа='кw о_у=слы'шитъ гд\сь гла'съ моле'нiя моегw`, JА='кw
приклони` о_у='хо свое` мнjь`: и= во дни^ моя^ призову`. W=б\ъя'ша мя`
болjь^зни см_е'ртныя, бjьды^ а='довы w=брjьто'ша мя`, ско'рбь и= болjь'знь
w=брjьто'хъ: и= и='мя гд\сне призва'хъ: _W гд\си, и=зба'ви ду'шу
мою`. мл\стивъ гд\сь и= прв\днъ, и= бг~ъ на'шъ ми'луетъ. Храня'й младе'нцы
гд\сь: смири'хся, и= спасе' мя. W=брати'ся душе` моя` въ поко'й тво'й, jа='кw
гд\сь бл~годjь'йствова тя`. JА='кw и=з\ъя'тъ ду'шу мою` w\т сме'рти, _о='чи
мои` w\т сле'зъ, и= но'зjь мои` w\т поползнове'нiя. Бл~гоугожду` пред\ъ
гд\семъ во странjь` живы'хъ.

\slava

\delimpict

\subsection{А=ллилу'iа, р_е~_i}

\bukv{В}jь'ровахъ, тjь'мже возглаго'лахъ: а='зъ же смири'хся
sjьлw`. А='зъ же рjь'хъ во и=зступле'нiи мое'мъ: вся'къ
человjь'къ ло'жъ. Что` возда'мъ гд\сви w= всjь'хъ, ja='же
воздаде' ми; ча'шу сп~се'нiя прiиму`, и= и='мя гд\сне
призову`, мл~твы моя^ гд\сви возда'мъ пред\ъ всjь'ми людьми`
_e=гw`. Ч\Стна` пред\ъ гд\семъ сме'рть прп\дбныхъ _e=гw`. _W гд\си,
а='зъ ра'бъ тво'й, а='зъ ра'бъ тво'й, и= сы'нъ рабы'ни твоея`:
растерза'лъ _e=си` о_у='зы моя^. Тебjь` пожру` же'ртву хвалы`,
и= во и='мя гд\сне призову`: мл~твы моя^ гд\сви возда'мъ пред\ъ
всjь'ми людьми` _e=гw`: Во дво'рjьхъ до'му гд\сня, посредjь`
тебе`, i=ер\сли'ме.

\delimpict

\subsection{А=ллилу'iа, рs~_i}

\bukv{Х}вали'те гд\са вси` jа=зы'цы, похвали'те _е=го` вси` лю'дiе: JА='кw
о_у=тверди'ся мл\сть _е=гw` на на'съ, и= и='стина гд\сня пребыва'етъ во
вjь'къ.

\delimpict

\subsection{А=ллилу'iа, рз~_i}

\bukv{И=}сповjь'дайтеся гд\сви, jа='кw бл~гъ, jа='кw въ вjь'къ мл\сть
_е=гw`. Да рече'тъ о_у='бw до'мъ i=и~левъ: jа='кw бл~гъ, jа='кw въ вjь'къ
мл\сть _е=гw`. Да рече'тъ о_у='бw до'мъ а=арw'нь: jа='кw бл~гъ, jа='кw въ
вjь'къ мл\сть _е=гw`. Да реку'тъ о_у='бw вси` боя'щiися гд\са: jа='кw бл~гъ,
jа='кw въ вjь'къ мл\сть _е=гw`. W\т ско'рби призва'хъ гд\са, и= о_у=слы'ша мя`
въ простра'нство. Гд\сь мнjь` помо'щникъ, и= не о_у=бою'ся, что` сотвори'тъ
мнjь` человjь'къ; Гд\сь мнjь` помо'щникъ, и= а='зъ воззрю` на враги`
моя^. Бл~го _е='сть надjь'ятися на гд\са, не'жели надjь'ятися на человjь'ка:
Бл~го _е='сть о_у=пова'ти на гд\са, не'жели о_у=пова'ти на кня'зи. Вси`
jа=зы'цы w=быдо'ша мя`, и= и='менемъ гд\снимъ противля'хся и=`мъ: W=быше'дше
w=быдо'ша мя`, и= и='менемъ гд\снимъ противля'хся и=`мъ: W=быдо'ша мя` jа='кw
пче'лы со'тъ, и= разгорjь'шася jа='кw _о='гнь въ те'рнiи: и= и='менемъ
гд\снимъ противля'хся и=`мъ. W\тринове'нъ преврати'хся па'сти, и= гд\сь
прiя'тъ мя`. Крjь'пость моя` и= пjь'нiе мое` гд\сь, и= бы'сть ми` во
сп~се'нiе. Гла'съ ра'дости и= сп~се'нiя въ селе'нiихъ прв\дныхъ: десни'ца
гд\сня сотвори` си'лу. Десни'ца гд\сня вознесе' мя, десни'ца гд\сня сотвори`
си'лу. Не о_у=мру`, но жи'въ бу'ду, и= повjь'мъ дjьла` гд\сня: Наказу'я
наказа' мя гд\сь, сме'рти же не предаде' мя. W\тве'рзите мнjь` врата`
пра'вды: вше'дъ в ня` и=сповjь'мся гд\сви. Сiя^ врата^ гд\сня, прв\днiи
вни'дутъ в ня`. И=сповjь'мся тебjь`, jа='кw о_у=слы'шалъ мя` _е=си`, и= бы'лъ
_е=си` мнjь` во сп~се'нiе. Ка'мень, _е=гw'же небрего'ша зи'ждущiи, се'й
бы'сть во главу` о_у='гла: W\т гд\са бы'сть се'й и= _е='сть ди'венъ во
_о=чесjь'хъ на'шихъ. Се'й де'нь, _е=го'же сотвори` гд\сь, возра'дуемся и=
возвесели'мся во'нь. _W г\сди сп~си' же: _w гд\си, поспjьши' же. Бл~гослове'нъ
гряды'й во и='мя гд\сне: бл~гослови'хомъ вы` и=з\ъ до'му гд\сня. Бг~ъ гд\сь,
и= jа=ви'ся на'мъ: соста'вите пра'здникъ во о_у=чаща'ющихъ до рw'гъ
_о=лтаре'выхъ. Бг~ъ мо'й _е=си` ты`, и= и=сповjь'мся тебjь`: бг~ъ мо'й _е=си`
ты`, и= вознесу' тя. и=сповjь'мся тебjь`, jа='кw о_у=слы'шалъ мя` _е=си`, и=
бы'лъ _е=си` мнjь` во сп~се'нiе. И=сповjь'дайтеся гд\сви, jа='кw бл~гъ,
jа='кw въ вjь'къ мл\сть _е=гw`.

\slava

\delimpict

\subsection{По s~_i-й каf_i'смjь, трист~о'е}

\subsubsection{Та'же тропари`, гла'съ а~:}

\bukv{И=}нъ мi'ръ тебе`, душе`, w=жида'етъ, и= судiя` хо'щетъ твоя^ w=бличи'ти
та^йная и= лю^тая: не пребу'ди о_у='бw въ здjь'шнихъ, но предвари`, вопiю'щи
судiи`: бж~е, w=чи'сти мя`, и= сп~си' мя.

\rem{С__л__а'__в__а:} \bukv{JА='}кw прегрjьше'ньми мно'гими, и= jа='звами
безмjь'рными w=блежи'мь _е='смь, сп~се, согрjьша'яй молю` твое` бл~гоутро'бiе,
хр\сте`: врачу` неду'гющихъ посjьти` и= и=сцjьли` и= сп~си' мя.

\rem{И=__ н__ы'__н__jь:__} \bukv{Д}уше` моя`, что` неради'во живе'ши
лjьня'щися; что` не пече'шися w= sлы'хъ, и='хже содjь'яла _е=си` въ житiи`;
потщи'ся и=спра'витися пре'жде да'же две'рь не заключи'тъ тебjь` гд\сь, бц\дjь
притецы`, припади` и= возопi'й: о_у=пова'нiе ненадjь'емыхъ, сп~си' мя мно'гw
къ тебjь` прегрjьши'вша, вл\дчце преч\стая.

\vskip 0.5\baselineskip
\bukv{Г}д\си, поми'луй, \bukv{м~}. \rem{и= моли'тва:}
\vskip 0.5\baselineskip

\bukv{Г}д\си ст~ы'й, и='же въ вы'шнихъ живы'й, и= всеви'дящимъ
твои'мъ _о='комъ призира'яй на всю` тва'рь, тебjь` приклони'хомъ вы'ю
души` и= тjьлесе`, и= тебjь` мо'лимся ст~ы'й ст~ы'хъ: простри` ру'ку
твою` неви'димую w\т ст~а'гw жили'ща твоегw`, и= бл~гослови` вся^
ны`: и= прости` на'мъ вся'кое согрjьше'нiе, во'льное же и= нево'льное,
сло'вомъ и=ли` дjь'ломъ: да'руй на'мъ гд\си о_у=миле'нiе, да'руй сле'зы
духw'вныя w\т души`, во w=чище'нiе мно'гихъ на'шихъ грjьхw'въ: да'руй ве'лiю
твою` мл\сть на мi'ръ тво'й, и= на ны` недостw'йныя рабы^ твоя^: jа='кw
бл~гослове'но и= препросла'влено _е='сть и='мя твое`, _о=ц~а`, и= сн~а, и=
ст~а'гw дх~а, ны'нjь и= при'снw и= во вjь'ки вjькw'въ, а=ми'нь.

\csendpict

\clearpage
\hdrcrosspage
\subsection{Мл~тва w= дх~о'внjьмъ _о=тцjь`}

\bukv{С}п~си`, гд\си, и= поми'луй _о=тца` моего` дх~о'внаго
\rem{[и=м\ркъ]} и= прости` _e=му` вся^ согрjьш_e'нiя, не w=суди` и= не
и=стяжи` _e=го` ра'ди грjьхо'вныя моея` жи'зни, о_у=мно'жи въ не'мъ
дарова^нiя, и=спо'лни _e=го` му'дростiю, мл~твою и= любо'вiю, и=
ра'ди ст~ы'хъ _e=гw` мл~твъ да'руй мои'мъ грjьха'мъ проще'нiе, жи'зни
и=справле'нiе, въ добродjь'телехъ преспjь'янiе.  низпосли` _e=му`
мл\сти твоя^ бога^тыя: сподо'би _e=го` въ де'нь се'й \rem{ [въ но'щь
  сiю`]} без\ъ грjьха` сохрани'тися. побори` враги` _e=гw` плwтск_i'я
и= безплw'тныя и= w\т ви'димыхъ и= неви'димыхъ врагw'въ и=зба'ви
_e=го`. И=зми` _e=го` w\т человjь'ка льсти'ва и= му'жа непра'ведна,
соблюди` па'ству _e=гw` и= всjь'хъ _e=гw` дх~о'вныхъ ча^дъ да'же до
кончи'ны и='хъ и=здыха'нiя. да'руй w=сла'бу неду'гу _e=го` гнjьту'щему
и= w\т _о=дра` не'мощи воздви'гни цjь'ла и= всесоверше'нна. помяни`,
посjьти`, о_у=крjьпи`, о_у=тjь'ши и= сохрани` _e=го`, гд\си, на
мнw'гая лjь^та! _w сладча'йшiй _i=и~се! ты _e=си` безмjь'рнw ст~ъ,
безмjь'рнw прв\днъ, безмjь'рнw мл\срдъ! W=ст~и`, гд\си, ст~ы'нею
твое'ю _о=тца` моего` дх~о'внаго \rem{[и=м\ркъ]}, w=правда'й _e=го`
твое'ю прв\дностiю, покры'й _e=го` твои'мъ мл\срдiемъ! _w гд\си! ты
соедини'лъ _e=си` на'съ на земли`, не разлучи` на'съ и= въ нб\снjьмъ
твое'мъ цр\ствiи, и= _e=гw` ст~ы'ми мл~твами прости' мя грjь'шнаго
и= всjь'хъ _e=гw` дх~о'вныхъ ча^дъ.

%\vskip -0.8\baselineskip
\csendpict

\end{document}
