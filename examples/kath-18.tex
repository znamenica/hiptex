\documentclass[12pt,twoside,a6paper,xdvi,civil=antiqua]{hipbook}

\tolerance 1000

% ==========================================================================================

\begin{document}

\stdcrosstitle{
_ПСАЛТИРЬ \\
\vskip 5 mm
Каfi'сма и~_i}

\maketitle

\stdsecondpage
\clearpage

% ==========================================================================================

\hdrcrosspage

\section{{\Large Разу'мно да бу'детъ,}}
\section[Разу'мно да бу'детъ,]{ка'кw подоба'етъ _о=со'бь пjь'ти _псалти'рь.}
\subsection[ка'кw подоба'етъ _о=со'бь пjь'ти _псалти'рь.]{}
\vskip -0.5\baselineskip

\subsubsection{А='ще _i=ере'й, глаго'летъ:}

\bukv{Б}л~гослове'нъ бг~ъ на'шъ, всегда`, ны'нjь и= при'снw, и= во
вjь'ки вjькw'въ.

\subsubsection{А='ще ли ни`, глаго'ли о_у=миле'ннw:}

\bukv{М}л~твами ст~ы'хъ _о=т_e'цъ на'шихъ, гд\си, _i=и~се хр\сте`,
бж~е на'шъ, поми'луй на'съ, а=ми'нь.

\bukv{Ц}р~ю` нб\сный: \rem{Трист~о'е.} \bukv{О='}ч~е на'шъ:

\subsubsection{И= тропари` сiя^, гла'съ s~:}

\bukv{П}оми'луй на'съ, гд\си, поми'луй на'съ: вся'\-кагw бо w\твjь'та
недоумjь'юще, сiю' ти мл~тву, ja='кw вл\дцjь, грjь'шнiи прино'симъ:
поми'луй на'съ.

\rem{С__л__а'__в__а:} Ч\Стно'е пр\оро'ка твоегw`, гд\си, торжество`,
нб~о цр~ковь показа`, съ человjь'ки ли\-ку'\-ютъ а='гг~ли: тогw`
мл~твами хр\сте` бж~е, въ ми'рjь о_у=пра'ви живо'тъ на'шъ, да пое'мъ
ти`: а=ллилу'iа.

\rem{И=__ н__ы'__н__jь:} Мнw'гая мно'ж_eства мои'хъ бц\де,
прегрjьше'нiй, къ тебjь` прибjьго'хъ ч\стая, сп~се'\-нiя тре'\-буя:
посjьти` не\-мощ\-ст\-ву'\-ю\-щую мою` ду'шу, и= моли` сн~а твоего`,
и= бг~а на'\-ше\-го, да'ти ми` w=став\-ле'\-нiе, ja=`же содjь'яхъ
лю'\-тыхъ, _e=ди'на бл~гослове'нная.

\vskip 0.5\baselineskip
\centerline{\bukv{Г}д\си поми'луй, \rem{м~}.}
\vskip 0.5\baselineskip

\centerline{\rem{И= поклони'ся, _e=ли'кw ти` мо'щно.}}

\subsubsection{Та'же мл~тва ст~jь'й живонача'льнjьй тр\оцjь:}

\bukv{В}сест~а'я тр\оце, бж~е и= содjь'телю всегw` м_i'\-ра, поспjьши`
и= напра'ви се'рдце мое` нача'ти съ ра'\-зу\-момъ, и= конча'ти дjь'лы
бл~ги'ми бг~о\-духнов_e'нныя сiя^ кни^ги, ja=`же ст~ы'й дх~ъ
о_у=с\-ты^ дв~дwвы w\тры'гну, и=`хже ны'нjь хощу` гла\-го'\-ла\-ти
а='зъ недосто'йный. разумjь'я же свое` невjь'жество, припа'дая молю'ся
ти`, и= _e='же w\т тебе` по'мощи прося`: гд\си, о_у=пра'ви о_у='мъ
мо'й, и= о_у=тверди` се'рдце мое`, не w= гла\-го'\-ла\-нiи о_у=сте'нъ
стужа'ти си`, но w= ра'\-зумjь гла\-го'\-ле\-мыхъ весели'тися, и=
при\-го\-то'\-ви\-ти\-ся на творе'нiе до'брыхъ дjь'лъ, ja=`же
о_у=чу'ся, и= глаго'лю: да до'брыми дjь'лы просвjьще'нъ, на суди'щи
десны'я ти` страны` при\-ча'ст\-никъ бу'\-ду со всjь'ми и=збра'нными
твои'ми. и= ны'нjь вл\дко, бл~гослови`, да воздохну'въ w\т се'рдца, и=
я=зы'комъ воспою`, глаго'ля си'це:

\begin{center}
  %\vskip -0.5\baselineskip
  \bukv{П}рiиди'те, поклони'мся: \rem{три'жды.}
  %\vskip -0.5\baselineskip
\end{center}

{\small \color[named]{Red} Та'же посто'й ма'лw, до'ндеже о_у=тиша'тся вся^
  чу^вст\-ва: тогда` сотвори` нача'ло не вско'рjь, без\ъ лjь'ности, со
  о_у=миле'нiемъ и= сокруше'ннымъ се'рдцемъ. Рцы` сiе`: {\color[named]{Black}
    \normalfont\bukv{К}о гд\су, внегда` скорбjь'ти ми`, воззва'хъ:}
  ти'хw и= разу'мнw, со внима'нiемъ, а= не борзя'ся, ja='коже и= о_у=мо'мъ
  разумjьва'ти гла\-го'\-л_e\-мая.}

%\vskip 2.0\baselineskip
\csendpict

\clearpage
\hdrcrosspage

\section{{\Large Дв~да пр\оро'ка и= цр~я` пjь'снь}}
\vskip -0.3\baselineskip
\section[Каfi'сма и~_i]{Каfi'сма и~_i:}
%\vskip -0.3\baselineskip

\subsection{Пjь'снь степе'ней, рf~_i}

\bukv{К}о гд\су, внегда` скорбjь'ти ми`, воззва'хъ, и= о_у=слы'ша мя`. Гд\си,
и=зба'ви ду'шу мою` w\т о_у=сте'нъ непра'ведныхъ, и= w\т я=зы'ка
льсти'ва. Что` да'стся тебjь`, и=ли` что` приложи'тся тебjь` къ я=зы'ку
льсти'ву; Стрjь'лы си'льнагw и=зw\-щ\-р_е'\-ны со о_у='гльми пусты'нными. О_у=вы`
мнjь`, jа='кw прише'льствiе мое` продолжи'ся, все\-ли'х\-ся съ сел_е'нiи
кида'рскими: Мно'гw при\-ше'ль\-ст\-во\-ва душа` моя`. съ ненави'дящими ми'ра бjь'хъ
ми'ренъ. Е=гда` глаго'лахъ и=`мъ, боря'ху мя` ту'не.

\delimpict

\subsection{Пjь'снь степе'ней, рк~}

\bukv{В}озведо'хъ _о='чи мои` въ го'ры, w\тню'дуже прiи'детъ по'мощь
моя`. По'мощь моя` w\т гд\са, сотво'ршагw нб~о и= зе'млю. Не да'ждь во
смя\-те'\-нiе ноги` твоея`, ниже` воздре'млетъ храня'й тя`: Се` не воздре'млетъ,
ниже` о_у='снетъ хра\-ня'й i=и~ля. Гд\сь сохрани'тъ тя`, гд\сь покро'въ тво'й на
руку` десну'ю твою`. Во дни` со'лнце не w=жже'тъ тебе`, ниже` луна`
но'щiю. Гд\сь со\-хра\-ни'тъ тя` w\т вся'кагw sла`, сохрани'тъ ду'шу твою` гд\сь:
Гд\сь сохрани'тъ вхожде'нiе твое`, и= и=схожде'нiе твое`, w\т ны'нjь и= до
вjь'ка.

\delimpict

\subsection{Пjь'снь степе'ней, рк~а}

\bukv{В}озвесели'хся w= рjь'кшихъ мнjь`, въ до'мъ гд\снь по'йдемъ: Стоя'ще
бя'ху но'ги на'шя во дво'рjьхъ твои'хъ i=ер\сли'ме. I=ер\сли'мъ зи'ждемый
jа='кw гра'дъ, _е=му'же прича'стiе _е=гw` вку'пjь. Та'\-мw бо взыдо'ша кwлjь'на,
кwлjь'на гд\сня, свидjь'нiе i=и~лево, и=сповjь'датися и='мени гд\сню. JА='кw
та'мw сjьдо'ша престо'ли на су'дъ, престо'ли въ дому` дв~довjь. Вопроси'те же
jа=`же w= ми'рjь i=ер\сли'ма: и= w=би'лiе лю'бящымъ тя`. Бу'ди же ми'ръ въ
си'лjь твое'й, и= w=би'лiе въ столпостjьна'хъ твои'хъ. Ра'ди бра'тiй мои'хъ и=
бли'жнихъ мои'хъ глаго'лахъ о_у='бw ми'ръ w= тебjь`. До'му ра'ди гд\са бг~а
на'шегw взыска'хъ бл~га^я тебjь`.

\delimpict

\subsection{Пjь'снь степе'ней, рк~в}

\bukv{К}ъ тебjь` возведо'хъ _о='чи мои`, живу'щему на нб~си. Се` jа='кw _о='чи
ра^бъ въ руку` госпо'дiй свои'хъ, jа='кw _о='чи рабы'ни въ руку` госпожи`
своея`: та'кw _о='чи на'ши ко гд\су бг~у на'шему, до'ндеже о_у=ще'дритъ
ны`. Поми'луй на'съ гд\си, поми'луй на'съ, jа='кw по мно'гу и=спо'лнихомся
о_у=ничиже'нiя: Наипа'че напо'лнися душа` на'шя поноше'нiя гобзу'ющихъ, и=
о_у=ничиже'нiя го'р\-дыхъ.

\delimpict

\subsection{Пjь'снь степе'ней, рк~г}

\bukv{JА='}кw а='ще не гд\сь бы бы'лъ въ на'съ, да рече'тъ о_у='бw i=и~ль:
JА='кw а='ще не гд\сь бы бы'лъ въ на'съ, внегда` воста'ти человjь'кwмъ на ны`,
о_у='бw живы'хъ поже'рли бы'ша на'съ: Внегда` прогнjь'ватися jа='рости и='хъ
на ны`, о_у=`бо вода` потопи'ла бы на'съ. Пото'къ пре'йде душа` на'ша: О_у='бw
пре'йде душа` на'ша во'ду непостоя'нную. Бл~гослове'нъ гд\сь, и='же не даде`
на'съ въ лови'тву зубw'мъ и='хъ. Душа` на'шя jа='кw пти'ца и=зба'вися w\т
сjь'ти ловя'щихъ: сjь'ть сокруши'ся и= мы` и=зба'влени бы'хомъ. По'мощь на'ша
во и='мя гд\са, сотво'ршагw не'бо и= зе'млю.

\slava

\delimpict

\subsection{Пjь'снь степе'ней, рк~д}

\bukv{Н}адjь'ющiися на гд\са jа='кw гора` сiw'нъ: не подви'жится въ вjь'къ
живы'й во i=ер\сли'мjь. Го'\-ры _о='крестъ _е=гw`, и= гд\сь _о='крестъ люде'й
сво\-и'хъ, w\т ны'нjь и= до вjь'ка. JА='кw не w=ста'витъ гд\сь жезла` грjь'шныхъ
на жре'бiй пра'ведныхъ, jа='кw да не про'струтъ пра'веднiи въ беззакw'нiя
ру'къ свои'хъ. О_у=бл~жи` гд\си бл~гi^я, и= пра^выя
се'рдцемъ. О_у=клоня'ющыяся же въ раз\-вра\-щ_е'\-нiя w\тведе'тъ гд\сь с дjь'\-ла\-ю\-щи\-ми
без\-за\-ко'\-нiе: ми'ръ на i=и~ля.

\delimpict

\subsection{Пjь'снь степе'ней, рк~_е}

\bukv{В}негда` возврати'ти гд\су плjь'нъ сiw'нь, бы'хомъ jа='кw
о_у=тjь'шени. Тогда` и=спо'лнишася ра'дости о_у=ста` на^ша, и= я=зы'къ на'шъ
весе'лiя, тогда` реку'тъ во jа=зы'цjьхъ: возвели'чилъ _е='сть гд\сь сотвори'ти
съ ни'ми. Возвели'чилъ _е='сть гд\сь сотвори'ти съ на'ми: бы'хомъ
веселя'щеся. Возврати` гд\си пленjь'нiе на'ше, jа='кw пото'ки
ю='гомъ. Сjь'ющiи слеза'ми ра'достiю по'жнутъ. Ходя'щiи хожда'ху и=
пла'кахуся, мета'юще сjь'мена своя^: гряду'ще же прi\-и'\-дутъ ра'достiю,
взе'млюще рукwя'ти своя^.

\delimpict

\subsection{Пjь'снь степе'ней, рк~s}

\bukv{А='}ще не гд\сь сози'ждетъ до'мъ, всу'е тру\-ди'\-ша\-ся зи'ждущiи:
а='ще не гд\сь со\-хра\-ни'тъ гра'дъ, всу'е бдjь` стрегi'й. Всу'е ва'мъ
_e='сть о_у='треневати, воста'нете по сjь\-дjь'\-нiи, ja=ду'щiи хлjь'бъ
болjь'зни, _e=гда` да'стъ воз\-лю'б\-л_eн\-нымъ свои^мъ со'нъ. Се` достоя'нiе
гд\сне сы'нове, мзда` плода` чре'внягw. JА='кw стрjь'лы въ руцjь` си'льнагw,
та'кw сы'нове w\ттрясе'нныхъ.  Бл~же'нъ, и='же и=спо'лнитъ же\-ла'\-нiе свое`
w\т ни'хъ: не постыдя'тся, _e=гда` глаго'лютъ врагw'мъ свои^мъ во вратjь'хъ.

\delimpict

\subsection{Пjь'снь степе'ней, рк~з}

\bukv{Б}л~же'ни вси` боя'щiися гд\са, ходя'щiи въ путе'хъ _е=гw`: Труды`
\kavykabegin{}плодw'въ\kavykaend{ру'къ} твои'хъ снjь'си: бл~же'нъ _е=си`, и=
добро` тебjь` бу'детъ. Жена` твоя` jа='кw лоза` плодови'та въ стра\-на'хъ до'му
твоегw`: Сы'нове твои` jа='кw новосажд_е'нiя ма^сличная _о='крестъ трапе'зы
тво\-ея`. Се` та'кw бл~гослови'тся человjь'къ боя'йся гд\са. Бл~гослови'тъ тя`
гд\сь w\т сiw'на, и= о_у='зриши бл~г^я i=ер\сли'ма вся^ дни^ живота` твоегw`,
И= о_у='зриши сы'ны сынw'въ твои'хъ: ми'ръ на i=и~ля.

\delimpict

\subsection{Пjь'снь степе'ней, рк~и}

\bukv{М}но'жицею бра'шася со мно'ю w\т ю='ности моея`, да рече'тъ о_у='бw
i=и~ль: Мно'жицею бра'шася со мно'ю w\т ю='ности моея`, и='бо не премого'ша
мя`. На хребтjь` мое'мъ дjь'лаша грjь^шницы, продолжи'ша беззако'нiе
свое`. Гд\сь пра'веденъ ссjьче` вы^я грjь'шникwвъ. Да постыдя'тся и=
возвратя'тся вспя'ть вси` ненави'дящiи сiw'на: Да бу'дутъ jа='кw трава`
\kavykabegin{}на здjь'хъ\kavykaend{на кро'вjьхъ}, jа='же пре'жде восторже'нiя
и='зше: Е='юже не и=спо'лни руки` своея` жня'й, и= нjь'дра своегw` рукwя'ти
собира'яй: И= не рjь'ша мимоходя'щiи: бл~гослове'нiе гд\сне на вы`,
бл~гослови'хомъ вы` во и='мя гд\сне.

\slava

\delimpict

\subsection{Пjь'снь степе'ней, рк~f}

\bukv{И=}з\ъ глубины` воззва'хъ къ тебjь` гд\си: гд\си о_у=слы'ши гла'съ
мо'й. Да бу'дутъ о_у='ши твои` вне'млющjь гла'су моле'нiя моегw`. А='ще
беззакw'нiя на'зриши, гд\си, гд\си, кто` постои'тъ; jа='кw о_у= тебе`
w=чище'нiе _е='сть. И='мене ра'ди твоегw` потерпjь'хъ тя` гд\си, потерпjь`
душа` моя` въ сло'во твое`: о_у=пова` душа` моя` на гд\са, W\т стра'жи
о_у='треннiя до но'ши. w\т стра'жи о_у='треннiя да о_у=пова'етъ i=и~ль на
гд\са: JА='кw о_у= гд\са ми'лость, и= мно'гое о_у= негw` и=збавле'нiе: и= то'й
и=зба'витъ i=и~ля w\т всjь'хъ беззако'нiй _е=гw`.

\delimpict

\subsection{Пjь'снь степе'ней, рл~}

\bukv{Г}д\си, не вознесе'ся се'рдце мое`, ниже` вознесо'стjься _о='чи мои`:
ниже` ходи'хъ въ вели'кихъ, ниже` въ ди'вныхъ па'че мене`. А='ще не
смиренному'дрствовахъ, но вознесо'хъ ду'шу мою`, jа='кw w\тдое'ное на ма'терь
свою`, та'кw возда'си на ду'шу мою`. Да о_у=пова'етъ i=и~ль на гд\са, w\т
ны'нjь и= до вjь'ка.

\delimpict

\subsection{Пjь'снь степе'ней, рл~а}

\bukv{П}омяни`, гд\си, дв~да, и= всю` кро'тость _e=гw`: JА='кw кля'тся гд\сви,
w=бjьща'ся бг~у i=а'\-кwв\-лю: А='ще вни'ду въ селе'нiе до'му моегw`, и=ли`
взы'ду на _о='дръ посте'ли моея`: А='ще да'мъ со'нъ _о=чи'ма мои'ма, и=
вjь'ждома мои'ма дрема'нiе, и= поко'й скранiа'ма мои'ма: До'ндеже w=бря'щу
мjь'сто гд\сви, селе'нiе бг~у i=а'кwвлю. Се` слы'шахомъ я=` во _e=vфра'fjь,
w=брjьто'хомъ я=` въ поля'хъ дубра'вы: Вни'демъ въ сел_e'нiя _e=гw`,
поклони'мся на мjь'сто, и=дjь'же стоя'стjь но'зjь _e=гw`: Воскр\сни`, гд\си,
въ поко'й тво'й, ты` и= кiвw'тъ святы'ни твоея`. Сщ~е'нницы твои` w=блеку'тся
пра'вдою, и= прп\дбнiи твои` возра'дуются. Дв~да ра'ди раба` твоегw`, не
w\тврати` лице` пома'заннагw твоегw`. Кля'тся гд\сь дв~ду и='стиною, и= не
w\тве'ржется _e=я`: w\т плода` чре'ва твоегw` посажду` на пре\-сто'\-лjь
твое'мъ: А='ще сохраня'тъ сы'нове твои` за\-вjь'тъ мо'й, и= свидjь^нiя моя^
сiя^, и=`мже научу` я=`, и= сы'нове и='хъ до вjь'ка ся'дутъ на престо'лjь
твое'мъ.  JА='кw и=збра` гд\сь сiw'на, и=зво'ли и=` въ жили'ще себjь`. Се'й
поко'й мо'й во вjь'къ вjь'ка, здjь` вселю'ся, ja='кw и=зво'лихъ и=`. Лови'тву
_e=гw` бл~гословля'яй бл~гословлю`, ни'щыя _e=гw` насыщу` хлjь^бы: Сщ~е'нники
_e=гw` w=блеку` во сп~се'нiе, и= прп\дбнiи _e=гw` ра'достiю возра'дуются.
Та'мw возращу` ро'гъ дв~дови, о_у=гото'вахъ свjьти'льникъ пома'занному моему`.
Враги` _e=гw` w=блеку` студо'мъ: на не'мже процвjьте'тъ ст~ы'ня моя`.

\delimpict

\subsection{Пjь'снь степе'ней, рл~в}

\bukv{С}е` что` добро`, и=ли` что` красно`, но _e='же жи'ти бра'тiи вку'пjь;
JА='кw мv'ро на гла\-вjь`, схо\-дя'\-щее на браду`, браду` а=арw'ню,
схо\-дя'\-щее на w=ме'ты _о=де'жды _e=гw`: JА='кw роса` а=ермw'нская сходя'щая
на го'ры сiw^нскiя: ja='кw та'мw заповjь'да гд\сь бл~гослове'нiе и= живо'тъ до
вjь'ка.

\delimpict

\subsection{Пjь'снь степе'ней, рл~г}

\bukv{С}е` ны'нjь бл~гослови'те гд\са вси` раби` гд\сни, стоя'щiи въ хра'мjь
гд\сни, во дво'рjьхъ до'му бг~а на'шегw. Въ но'щехъ воздjьжи'те ру'ки ва'шя во
ст~а^я, и= бл~гослови'те гд\са. Бл~гослови'тъ тя` гд\сь w\т сiw'на,
сотвори'вый не'бо и= зе'млю.

\slava

\delimpict

\subsection{По и~_i-й каf_i'смjь, трист~о'е}

\subsubsection{Та'же тропари`, гла'съ в~:}

\bukv{П}ре'жде да'же не w=су'диши мя`, гд\си мо'й гд\си, да'ждь ми`
w=браще'нiе и= и=справле'нiе мно'гихъ мои'хъ грjьхw'въ: да'ждь ми`
о_у=\-ми\-ле'\-нiе дх~о'вное, jа='кw да возопiю` къ тебjь`: бл~гоутро'бне
чл~вjьколю'бче бж~е мо'й, сп~си' мя.


\rem{С__л__а'__в__а:} \bukv{Н}есмы'сл_еннымъ скотw'мъ
о_у=\-по\-до'\-би\-вый\-ся а='зъ блу'дный, приложи'хся и=`мъ: w=браще'нiе ми`
да'руй, хр\сте`, jа='кw да прiиму` о_у= тебе` ве'лiю мл\сть.

\rem{И=__ н__ы'__н__jь:__} \bukv{Н}е w\тврати`, вл\дчце, лица` твоегw` w\т
мене` моля'щагwся тебjь`, но jа='кw бл~гоутро'бная мт~и ще'драгw бг~а,
потщи'ся пре'жде конца` w=браще'нiе мнjь` дарова'ти, jа='кw да сп~сся тобо'ю,
воспою' тя, jа='кw сп~се'нiе, и= о_у=пова'нiе мое` непосра'мленное, гп\сже моя`.

\bukv{Г}д\си, поми'луй, \bukv{м~}. \rem{и= моли'тва:}

\bukv{Г}д\си, да не jа='ростiю твое'ю w=бличи'ши мене`, ниже` гнjь'вомъ
твои'мъ нака'жеши мя`. Вл\дко гд\си i=и~се хр\сте`, сн~е бг~а жива'гw,
поми'луй мя грjь'шнаго, ни'щаго, w=бнаже'ннаго, лjь\-ни'\-ва\-го, неради'ваго,
прекосло'внаго, _о=кая'ннаго, блудника`, прелюбодjь'я, малакi'я,
му\-же\-ло'ж\-ни\-ка, скве'рнаго, блу'днаго, неблагода'рнаго,
не\-ми'\-ло\-сти\-ва\-го, жесто'каго, пiя'ницу, со\-жже'н\-на\-го со'вjьстiю,
безли'чнаго, без\-дерз\-но\-ве'н\-на\-го, без\-w\т\-вjь'т\-на\-го, недосто'йнаго твоегw`
чл~вjь\-ко\-лю'\-бiя, и= досто'йна вся'кагw муче'нiя, и= гее'нны, и= му'ки. И= не
ра'ди мно'жества толи'кихъ мои'хъ согрjьше'нiй, мно'жеству подложи'ши
и=зба'вителю му'къ: но поми'луй мя, jа='кw не'мощенъ _е='смь, и= душе'ю, и=
пло'тiю, и= ра'зумомъ, и= помышле'нiемъ: и= и='миже вjь'си судьба'ми, сп~си'
мя недосто'йнаго раба` твоего`, мл~твами преч\стыя вл\дчцы на'шея бц\ды, и=
всjь'хъ ст~ы'хъ, w\т вjь'ка тебjь` бл~гоуго'ждшихъ: jа='кw бл~гослове'нъ
_е=си` во вjь'ки вjькw'въ, а=ми'нь.

\vskip 1\baselineskip
\csendpict

\clearpage
\hdrcrosspage
\subsection{Мл~тва w= дх~о'внjьмъ _о=тцjь`}

\bukv{С}п~си`, гд\си, и= поми'луй _о=тца` моего` дх~о'в\-на\-го
\rem{[и=м\ркъ]} и= прости` _e=му` вся^ согрjьш_e'нiя, не w=суди` и= не
и=стяжи` _e=го` ра'ди грjьхо'вныя моея` жи'зни, о_у=мно'жи въ не'мъ
дарова^нiя, и=спо'лни _e=го` му'\-дро\-с\-тiю, мл~твою и= любо'вiю, и=
ра'ди ст~ы'хъ _e=гw` мл~твъ да'руй мои'мъ грjьха'мъ проще'нiе, жи'зни
и=справле'нiе, въ добродjь'телехъ преспjь'янiе.  низпосли` _e=му`
мл\сти твоя^ бога^тыя: сподо'би _e=го` въ де'нь се'й \rem{ [въ но'щь
  сiю`]} без\ъ грjьха` сохрани'тися. побори` враги` _e=гw` плwтск_i'я
и= безплw'тныя и= w\т ви'димыхъ и= неви'димыхъ врагw'въ и=зба'ви
_e=го`. И=зми` _e=го` w\т человjь'ка льсти'ва и= му'жа непра'ведна,
соблюди` па'ству _e=гw` и= всjь'хъ _e=гw` дх~о'вныхъ ча^дъ да'же до
кончи'ны и='хъ и=здыха'нiя. да'руй w=сла'бу неду'гу _e=го` гнjьту'щему
и= w\т _о=дра` не'мощи воздви'гни цjь'ла и= всесоверше'нна. помяни`,
посjьти`, о_у=крjьпи`, о_у=тjь'ши и= сохрани` _e=го`, гд\си, на
мнw'гая лjь^та! _w сладча'йшiй _i=и~се! ты _e=си` безмjь'рнw ст~ъ,
безмjь'рнw прв\днъ, безмjь'рнw мл\срдъ! W=ст~и`, гд\си, ст~ы'нею
твое'ю _о=тца` моего` дх~о'внаго \rem{[и=м\ркъ]}, w=прав\-да'й _e=го`
твое'ю прв\дностiю, покры'й _e=го` тво\-и'мъ мл\срдiемъ! _w гд\си! ты
соедини'лъ _e=си` на'съ на земли`, не разлучи` на'съ и= въ нб\снjьмъ
твое'мъ цр\ствiи, и= _e=гw` ст~ы'ми мл~твами прости' мя грjь'ш\-на\-го
и= всjь'хъ _e=гw` дх~о'вныхъ ча^дъ.

%\vskip -0.8\baselineskip
\vskip 1\baselineskip
\csendpict

\end{document}
