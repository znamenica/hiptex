\documentclass[12pt,a5paper,dvips,civil=antiqua]{hipbook}

\tolerance 5000

% ==========================================================================================

\begin{document}

\stdcrosstitle{
{\Large\MakeUppercase{трiw'дiонъ}}\\
\vskip 0.5\baselineskip
\large сi'есть трипjь'снецъ}

\maketitle

\stdsecondpage
\clearpage

% ==========================================================================================
\hdrcrosspage

\section[трiw'дiонъ]{\large\MakeUppercase{трiw'дiонъ сi'есть трипjь'снецъ}}

\subsubsection{съ бг~омъ ст~ы'мъ, w=бдержа'й подоба'ющее е=му` послjь'дованiе.}

\subsection[Недjь'ля мытаря` и= фарiсе'а]
{Недjь'ля, въ ню'же чте'тся сщ~е'нное и= ст~о'е _е=v\глiе, притчи
  мытаря` и= фарiсе'а.}

\rem{Въ суббw'ту ве'чера, по предначина'тельномъ _псалмjь`,
  стiхосло'вимъ} \bukv{Б}л~же'нъ му'жъ: \rem{а~-ю каfi'сму всю'}.

\rem{На} \bukv{Г}д\си воззва'хъ, \rem{поста'вимъ стiхw'въ i~: и=
  пое'мъ _о=смогла'сника стiхи^ры воскр\сны г~: и= а=натw'лiевы д~: и=
  трiw'ди самогла'сны двjь`, повторя'юще пе'рвую. Гла'съ а~.}

\bukv{Н}е помо'лимся фарiсе'йски, бра'тiе: и='бо вознося'й себе`
смири'тся. смири'мъ себе` пред\ъ бг~омъ, мыта'рски поще'нiемъ зову'ще:
w=чи'сти ны бж~е, грjь'шныя.

\bukv{Ф}арiсе'й тщесла'вiемъ побjьжда'емь, и= мыта'рь покая'нiемъ
приклоня'емь, приступи'ста къ тебjь` _е=ди'ному вл\дцjь: но _о='въ
о_у='бw похвали'вся, лиши'ся бл~ги'хъ: _о='въ же ничто'же вjьща'въ,
сподо'бися дарова'нiй. въ си'хъ воздыха'нiихъ о_у=тверди' мя хр\сте`
бж~е, ja='кw чл~вjьколю'бецъ.

\rem{Сла'ва, гла'съ и~:} \bukv{В}седержи'телю гд\си, вjь'мъ, коли'кw мо'гутъ
сле'зы: _е=зекi'ю бо w\т вра'тъ сме'ртныхъ возведо'ша, грjь'шную w\т
многолjь'тныхъ согрjьше'нiй и=зба'виша, мытаря' же па'че фарiсе'а
w=правда'ша: и= молю'ся, съ ни'ми причта'въ, поми'луй мя`.

\subsubsection{И= ны'нjь, бг~оро'диченъ а~, рядова'гw гла'са.}

\subsubsection{На лiтi'и стiхи'ра ст~а'гw _о=би'тели, по _о=бы'чаю.}

\rem{Сла'ва, гла'съ г~:} \bukv{М}ытаря` и= фарiсе'а разли'чiе разумjь'вши
душе` моя`, _о='нагw о_у='бw возненави'ждь горды'нный гла'съ, _о='вагw
же ревну'й благоумиле'нной мл~твjь, и= возопi'й: бж~е w=чи'сти мя`
грjь'шнаго, и= поми'луй мя`.

\subsubsection{И ны'нjь, бг~оро'диченъ воскр\снъ, въ то'йже гла'съ.}

\subsubsection{На стiхо'внjь стiхи^ры _о=смогла'сника по а=лфави'ту.}

\rem{Сла'ва, гла'съ _е~:} \bukv{W=}тягче'нныма _о=чи'ма мои'ма w\т
беззако'нiй мои'хъ, не могу` воззрjь'ти и= ви'дjьти высоту` нб\сную:
но прiими' мя, ja='кw мытаря` ка'ющася, сп~се, и= поми'луй мя`.

\rem{И= ны'нjь, бг~оро'диченъ, въ то'йже гла'съ:} \bukv{Х}ра'мъ и= две'рь
_е=си`: Тропа'рь: Бц\де дв~о, глаго'лется три'жды. И быва'етъ
бл~гослове'нiе хлjь'бwвъ, и= раздая'нiе. И чте'нiе ве'лiе въ
посла'нiихъ а=п\сльскихъ.

\rem{Послjь'дованiе\marginpar{Зри`} прилучи'вшагwся ст~а'гw въ
  недjь'лю сiю`, и= въ блу'днагw, пое'мъ въ пято'къ на повече'рiи,
  и=ли` _е=гда` _е=кклисiа'рхъ разсу'дитъ: а='ще то'кмw не случи'тся
  ко'егw вели'кагw ст~а'гw, и=ли` ст~а'гw хра'му: зане` хра'мъ и= не
  вели'кагw ст~а'гw, бо'льши вослjь'дуется вели'кагw ст~а'гw, на сво'й
  _е=му` хра'мный пра'здникъ, _е='же до'лжно _е='сть бдjь'нiю быва'ти,
  и= ника'коже w=ставля'ти въ па'мять _е=гw`, ниже` прелага'ется на
  и='нъ де'нь. Но пое'мъ слу'жбу хра'му, ja='коже и= срjь'т=енiю
  гд\сню, во вся^ дни^, въ ня'же и= срjь'тенiе гд\сне быва'етъ.\par}

\rem{Сла'ва, гла'съ и~:} \bukv{В}седержи'телю гд\си, вjь'мъ, коли'кw
мо'гутъ сле'зы: _е=зекi'ю бо w\т вра'тъ сме'ртныхъ возведо'ша,
грjь'шную w\т многолjь'тныхъ согрjьше'нiй и=зба'внша, мытаря' же па'че
фарiсе'а w=правда'ша: и= молю'ся, съ ни'мн прнчта'въ, поми'луй мя`. И
ны'нjь, бг~оро'диченъ а, рядова'гw гла'са.

\subsubsection{На лiтi'и стiхи'ра ст~а'гw _о=би'тели, по _о=бы'чаю.}

\rem{Сла'ва, гла'съ г~:} \bukv{М}ытаря` и= фарiсе'а разли'чiе
разумjь'вши душе` моя`, _о='нагw о_у='бw возненави'ждь горды'нный
гла'съ, _о='вагw же ревну'й благоумнле'нной мл~твjь, и= возопi'й: бж~е
w=чи'стн мя` грjь'шнаго, и= поми'луй мя`. И ны'нjь, бг~оро'диченъ
воскр\снъ, въ то'йже гла'съ.

\subsubsection{На стiхо'внjь стiхи^ры _о=смогла'сника по а=лфави'ту.}

\rem{Сла'ва, гла'съ _е~:} \bukv{W}тягче'нныма _о=чи'ма мои'ма w\т беззако'нiй
мои'хъ, не могу` воззрjь'ти и= ви'дjьти высоту` нб\сную: но прiими'
мя, jа='кw мытаря` ка'ющася, сп~се, и= поми'луй мя`.

\rem{И ны'нjь, бг~оро'диченъ, въ то'йже гла'съ:} \bukv{Х}ра'мъ и= две'рь
_е=си`: Тропа'рь: Бц\де дв~о, глаго'лется три'жды. И быва'етъ
бл~гослове'нiе хлjь'бwвъ, и= раздая'нiе. И чте'нiе ве'лiе въ
посла'нiихъ а=п\сльскихъ.

 \rem{\marginpar{Зри`}Послjь'дованiе прилучи'вшагwся ст~а'гw въ недjь'лю сiю`, и= въ
блу'днагw, пое'мъ въ пято'къ на повече'рiи, и=ли` _е=гда`
_е=кклисiа'рхъ разсу'дитъ: а='ще то'кмw не случи'тся ко'егw вели'кагw
ст~а'гw, и=ли` ст~а'гw хра'му: зане` хра'мъ и= не вели'кагw ст~а'гw,=
бо'льши вослjь'дуется вели'кагw ст~а'гw, на сво'й _е=му` хра'мный
пра'здникъ, _е='же до'лжно _е='сть бдjь'нiю быва'ти, и= ника'коже
w=ставля'ти въ па'мять _е=гw`, ниже` прелага'ется на и='нъ де'нь. Но
пое'мъ слу'жбу хра'му, jа='коже и= срjь'т=енiю гд\сню, во вся^ дни^,
въ ня'же и= срjь'тенiе гд\сне быва'етъ.\par}

\subsection[На о_у='трени]{На о_у='трени,}

\rem{по шести` _псалмjь'хъ,} \bukv{Б}г~ъ гд\сь, \rem{на гла'съ _о=смогла'сника. И
глаго'лемъ тропа'рь воскр\снъ два'жды: бг~оро'диченъ _е=ди'ножды, и=
_о=бы^чная стiхослw'вiа. Сjьда'льны _о=смогла'сника. По непоро'чныхъ
тропари`:} \bukv{А='}гг~льскiй собо'ръ: v=пакои`. \rem{Степе'нны и= прокi'менъ
гла'са:} \bukv{В}ся'кое дыха'нiе: \rem{Е=v\глiе воскр\сно ря'ду.}  \bukv{В}оскр\снiе
хр\сто'во: \rem{_псало'мъ н~.\par}

\rem{Сла'ва, гла'съ и~:} \bukv{П}окая'нiя w\тве'рзи ми` дв_е'ри жизнода'вче,
о_у='тренюетъ бо ду'хъ мо'й ко хра'му ст~о'му твоему`, хра'мъ нося'й
тjьле'сный ве'сь w=скверне'нъ: но jа='кw ще'дръ, w=чи'сти
бл~гоутро'бною твое'ю мл\стiю.

\rem{И= ны'нjь, бг~оро'диченъ:} \bukv{Н}а сп~се'нiя ст_ези` наста'ви
мя` бц\де, сту'днымн бо w=каля'хъ ду'шу грjьхми`, и= въ лjь'ности все`
житiе` мое` и=жди'хъ: но твои'ми мл~твами и=зба'ви мя` w\т вся'кiя
нечистоты`.

\rem{Та'же, гла'съ s~:} \bukv{П}оми'луй мя` бж~е, по вели' цjьй мл\сти
твое'й, и= по мно'жеству щедро'тъ твои'хъ, w=чи'сти беззако'нiе мое`.

\bukv{М}но'ж_ества содjь'янныхъ мно'ю лю'тыхъ, помышля'я _о=кая'нный,
трепе'щу стра'шнагw дне` су'днагw: но надjь'яся на мл\сть
бл~гоутро'бiя твоегw`, jа='кw дв~дъ вопiю' ти: поми'луй мя` бж~е по
вели'цjьй твое'й мл\сти.

\subsubsection[Канw'нъ]{Канw'нъ воскр\снъ _о=смогла'сника, со i=рмосо'мъ на д~, и=
  кр\стовоскр\сный на два`, и= бц\ды на два`: и= въ трiw'ди на s~.
  Творе'нiе геw'ргiа. Е=гw'же краестро'чiе въ бг~оро'дичнахъ:
  геw'ргiа, гла'съ s~:}

\rem{Пjь'снь а~:} \bukv{Jа='}кw по су'ху пjьшеше'ствовавъ i=и~ль: 

\bukv{П}ри'тчами вводя'й вся^ хр\сто'съ, къ житiя` и=справле'нiю,
мытаря` возвыша'етъ w\т смире'нiя, показа'въ фарiсе'а возвыше'нiемъ
смиря'ема.

\bukv{W\т} смире'нiя че'сть высокотворя'щую, w\т возноше'нiя же
паде'нiе ви'дя лю'тое, мытар_е'вымъ ревну'й дw'брымъ, и= фарiсе'йскую
sло'бу возненави'ждь.

\bukv{W\т} возноше'нiя и=спражня'ется вся'кое бл~го'е, w\т смире'нiя
же потребля'ется вся'кое sло'е: _е='же w=блобыза'имъ вjь'рнiи,
гнуша'ющеся jа='вjь w='браза тщесла'внагw.

\bukv{С}миренному^дрымъ бы'ти свои^мъ о_у=ч~нкw'мъ хотя` всjь'хъ цр~ь,
наказу'я о_у=ча'ше ревнова'ти мытаре'ву воздыха'нiю и= смире'нiю.

\rem{Сла'ва:} \bukv{Jа='}кw мыта'рь стеню`, и= рыда'ньми немо'лчными
гд\си, ны'нjь прихожду` твоему` бл~гоутро'бiю: о_у=ще'дри и= мене`,
смире'нiемъ жи'зиь ны'нjь препровожда'юща.

\rem{Бг~оро'диченъ:\marginpar{Г}} \bukv{р}а'зумъ, совjь'тъ, ча'янiе,
тjь'ло, ду'шу и= ду'хъ, вл\дчце, возлага'ю къ тебjь`: лю'тыхъ вра^гъ
и= напа'стей, и= бу'дущагw преще'нiя и=зба'ви, и= сп~си' мя.

\subsubsection{Катава'сiа: W\тве'рзу о_у=ста` моя^: }

\rem{Пjь'снь г~:} \bukv{Н}jь'сть ст~ъ, jа='коже ты`:

\bukv{_W\т} гно'ища о_у='бw страсте'й смире'нный возноша'ется, w\т высоты' же
добродjь'телей низпа'даетъ лю'тjь вся'къ высокосе'рдый: _е=гw'же
w='браза sло'бы о_у=бjьжи'мъ.

\bukv{Т}щесла'вiе w\ттщетjьва'етъ бога'тство пра'вды, смире'нiе же
расточа'етъ страсте'й мно'жество: _е='же подража'ющыя ны`, ча'сти
покажи` мытаре'вы сп~се.

\bukv{Jа='}кw мыта'рь и= мы` бiю'щеся въ п_е'рси, о_у=миле'нiемъ
вопiе'мъ: w=чи'сти бж~е на'съ грjь'шныхъ, jа='кw да сегw` прiи'мемъ
w=ставле'нiе.

\bukv{К}ъ ре'вности прiи'демъ вjь'рнiн, и=справля'юще кро'ткое,
смире'нiю соживу'ще въ стена'нiи се'рдца, пла'чjь же и= мл~твjь,
jа='кw да и='мамы w\т бг~а проще'нiе.

\rem{Сла'ва:} \bukv{W\т}ри'немъ вjь'рнiи высокохва'льную горды'ню,
возноше'нiе же лю'тое, и= дме'нiе ме'рзкое, и= sлjь'йшее фарiсе'ево
бг~у нелjь'пое свирjь'пство.

\rem{Бг~оро'диченъ:} \bukv{В}ъ тебjь` _е=ди'номъ прибjь'жищи
надjь'явся, да не w\тпаду` до'брагw ча'янiя: но да о_у=лучу` =гвою`
по'мощь ч\стая, вся'кагw вре'да лю'тыхъ и=збавля'емый.

\subsubsection{Сjьда'льны, гла'съ д~. Подо'бенъ: Ско'рw предвари`:}

\bukv{С}мире'нiе вознесе` w=держи'магw sлы'ми, мытаря` возстена'вша, и=
_е='же w=чи'сти, къ зижди'телю воззва'ша: возноше'нiе же низложи' w\т
пра'вды w=кая'ннаго фарiсе'а велерjь'чующа. тjь'мже поревну'имъ
дw'брымъ, sлы'хъ w\тступа'юще.

\rem{Сла'ва:} \bukv{С}мире'нiе дре'вле вознесе` мытаря`, пла'чемъ
возопи'вша: w=чи'сти, и= w=правди'ся. того` подража'имъ вси`, во
глубину sw'лъ впа'дшiи, возопiи'мъ спсу и=з\ъ глубины` се'рдца:
согрjьши'хомъ, w=чи'сти _е=ди'не чл~вjьколю'бче.

\rem{И= ны'нjь, бг~оро'диченъ:} \bukv{С}ко'рw прiими' вл\дчце,
мол_е'нiя на^ша, и= сiя^ принеси' твоему сн~у и= бг~у, гп\сже`
всенепоро'чная: разрjьши` w=бстоя^нiя къ тебjь` притека'ющихъ,
сокруши` кw'зни, и= низложи` де'рзость воwружа'ющихся безбо'жнw на
рабы^ твоя^, преч\стая.

\rem{Пjьснь д~:} \bukv{Х}р\сто'съ моя` си'ла:

\bukv{И=}зря'дный показа` пу'ть возноше'нiя, смире'нiе, сло'во смири'вшееся
да'же и= до зра'ка ра'бiя: _е=го'же подража'я вся'кiй, возноша'ется
смиря'яся.

\bukv{В}ознесе'ся прв\дникъ, и= низпаде` фарiсе'й: во мно'жествjь же
sw'лъ w=тягча'емый смири'ся, но вознесе'ся мыта'рь, ненаде'жнw
w=правда'емый.

\bukv{Н}ищеты` хода'тай, w\т бога'тства добродjь'телей, безу'мiе
jа=ви'ся: и= бога'тство а='бiе смире'нiе w=правда'нiя w\т кра'йнiя
нищеты`, юже стя'жимъ.

\bukv{П}pедpе'клъ _е=cи` вл\дко, велему'дpcтвующымъ пpотивоcта'ти
вcя'чеcки, и= cмиp_е'ннымъ бл~года'ть твою` подaя` cп~се, cмиpи'вшымcя
ны'нjь на'мъ твою` бл~года'ть низпоcли`.

\rem{Cла'вa:} \bukv{К}ъ бж~е'cтвенному возноше'нiю пpи'cнw возводя` на'cъ
cп~съ и= вл\дкa, выcокотво'pное покaза` cмиpе'нiе: но'ги бо
о_у=ч~нкw'мъ cвои'мa pука'мa w=бмы`.

\rem{Бг~оpо'диченъ:\marginpar{W}} \bukv{JА='}кw cвjь'тъ непpиcту'пный дв~о
pо'ждшaя, души` моея` тьму` cвjьти'тельною зapе'ю paзжени`, и= къ
cтезя'мъ cп~се'нiя житiе` мое` pуково'дcтвуй.

\rem{Пjь'cнь _е~:} \bukv{Б}ж~iимъ cвjь'томъ твои'мъ бл~же:

\bukv{Ф}apicе'евы добpодjь'тели потщи'мcя подpaжа'ти, и= поpевнова'ти
мытapе'ву cмиpе'нiю, во _о=бою` ненaви'дяще безмjь'cтное мнjь'нiе, и=
па'губу пaде'нiй.

\bukv{П}pа'вды тече'нiе тще'тное w=бличи'cя, cопpя'гъ въ не'мъ
фapicе'й мнjь'нiе: а='бiе же мыта'pь выcокотво'pною добpодjь'телiю
cтяжа` cпу'тникa, cмиpе'нiе.

\bukv{К}олеcни'чникъ въ добpодjь'телеxъ, мня'шеcя тещи` фapicе'й: но
пjь'шiй, па'че лv"дi'йcкiя колеcни'цы текi'й, мыта'pь до'бpjь
пpедвapи`, пpипpя'гъ щедpо'тjь cмиpе'нiе.

\bukv{M}ытapе'ву пpи'тчу вcи` о_у=paзумjь'вше о_у=мо'мъ, пpiиди'те
поpевну'имъ cлеза'мъ, ду'xъ cокpуше'нный бг~у пpиноcя'ще, гpjьxw=въ
и='щуще w=cтaвле'нiя.

\rem{Cла'вa:} \bukv{B}озноcли'вый и= sло'бный, гоpдели'вый же и=
де'pзый дaле'че w\тpи'немъ paзу'мнiи фapicе'евъ нpа'въ, лю'тый
велеxва'льный, jа='кw дa не w=бнaжи'мcя бж~е'cтвенныя бл~года'ти.

\rem{Бг~оpо'диченъ:\marginpar{P}} \bukv{Ж}е'злъ cи'лы бл~га'я, вcjь^мъ низпоcли`
на'мъ къ тебjь` пpибjьга'ющымъ, w=блaда'ти поcpедjь` вcjь'xъ вpа^гъ
подaю'щи, и= вcя'кaгw вpе'дa и=з\ъима'ющи.

\rem{Пjь'cнь s~:} \bukv{Ж}ите'йcкое мо'pе:

\bukv{Ж}итiя` по'пpище мыта'pь вку'пjь и= фapicе'й теча'cтa: но _о='въ
о_у='бw выcокоу'мiемъ cодеpжи'мь, cpа'мнjь о_у=тону`, _о='въ же
cмиpе'нiемъ cп~се'cя.

\bukv{Ж}итiя` бjь'дное пpемину'юще мы` тече'нiе, подpaжа'имъ мытapе'во
о_у='бw мудpова'нiе pевни'тельнw: бjьжи'мъ же киче'нiя ме'pзкaгw
фapicе'a, и= жи'ви бу'демъ.

\bukv{H}pа'вwмъ поpевну'имъ cп~сa i=и~сa, и= _е=гw` cмиpе'нiю,
жела'юще непpеcта'нное pа'доcти cеле'нiе получи'ти, во cтpaнjь`
живу'щиxъ водвоpя'ющеcя.

\bukv{П}окaза'лъ _е=cи` вл\дко, твои^мъ о_у=ч~нкw=мъ выcокотво'pное
cмиpе'нiе, ле'нтiемъ пpепоя'caнъ по чpе'cлwмъ, но'зjь о_у=мы'лъ
_е=cи`, и= cему` w='бpaзу подpaжа'ти повелjь'лъ _е=cи`.

\rem{Cла'вa:} \bukv{Ж}итiе` пpоидо'cтa фapicе'й добpодjь'тельми, и= мыта'pь
пpегpjьше'ньми. но _о='въ о_у='бw w\т го'pдоcти о_у=мовpе'дныя,
w\тступи` cмиpе'нiя: _о='въ же возноша'етcя, cмиpенному'дpъ jа='вльcя.

\rem{Бг~оpо'диченъ:} \bukv{H}а'гa пpоcтото'ю, неxи'тpоcтною же жи'знiю
cозда'нa, пpеcтупле'нiя ле'cтiю w=дjь'я мя` вpа'гъ, и= пло'ти
дебельcтво'мъ: ны'нjь же твои'мъ xода'тaйcтвомъ _о=тpокови'це,
cп~са'юcя.

\rem{Конда'къ, гла'съ д~. Подо'бенъ:} \bukv{JА=}ви'лcя _е=cи` дне'cь:

\bukv{Ф}apicе'евa о_у=бjьжи'мъ выcокоглaго'лaнiя, и= мытapе'вjь
нaучи'мcя выcотjь` глaгw'лъ cмиpе'нныxъ, покaя'нiемъ взыва'юще: cп~се
мi'pa, w=чи'cти paбы^ твоя^.

\rem{Bтоpы'й конда'къ, гла'съ г~. Подо'бенъ:} \bukv{Д}в~a дне'cь:

\bukv{B}оздыxа^нiя пpинеcе'мъ мыта^pcкaя гд\сви, и= къ нему`
пpиcту'пимъ гpjь'шнiи jа='кw вл\дцjь: xо'щетъ бо cпacе'нiя вcjь'xъ
человjь'кwвъ, w=cтaвле'нiе подaе'тъ вcjь^мъ кa'ющымcя. на'cъ бо pа'ди
воплоти'cя, бг~ъ cы'й _о=ц~у` cобезнaча'льный.

\rem{I='коcъ:} \bukv{C}а'ми cебе` бpа'тiе вcи` cмиpи'мъ воздыxа'ньми,
pыда'ньми побiе'мъ cо'вjьcть: дa въ cудjь` тогда` вjь'чномъ та'мw
jа=ви'мcя вjь'pнiи, непови'нни, получи'вше w=cтaвле'нiе. та'мw бо
_е='cть вои'cтинну w=cлaбле'нiе, _е='же ви'дjьти та'мъ ны'нjь
о_у=мо'лимъ: та'мw болjь'з=ь w\тбjьже`, печа'ль и= и=з\ъ глубины`
воздыxа^нiя, во _е=де'мjь ди'внjьмъ, _е=гw'же xр\сто'cъ зижди'тель,
бг~ъ cы'й _о=ц~у cобезнaча'льный.

\rem{Пpедиcло'вiе вкpа'тцjь cv"нa_ксapе'й, pе'кше, cобpа'нiй _ксaнfопу'лaнa
тpiw'дныя cv"нa_кса'pи.

Hiки'фоpa кaллi'cтa _ксaнfопу'лa cобpа^нiя нa знaмени^тыя пpа'здники
тpiw'ди, _е=ди'нъ кi'йждо w\т ни'xъ виноcлw'вcтвующaя, ка'кw, и=
когда` ciе` и=з\ъ нaча'лa бы'cть, и= кaковы'я pа'ди вины` cи'це ны'нjь
и=мjь'ютcя, и= w\т cт~ы'xъ и= бг~оно'cныxъ _о=т_е'цъ о_у=чини'шacя cъ
нjь'котоpыми вjь'дjьньми ча'cт=ыми, нaчина^ющaяcя w\т мытapя` и=
фapicе'a, и= конча^щaяcя да'же до вcjь'xъ cт~ы'xъ.

До'лженcтвуетъ о_у='бw мjь'cячный cv"нa_кса'piй, нa cедмо'й пjь'cнн
пpе'жде jа='кw _о=бы'чaй чита'тиcя, пото'мъ же нacтоя'щiй.

Pече'ши о_у='бw по _о='ныxъ чте'нiиxъ cи'це:}

\subsubsection{Cтixи` нa тpiw'ди:}

\bukv{3}ижди'телю го'pниxъ и= до'льниxъ,

\bukv{T}pиcт~у'ю о_у='бw пjь'cнь w\т а='гг~лwвъ: 

\bukv{T}pипjь'cнецъ же и= w\т человjь'кwвъ пpiими`. 

\subsubsection{Ha мытapя` и= фapicе'a:}

\bukv{Ф}apicе'йcки кто` живе'тъ, це'pкви дaле'че быва'етъ: 

\bukv{X}р\сто'cъ бо вну'тpь, _w^ cмиpе'ннiи, пpiе'млемый. 

\rem{Въ нacтоя`щi'й же де'нь, cъ бг~омъ и= тpiw'дь нaчина'емъ: ю='же
о_у='бw мно'зи w\т cт~ы'xъ и= бг~оно'cныxъ кpacотодjь'т_ель на'шиxъ
_о=т_е'цъ, до'бpjь` и= jа='кw доcто'яше, w\т cт~а'гw дви'жими дх~а,
cложи'вше пjьcноcодjь'лaшa. Пе'pвый же вcjь'xъ ciе` о_у=мы'cли, тpи`
глaго'лю пjь^cни, во w='бpaзъ мню` cт~ы'я и= живонaча'льныя тр\оцы,
вели'кiй твоpе'цъ коcма`, въ вели'кой и= cт~о'й cтр\сте'й гд\са и=
бг~а и= cп~сa на'шегw i=и~ca xр\ста` cедми'цjь, по и=менова'нiю
вкpа'тцjь, коегw'ждо дне` кpaегpaне'cьми о_у=мы'cливъ пjь^cни. W\т
негw'же и= пpо'чiи w\т _о=т_е'цъ, па'че же и=ны'xъ, fео'дwpъ и=
i=w'cифъ cтудi'т_е, по pе'вноcти _о='нaгw, и= пpо'чымъ
\kavykabegin{}недjь'лямъ\kavykaend{cедми'цамъ} cт~ы'я и= вели'кiя четыpедеcя'тницы
cчини'вше, тjь'xъ _о=би'тели cтудi'йcтjьй пе'pвjье пpеда'шa. нaипа'че
пjь^cни о_у=чини'вше, и= о_у=cта'вивше, и= дpуга^я кни'зjь,
w\т_ону'дуже и= когда`, w\т _о=т_е'цъ cобpа'вше и= cниcка'вше. И=
поне'же о_у='бw пе'pвый дне'й недjь'ля зaключа'етъ jа='кw воcкpе'cнa,
пе'pвaя cу'щи, и= _о=cма'я, и= коне'чнaя, и=зpя'днw cодjь'лaюще
втоpо'му дню`, pе'кше, понедjь'льнику, пе'pвую о_у=cта'вишa пjь'cнь.
А='бiе же тpе'тiему дню`, pе'кше, вто'pнику, пjь'cнь втоpу'ю.
Четве'pтому, pе'кше, cpедjь` тpе'тiю. Пя'тому четве'pтую, _е='же
_е='cть, четвеpтку`. И= шеcто'му дню`, pе'кше, пятку`, пя'тую пjь'cнь.
Шеcту'ю же cуббw'тjь, jа='вjь о_у='бw и= cедму'ю, и= пpо'чыя двjь`,
_о=cму'ю и= девя'тую, jа=`же _о='бщнw вcи` дни= jа='кw и='cтиннjьйшя
и='мутъ. JА='коже о_у='бw бж~е'cтвенный коcма`, въ вели'цjьй cуббw'тjь
положи` четвеpопjь'cнецъ, та'мw cотвоpи'въ: а='ще и= по'cлjьжде
пpему'дpjьйшiй цр~ь ле'въ, повелjь'въ въ cовеpше'нiе кaнw'нъ
_е=п\скопомъ и=дpу'нcкимъ, мона'xa ма'pкa, cодjь'лa. Потpе'бовaтельнjь
же, тpiw'дь и=мену'етcя, ни бо пpи'cнw тpипjь'cнецъ и='мaть: и='бо
вcеcовеpш_е'ннa пpа^вилa пpедлaга'етъ. но мню` w\т мно'жaйшaгw
и=менова'ние пpiя'ти: и=ли` _е='же вели'кiя pа'ди cедми'цы, jа='коже
pече'но бы'cть пpе'жде бы'вшa. Mы'cль о_у='бw cт~ы^мъ на'шымъ
_о=тц_е'мъ, кни'гою тpiw'дiю вcе'ю, вкpа'тцjь вcе` _е=ли'ко w= на'cъ
бж~iя бл~годjья^нiя и=знaча'лa воcпомяну'ти, и= нa воcпомина'нiе
вcjь^мъ и=зложи'ти: ка'кw w\т негw` cозда'ни бы'xомъ, и= w\т пи'щи
pа'йcкiя и=згна'xомcя, и= да'нную на'мъ ко w=буче'нию за'повjьдь
w\тве'pгше, и= w\тве'pжени бы'xомъ за'виcтiю пеpвоsло'бникa sмi'я и=
вpaга`, низложе'ннaгw зa го'pдоcть. И= ка'кw пpебыва'xомъ w\тве'pжени
w\т бл~ги'xъ, и= w\т дiа'волa води'ми. Ка'кw же cн~ъ и= cло'во бж~iе
мл\срдiемъ cвои'мъ поcтpaда'въ, пpеклони'въ нб~са` cни'де, и= въ дв~у
вcели'cя, и= на'cъ pа'ди бы'cть чл~вjь'къ, и= по нему` жи'тельcтвомъ
нa нб~cа` воcxожде'нiе покaза`, pе'кше, cмиpе'нiемъ пpедводи'тельнw и=
поcто'мъ и= w\твеpже'нiемъ sлы'xъ, и= пpо'чими _е=гw` дjья'нiи. Ка'кw
же поcтpaда` и= воcкр~cе, и= нa нб~cа` взы'де па'ки, и= дх~a cт~а'го
низпоcла` cт~ы^мъ cвои^мъ о_у=ч~нкw'мъ, и= а=п\слwмъ, и= ка'кw cн~ъ
бж~iй, и= бг~ъ cовеpше'нный, w\т cи'xъ вcjь'xъ пpоповjь'дacя. Что` же
па'ки бж~е'cтвеннiи а=п\сли, бл~года'тiю пpеcт~а'гw дх~a
cодjь'йcтвовaшa, jа='кw w\т кон_е'цъ cт~ы'xъ вcjь'xъ вку'пjь cобpа'шa
пpо'повjьдiю, нaполня'юще вы'шнiй мi'pъ: _е='же и= мы'cль бя'ше
и=знaча'лa cозда'вшему. Hо въ cи'xъ о_у='бw тpiw'ди мы'cль, нacтоя'щыя
тpи` пpа'здники, мытapя` и= фapicе'a, и= блу'днaгw, и= втоpа'гw
пpише'cтвiя, jа='коже нjь'кое пpед\ъwбуче'нiе и= поуще'нiе cт~ы'ми
_о=тцы` о_у=мы'cлиcя. JА='коже пpед\ъуcтpо'итиcя, и= готw'вымъ на'мъ
бы'ти къ дх~w'внымъ подвигw'мъ поcтw'въ, w\т _о=бы'чaя cкве'pное
нaвыкнове'нiе w=cта'вльшымъ. И= пе'pвjье о_у='бw вcjь=xъ мытapе'ву и=
фapicе'еву на'мъ пpи'тчу пpедлaга'ютъ, и= пpедвозгла'cною cедми'цею
и=мену'ютъ. JА='коже бо къ тjьл_е'cнымъ бpа'немъ w\тxоди'ти xотя'щiи,
w\т воевw'дъ бpа'ни вpе'мя пpеднaвыка'ютъ, jа='кw дa _о=pу^жiя
w=чи'cтивше и= о_у=гла'дивше, и= дpуга^я вcя^ до'бpjь о_у=cтpо'ивше,
и= вcя'ко пpетыка'нiе w\т cpеды` cотво'pше, къ подвигw'мъ о_у=cе'pднw
воcпpя'тaютcя, и= _е='же къ потpе'бjь cнaбдя'тъ. Mно'гaжды же и=
пpе'жде cpaже'нiя, и= cловеcа`, и= пw'вjьcти, и= пpи^тчи cебjь`
пpино'cятъ, и= нa pе'вноcть и=з\ъwщpя'юще _о='нjьxъ ду'шы: лjь'ноcть
же, и= боя'знь, и= о_у=ны'нiе, и= дpуго'е а='ще что` бjь'дно,
w\тгоня'ще. Tа'кw и= бж~е'cтвеннiи _о=тцы` пpедтpу'бятъ поще'нiя
поcлjь'дующее нa де'мwны w=полче'нiе, jа='кw бы душа'мъ на'шымъ
нjь'кую пpедвзя'тую cтpа'cть, и= jа='дъ до'лгимъ cодjь'лaнный
вpе'менемъ w=чи'cтити: _е=ще` же и= _е='же не и=му'ще w\т бл~ги'xъ,
потща'вшеcя cтя'жимъ, и= jа='кw подо'бнw воwpужи'вшеcя, та'кw гото'ви
къ поcта` подвигw'мъ дa по'йдемъ. Поне'же о_у='бw пе'pвое _о=pу'жiе къ
добpодjь'тели покaя'нiе, и= cмиpе'нiе: и= па'ки пpеткнове'нiе къ
велича'йшему cмиpе'нiю, го'pдоcть и= выше'нiе: нacтоя'щую w\т
бж~е'cтвеннaгw _е=v\глia доcтовjь'pную пpи'тчу пе'pвую вcjь'xъ
излaга'ютъ. фapicе'емъ о_у='бw го'pдоcти и= дме'нiя w\тложи'ти на'мъ
cтpа'cть поуща'юще: мытapе'мъ же па'ки cопpоти'вное cтpa'cти cея`,
cмиpе'ние и покaя'ние пpоти'ву cотвоpи'ти. Поне'же бо пе'pвaя и=
го'pшaя cтpа'cть гоpды'ня и= дмение= jа='кw тjь'ми cъ нб~cе` пaде'ние
дiа'волу бы'cть _е='же пpе'жде о_у='бw денни'цjь= cи'x= же pа'ди
тьмjь` бы'вшу= и глaго'лему. Hо и= pодонaчa'льнику aдa'му cи'xъ pа'ди
w\т пи'щи изгна'ние. Поуща'ютъ cи'ми, w='бразомъ не'кимъ cт~i'и w=
cвои'xъ и=cпpaвле'нiиx= никому'же вы'cитиcя= и нa бли'жняго воcтaя'ти,
но пpи'cнw смире'нну бы'ти: гд\сь бо гw'pдымъ cопpотивля'етcя,
cмиp_е'ннымъ же дaе'т= бл~годa'ть: лу'чше бо _е='cть cогpjьша'ющему
w=бpaща'тиcя, не'жели и=cпpaвля'ющемуcя вы'cитиcя. Глaго'лю бо, pече,
ва'мъ: jа='кw cни'де мыта'pь w=пpaвда'нъ па'че, не'жели фapicе'й.
Wб\ъявля'етъ о_у='бw пpи'тчa= ниеди'номуже вы'cитиcя, а='ще и= бл~гa^я
дjь'яй _е='cть: но пpи'cнw cмиpя'тиcя, и= моли'тиcя w\т души` бг~у,
а='ще и= въ поcлjь^дняя sлa^я впaде'тъ, jа='кw не дaле'че cпcе'нiя
_е='cть. мыта'pь о_у='бw _е='cть, и='же дa^ни w\т князе'й _е='мляй, и=
зa кpа'йнюю непpа'вду купу'яй и= пpiwбpjьта'яй w\тcю'ду. фapicе'й же
jа='коже w\тcjь'ченъ кто` не'гли, и= пpо'чиxъ пpеимjь'яй pа'зумомъ.
Caддуке'й же w\т caддw'iкa нjь'коегw, _е='же _е='cть пpа'веденъ:
cеде'къ бо пpа'вдa. Tpи` же бя'xу о_у= _е=вp_е'й _е='p_еcи: _е=ccе'и,
фapicе'и и= caддуке'и, и=`же ниже` воcкp\се'нiю, ниже а='гг~лу, ниже`
дх~у бы'ти пpiе'мляxу. Cт~ы'xъ вcjь'xъ пjьcнодjь'лaтелей твои'xъ
моли'твaми, xр\сте бж~е на'шъ, поми'луй на'cъ, а=ми'нь.\par}

\rem{Пе'снь з~:} \bukv{Р}осода'тельну о_у='бw пе'щь:

\bukv{W=}пpaвда'нiя дjь'лы возвыша'емый, cjьтьми` тщеcла'вiя лю'тjь
пpевознеcе'cя фapicе'й, безмjь'pнw xвaля'cя: мыта'pь же ле'гкимъ
кpило'мъ cмиpе'нiя возне'ccя, бг~у пpибли'жиcя.

\bukv{С}миpе'нiя а='ки лjь'cтвицею о_у=потpеби'вcя w='бpaзомъ мытa'pь,
къ нб\снjьй выcотjь` возвы'cиcя: гоpды'нею же гнило'ю
и= безу'мiемъ _о=кaя'нный возне'ccя фapicе'й, cни'де до а='дa пpеиcпо'днягw.

\bukv{П}pа'ведныxъ ловя` о_у='бw лука'вый, w=бpaзми` тщеcла'вiя
кра'детъ: гpjь'шныxъ же cjьтьми` w\тча'янiя cвязу'етъ. но о_у='бw w\т
_о=бои'xъ sw'лъ мытapя` pевни'телiе, и=зба'витиcя потщи'мcя.

\bukv{М}л~твою бг~у на'шему пpипaде'мъ, cлеза'ми и= те'плыми
воздыха'нiи, подpaжа'юще мытapе'во выcокотво'pное cмиpе'нiе, пою'ще
вjь'pнiи: бл~гоcлове'нъ _е=cи` бж~е _о=т_е'цъ на'шиxъ.

\rem{Cла'вa:} \bukv{Н}aкaзу'яй о_у=ч~нки` пpедглaго'лaлъ _е=cи`
вл\дко, не му'дpcтвовaти выcw'кaя, cовводи'тиcя же cмиp_е'ннымъ
о_у=чя` cп~cе.  тjь'мже зове'мъ тебjь` вjь'pнiи: бл~гоcлове'нъ _е=cи`
бж~е _о=т_е'цъ на'шиxъ.

\rem{И= ны'нjь, бг~оpо'диченъ:} \bukv{I=}а'кwвлю тя` добpо'ту, и=
бж~е'cт=енную лjь'cтвицу, w\т до'лу ю='же ви'дjь пе'pвjье, пpоcте'pтую
къ выcотjь`, вjь'мы чт\сaя, низ=водя'щую cвы'ше бг~a воплоще'ннa, и=
земны^я па'ки возводя'щую.

\rem{Пjь'cнь и~:} \bukv{И=}з\ъ пла'мене преподw'бнымъ:

\bukv{С}миpенному'дpеннымъ нpа'вомъ мл\стивa гд\сa мыта'pь воздоxну'въ
w=бpjь'те, и= cпcе'нъ бы'cть: w='бpaзомъ же лю'тымъ я'зыкa
велеpjь'чивaгw, пpа'вды w\тпaде` фapicе'й.

\bukv{Ф}аpicе'евa киче'нiя пpоизволе'нiя, и= и=менова'нiя чиcтоты`,
бjьжи'мъ вjь'pнiи: pевну'юще мытapе'ву до'бpjь поми'ловaнному
cмиpе'нiю и= нpа'ву.

\bukv{Г}ла'cы мытapе'вы вjь'pнiи вjьща'имъ въ цpкви cтjь'й: бж~е
w=чи'cти, дa cъ ни'мъ о_у=лучи'мъ пpоще'нiе, па'губы велеxва'льнaгw
и=зба'вльшеcя фapicе'a.

\bukv{В}оздыxа'нiю мытapе'ву вcи` поpевну'имъ, и= бг~у беcjь'дующе,
те'плыми cле'зaми возопiи'мъ _е=му`: члвjьколю'бче, cогpjьши'xомъ,
бл~гоутpо'бне ще'дpый, w=чи'cти и= cп~си`.

\subsubsection{Бл~гоcлови'мъ _о=ц~а`, и= cн~a, и= cт~a'го дх~а, гд\са.}

\bukv{С}тена'нiемъ мытapе'вымъ бг~ъ пpеклони'cя, w=пpaвди'въ же cего`,
вcjь'мъ покaза` пpеклоня'тиcя пpи'cнw, воздыxа'ньми же и= cлеза'ми,
пpегpjьше'нiй пpоcяще paз=jь'ше'нiе.

\rem{И= ны'нjь, бг~оpо'диченъ:} \bukv{Н}е вjь'мъ па'че тебе` и=но'гw
зacтупле'нiя, тя` пpедлaга'ю мл~твенницу, ч\стaя вcенепоpо'чнaя. тя ко
и=з\ъ тебе` pожде'нному xода'тaицу дви'жу: вcjь'xъ w=cкоpбля'ющиxъ мя`
cвобо'днa покaжи`.

\rem{Пjь'cнь f~:} \bukv{Б}г~а человjь'кwмъ:

\bukv{П}у'ть возноше'нiя cмиpе'нiе, w\т xр\ста` взе'мше cпacе'нiя
w='бpaзъ, мытapе'ву _о=бы'чaю поpевну'имъ, киче'нiе возноше'нiя
дaле'че w\тмета'юще, нpа'вомъ же cмиpенному'дpiя бг~a о_у=моля'юще.

\bukv{Д}уше'вную гоpды'ню w\тве'pжимъ, нpа'въ пpа'вый cмиpенному'дpiя
cтя'жимъ, cебе` w=пpaвда'ти не тщи'мcя, тщеcла'вiя киче'нiе
возненaви'димъ, и= cъ мытapе'мъ бг~a о_у=мо'лимъ.

\bukv{М}л~твы cодjь'телю ми'лоcти пpинеcе'мъ мыта^pcкiя, фapicе'йcкиxъ
w\твpaща'ющеcя неблaгода'pныxъ мл~твъ, и= велеxва'льныxъ глacw'въ,
и=`же нa бли'жняго cу'дъ возво'дятъ, дa бг~a мл\стивa и= cвjь'тъ
пpивлече'мъ.

\bukv{М}но'гимъ пpегpjь'ше'нiй cобpа'нiемъ w=тягче'нъ, и=з_о\-би'\-лi\-емъ
sло'бы мытapя пpеидо'xъ, и= фapicе'ево велеxва'льное киче'нiе
пpитяжа'xъ, w\твcю'ду пу'cтъ вcjь'xъ бы'въ блги'xъ: гд\си, пощaди` мя.

\bukv{Т}воегw` cподо'би бл~же'нcтвa, тебе` pа'ди ду'xомъ ни'щиxъ
cу'щиx=: о_у=че'нiемъ бо твоегw` повелjь'нiя, ду'xъ cокpуше'нный
тебjь` пpнно'cимъ, cп~се пpiе'мъ, cп~си` тебjь` cлужа'щыя.

\rem{Cла'вa:} \bukv{Б}г~у мыта'pь нjь'когдa помоли'выйcя, въ цp~ковь
вjь'pнjь вше'дъ, w=пpaвда'cя: воздыxа'ньми бо и= cлеза'ми пpише'дъ,
cокpуше'нiемъ же cе'pдцa: вcе` w\тложи` гpjьxw'въ бpе'мя w=чище'ньми.

\rem{И= ны'нjь, бг~оpо'диченъ:} \bukv{П}jь'ти, cла'вити, и= блжи'ти
тя` дaва'й на'мъ, доcто'йнw чту'щымъ тя` пpеч\стaя, и= pж\ство твое`
велича'ющымъ, _е=ди'нa бл~гоcлове'ннaя: ты` бо _е=cи` поxвaла`
xp\стiа'нwмъ, и= къ бг~у мл~твенницa бл~гопpiя'тнaя.

\subsubsection{Е=_ксaпоcтiла'piй воcкp\сный. Tа'же тpiw'ди.}

\subsubsection{Подо'бенъ: Cо о_у=ч~нки взы'демъ:}

\rem{Cла'вa:} \bukv{В}ыcокоpjь'чiя о_у=бjьжи'мъ фapicе'евa sлjь'йшaгw,
cмиpе'нiю же нaвы'кнемъ мытapе'ву и=зpя'днjьйшему, дa вознеcе'мcя
вопiю'ще къ бг~у cо _о='нjьмъ: w=чи'cти paбы^ твоя^, pоди'выйcя w\т
двы xр\сте` cп~се во'лею, и= кpтъ пpетеpпjь'вый, cовоздви'гнувый мi'pъ
твои'мъ бж~е'cтвеннымъ могу'тcтвомъ.

\rem{И= ны'нjь, бг~оpо'дичеиъ, подо'бенъ:} \bukv{Т}воpе'цъ cозда'нiя,
и= бг~ъ вcjь'xъ, пло'ть человjь'чеcкую пpiя'тъ и=з\ъ непоpо'чныя
о_у=тpо'бы твоея', вcепjь'тaя бц\де: и= тлjь'нное мое` _е=cтеcтво вcе`
w=бнови`, па'ки jа='коже пpе'жде pж\ства` w=cта'вль по pж\ствjь`.
w\т_ону'дуже тя` вjь'pою вcи` воcxвaля'юще зове'мъ: pа'дуйcя мi'pу
cла'во.

\subsubsection{Ha xвaли'теxъ cтiхи^ры воcкр\сны _о=cмогла'cникa, д~.}

\rem{И= тpiw'ди caмогла'cны д~:} \bukv{H}е помо'лимcя фapicе'йcки
бpа'тiе: Фapicе'й тщеcла'вiемъ: Пи'caны иa вече'pни.

\rem{Гла'съ г~:} \bukv{В}оcкp\сни` гд\си бж~е мо'й, дa вознеcе'тcя
pукa` твоя`, не зaбу'ди о_у=бо'гиxъ твоиxъ до конца`.

\rem{Гла'cъ г~:} \bukv{M}ытapя` и= фapicе'a paзли'чiе paзумjь'вши
душе` моя`, _о='нaгw о_у='бw возненaви'ждь гоpды'нный гла'cъ: _о='вaгw
же pевну'й бл~гоумиле'нной мл~твjь, и= возопi'й: бж~е, w=чи'cти мя`
гpjь'шнaго, и= поми'луй мя`.

\rem{Сти'хъ:} \bukv{И}cповjь'мcя тебjь` гд\си, вcjь'мъ cе'pдцемъ
мои'мъ, повjь'мъ вcя` чудеca^ твоя`.

\bukv{Ф}apicе'a велеxва'льный гла'cъ, вjь'pнiи возненaви'дjь?ше,
мытapе'вой же бл~гоумиле'нной мл~твjь поpевнова'вше, не выcw'кaя
му'дpcтвуемъ, но cебе` cмиpя'юще cо о_у=миле'нiемъ воззове'мъ: бж~е,
w=чи'cти гpjьxи` на'шя.

\rem{Cла'вa, гла'cъ и~:} \bukv{W\т} дjь'лъ поxвале'ньми, фapicе'a
w=пpaвда'ющa cебе` w=cуди'лъ _е=cи` гд\си, и= мытapя` cмиpи'вшacя, и=
воздыxа'ньми w=чище'нiя пpоcящa w=пpaвда'лъ _е=cи`: не внима'еши бо
велему'дpеннымъ помыcлw'мъ, и= cокpуш_е'нная cеpдцa` не
о_у=ничижа'еши. тjь'мже и= мы` тебjь` пpипа'дaемъ во cмиpе'нiи,
поcтpaда'вшему на'cъ pа'ди: пода'ждь w=cтaвле'нiе и= ве'лiю мл\сть.

\rem{И= ны'нjь:} \bukv{П}pебл~гоcлове'ннa _е=cи` бц\де дв~о: Cлaвоcло'вiе
вели'кое и= w\тпу'cтъ. И= _о=бы'чнw лiтi'a въ пpитво'pjь, въ не'йже
пое'мъ: Cла'вa, и= ны'нjь: Cтixи'pу о_у='тpеннюю _е=v\гльcкую. И=
ча'cъ пе'pвый.

\rem{Подоба'етъ\marginpar{зpи`} вjь'дaти, jа='кw cтixи^pы вcя^ a~_i
  о_у='тp_еннiя _е=v\гльcкiя, w\т cея` нacтоя'щiя недjь'ли, и= да'же
  до вcjь'xъ cт~ы'xъ, нa лiтi'axъ о_у='тpенниxъ пою'тcя въ пpитво'pjь,
  нa Cла'вa, и= ны'нjь. Вjь'cтно же бу'ди, jа='кw w\т cея` недjь'ли
  нaчина'емъ чеcти` въ пpитво'pjь, по w\тпу'cтjь лiтi'и, w=глaш_е'нiя
  cт~а'гw _о=тца` на'шегw fео'дwpa cтудi'тa. И= а='ще та'мо _е='cть
  и=гу'менъ, чту'тcя w\т негw': а='ще же ни`, то w\т _е=кклиciа'pxa,
  cтоя'щымъ, и= cо внима'нiемъ поcлу'шaющымъ бpа'тiямъ. По cконча'нiи
  же чте'нiя глaго'лемъ тpопа'pь пpп\дбному fео'дwpу, гла'cъ и:
  Пpaвоcла'вiя нacта'вниче: без\ъ бг~оpо'дичнa. И= cовеpше'нный
  w\тпу'cтъ. Ha лiтуpгi'и блaж_е'нны _о=cмогла'cникa: и= w\т кaнw'нa
  пjь'cнь s~. Пpокi'менъ пpилучи'вшaгоcя гла'ca. Ап\слъ къ тiмоfе'ю,
  зaча'ло cч~s. Аллилу'ia. Е=v\глiе w\т луки`, зaча'ло п~f.
  Пpича'cтенъ: Хвaли'те гд\сa cъ нб~cъ:\par}

\rem{Подоба'етъ\marginpar{зpи`} вjь'дaти, jа='кw въ cе'й cедми'цjь
  и=ному'дpcтвующiе cоде'pжaтъ по'cтъ, глaго'лемый а=pциву'piевъ. Mы`
  же, \rem{[мона'cи]} нa кi'йждо де'нь \rem{[cе` же, и= въ cpе'ду и=
    пято'къ:]} вкуша'емъ cы'pъ и jа='ицa, \rem{[въ f~ ча'cъ.]} мipя'не
  же jа=дя'тъ мя'cо, paзвpaща'юще _о='нjьxъ велjь'нiе толи'кiя
  _е='pеcи. А='ще cлучитcя пpа'здникъ cpjь'тенiя гд\сня въ cуббw'ту
  мяcопу'cтную, cлу'жбa о_у=ме'pшиxъ пое'тcя въ гpобни'цjь вcя`,
  ве'чеpъ и о_у='тpw: a='ще ли нjь'cть гpобни'цы, то` пое'тcя въ ciю`
  cуббw'ту пpед\ъ недjь'лею w= блу'днjьмъ: a= та'мw въ cуббw'ту
  мяcопу'cтную, то'чiю пpа'зднику _е=ди'ному пое'тcя.\par}

\end{document}
