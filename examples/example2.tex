\documentclass[12pt,twoside,a6paper,dvips,civil=antiqua,cs=izhitsa]{hipbook}

% ==========================================================================================

\tolerance 5000

\newcommand{\eirmos}[1]{{\small\rem{I=рмо'съ:} #1\par}}
\newcommand{\refrain}[1]{\vskip 0.3\baselineskip{\footnotesize{#1\par}}}
\newcommand{\xpref}[1]{\rem{#1:}\ }
\newcommand{\deacon}{\rem{Дiа'конъ:}\ }
\newcommand{\iereios}{\rem{I=ере'й:}\ }
\newcommand{\lik}{\rem{Ли'къ:}\ }

\renewcommand{\*}{~~\raise3pt\hbox{\footnotesize*}}

\newcommand{\num}[2]{\leavevmode\raise3pt\hbox{{\footnotesize[#1]}~~}#2}
%\newcommand{\num}[2]{#2\marginpar[abc]{\footnotesize#1}}
  %\footnotesize#1]
%  {\footnotesize#1}}

\stdcrosstitle{
{\Large\MakeUppercase{Божественная литургiя}}\\
\vskip 0.5\baselineskip
\large и='же во святы'хъ _о=тца` на'шегw I=wа'нна Златоу'стагw}

% ==========================================================================================

\begin{document}

\maketitle

\stdsecondpage
\clearpage

% ==========================================================================================
\hdrcrosspage

\section[Боже'ственная литургi'я]{\large\MakeUppercase{Боже'ственная литургi'я}}
\subsubsection{и='же во святы'хъ _о=тца` на'шегw I=wа'нна Златоу'стагw}

\deacon{} \bukv{Б}лагослови`, влады'ко.  

\iereios{} \bukv{Б}лагослове'нно ца'рство _о=ц~а`, и= сн~а, и= ст~а'гw дх~а,
ны'нjь и= при'снw, и= во вjь'ки вjькw'въ.

\lik{} \bukv{А=}ми'нь.

\subsection{Е=ктенiа` вели'кая:}

\deacon{} \bukv{М}и'ромъ гд\су помо'лимся.

\xpref{Ли'къ по ко'емждо проше'нiи пое'тъ} \bukv{Г}д\си, поми'луй.

\bukv{W=} свы'шнемъ ми'рjь, и= сп~се'нiи ду'шъ на'шихъ, гд\су
помо'лимся.

\bukv{W=} ми'рjь всегw` мi'ра, благостоя'нiи ст~ы'хъ бж~iихъ цр~кве'й
и= соедине'нiи всjь'хъ, гд\су помо'лимся.

\bukv{W=} святjь'йшихъ патрiа'рсjьхъ правосла'вныхъ,
преwсвяще'ннjьйшемъ _е=пи'скопjь на'шемъ \rem{[и=м\ркъ]}, честнjь'мъ
пресвv'терствjь, во хр\стjь дiа'конствjь, w= все'мъ при'чтjь и=
лю'дjьхъ, гд\су помо'лимся.

\bukv{W=} странjь` на'шей, прави'тельствjь и= во'инствjь, гд\су
помо'лимся.

\bukv{W=} гра'дjь се'мъ \rem{[ве'си, и=ли` _о=би'тели се'й]}, вся'комъ
гра'дjь, странjь, и= вjь'рою живу'щихъ въ ни'хъ, гд\су помо'лимся.

\bukv{W=} пла'вающихъ, путеше'ствующихъ, неду'гующихъ, стра'ждущихъ,
плjьне'нныхъ, и= w= спасе'нiи и='хъ, гд\су помо'лимся.

\rem{[Здjь` прилага'ются, а='ще потре'бно, моле^нiя w= боля'щихъ и=
  и=ны^я.]}

\bukv{W=} и=зба'витися на'мъ w\т вся'кiя ско'рби, гнjь'ва и= ну'жды,
гд\су помо'лимся.

\bukv{З}аступи`, спаси`, поми'луй и= сохрани` на'съ, бж`е, твое'ю
бл~года'тiю.

\lik{} \bukv{Г}д\си, поми'луй.

\bukv{П}рест~у'ю, преч\стую, пребл~гослове'нную, сла'вную вл\дчицу
на'шу бц\ду и= приснодв~у мр~i'ю, со всjь'ми ст~ы'ми помяну'вши, са'ми
себе` и= дру'гъ дру'га, и= ве'сь живо'тъ на'шъ хр\сту бг~у предади'мъ.

\lik{} \bukv{Т}ебjь`, гд\си.

\iereios{} \bukv{Jа='}кw подоба'етъ тебjь` вся'кая сла'ва, че'сть и=
поклоне'нiе _о=ц~у`, и= сн~у, и= ст~о'му дх~у, ны'нjь и= при'снw, и=
во вjь'ки вjькw'въ.

\lik{} \bukv{А=}ми'нь.

\subsection{И=зwбрази'тельныя}

{\rem{Глаго'лются въ недjь'лю, въ пра'здники богоро'дичны и=
    наро'читыхъ ст~ы'хъ, въ предпра'зднства и= попра'зднства, во всю`
    пятидеся'тницу и= коему'ждо ст~о'му, _е=му'же поло'жены во
    о_у=ста'вjь и=ли` мине'и блаже'нны.\par}
  
\xpref{_Псало'мъ р~в} {\large\bukv{Б}лагослови`, душе` моя`, гд\са,
    бл~госло\=ве'нъ _е=си`, гд\си.
}
  
{\large\bukv{Б}лагослови`, душе` моя`, гд\са, и= вся^ вну'тр_eнняя
  моя^ и='мя ст~о'е _e=гw`.\par}

{\large\bukv{Б}лагослови`, душе` моя`, гд\са, и= не забыва'й всjь'хъ
воздая'нiй _e=гw`:\par}

{\large\bukv{W=}чища'ющаго вся^ беззакw'нiя твоя^, и=сцjьля'ющаго
вся^ неду'ги твоя^:\par}

\bukv{И=}збавля'ющаго w\т и=стлjь'нiя живо'тъ тво'й, вjьнча'ющаго тя`
ми'лостiю и= щедро'тами:

\bukv{И=с}полня'ющаго во благи'хъ жела'нiе твое`: w=бнови'тся
ja='кw _о='рля ю='ность твоя`.

\bukv{Т}воря'й ми'лwстыни гд\сь, и= судьбу` всjь^мъ w=би'димымъ.

\bukv{С}каза` пути^ своя^ мwv"се'ови, сыновw'мъ _i=и~л_eвымъ хотjь^нiя своя^.

{\large\bukv{Щ}е'дръ и= ми'лостивъ гд\сь, долготерпjьли'въ и= многомл\стивъ.}

\bukv{Н}е до конца` прогнjь'вается, ниже` во вjь'къ вражду'етъ.

\bukv{Н}е по
беззако'нi_eмъ на'шымъ сотвори'лъ _e='сть на'мъ, ниже` по грjьхw'мъ
на'шымъ возда'лъ _e='сть на'мъ. 

\bukv{JА='}кw по высотjь` нб\снjьй w\т
земли`, о_у=тверди'лъ _e='сть гд\сь мл\сть свою` на боя'щихся _e=гw`.

\bukv{Е=}ли'кw w\тстоя'тъ восто'цы w\т за^падъ, о_у=да'лилъ _e='сть w\т на'съ
беззакw'нiя на^ша. 

\bukv{JА='}коже ще'дритъ _о=те'цъ сы'ны, о_у=ще'дри гд\сь
боя'щихся _e=гw`. 

\bukv{JА='}кw то'й позна` созда'нiе на'ше, помяну`, ja='кw
пе'рсть _e=смы`. 

\bukv{Ч}еловjь'къ, ja='кw трава` дн_i'е _e=гw`, ja='кw
цвjь'тъ се'льный, та'кw w=цвjьте'тъ: 

\bukv{JА='}кw ду'хъ про'йде въ не'мъ, и=
не бу'детъ, и= не позна'етъ ктому` мjь'ста своегw`. 

\bukv{М}и'лость же гд\сня
w\т вjь'ка и= до вjь'ка на боя'щихся _e=гw`, 

\bukv{И=} пра'вда _e=гw` на
сынjь'хъ сынw'въ, храня'щихъ завjь'тъ _e=гw`, и= по'мнящихъ за'пwвjьди
_e=гw` твори'ти я=`. 

\bukv{Г}д\сь на нб~си` о_у=гото'ва пр\сто'лъ сво'й, и=
цр\ство _e=гw` всjь'ми w=блада'етъ. 

\bukv{Б}лагослови'те гд\са вси` а='гг~ли
_e=гw`, си'льнiи крjь'постiю, творя'щiи сло'во _e=гw`, о_у=слы'шати
гла'съ слове'съ _e=гw`. 

\bukv{Б}лагослови'те гд\са вся^ си^лы _e=гw`, слуги^
_e=гw`, творя'щiи во'лю _e=гw`. 

\bukv{Б}лагослови'те гд\са вся^ дjьла`
_e=гw`, на вся'комъ мjь'стjь вл\дчества _e=гw`.

{\large\bukv{Б}лагослови`, душе` моя`, гд\са, и= вся^ вну'тр_eнняя моя^ и='мя ст~о'е
_e=гw`.\par}

{\large\bukv{Б}лагослови`, душе` моя`, гд\са, бл~гослове'нъ _е=си`, гд\си.}

{\rem{Нjь'цыи же пою'тъ то'лько здjь` w\тмjь'ченныя стихи`.\par}

\subsubsection{А=нтифw'нъ а~ по вся` дни~:\\
(Е=гда` не поло'жено по о_у=ста'ву пjь'сни на бл~же'нны.)}

\bukv{Б}ла'го _е='сть и=сповjь'датися гд\сви, и= пjь'ти и='мени твоему`,
вы'шнiй.

\refrain{\bukv{М}оли'твами бц\ды, сп~се спаси` на'съ.}

\bukv{В}озвjьща'ти зау'тра ми'лость твою`, и= и='стину твою` на вся'ку но'щь.

\refrain{\bukv{М}оли'твами бц\ды, сп~се спаси` на'съ.}

\bukv{JА='}кw пра'въ гд\сь бг~ъ на'шъ, и= нjь'сть непра'вды въ не'мъ.

\refrain{\bukv{М}оли'твами бц\ды, сп~се спаси` на'съ.}

\rem{Сла'ва и= ны'нjь: Припjь'въ то'йже}

\rem{По сконча'нiи а~ а=нтифw'на дiа'конъ глаго'летъ _е=ктенiю`
  ма'лую:}

\bukv{П}а'ки и= па'ки ми'ромъ гд\су помо'лимся.

\lik{\bukv{Г}д\си, поми'луй.}

\bukv{П}рест~у'ю, преч\стую, пребл~гослове'нную, сла'вную вл\дчцу
на'шу бц\ду и= приснодв~у марi'ю со всjь'ми ст~ы'ми помяну'вше, са'ми
себе` и= дру'гъ дру'га, и= ве'сь живо'тъ на'шъ хр\сту` бг~у
предади'мъ.

\lik{\bukv{Т}ебjь`, гд\си.}

\xpref{\bukv{В}о'згласъ} \bukv{JА='}кw твоя` держа'ва, и= твое`
_е='сть ца'рство, и= си'ла и= сла'ва, _о=ц~а`, и= сн~а, и= ст~а'гw
дх~а, ны'нjь и= при'снw, и= во вjь'ки вjькw'въ.

\lik{\bukv{А=}ми'нь.}

\subsubsection{Та'же в~ а=нтифw'нъ и=зwбрази'тельныхъ}

\bukv{С}ла'ва _о=ц~у`, и= сн~у, и= ст~о'му дх~у:

{\large\bukv{Х}вали` душе` моя` гд\са: Восхвалю` гд\са въ животjь`
  мое'мъ, пою` бг~у моему`, до'ндеже _e='смь.\par}

{\large\bukv{Н}е надjь'йтеся на кня^зи, на сы'ны человjь'ч_eскiя, въ
  ни'хже нjь'сть сп~се'нiя.\par}

{\large\bukv{И=}зы'детъ ду'хъ _e=гw`, и= возврати'тся въ зе'млю свою`:
  въ то'й де'нь поги'бнутъ вся^ помышл_e'нiя _e=гw`.\par}

\bukv{Б}л~же'нъ, _eму'же бг~ъ i=а'кwвль помо'щникъ _e=гw`,
о_у=пова'нiе _e=гw` на гд\са бг~а своего`:

\bukv{С}отво'ршаго нб~о и= зе'млю, мо'ре и= вся^, ja=`же въ ни'хъ:

\bukv{Х}раня'щаго и='стину въ вjь'къ: Творя'щаго су'дъ w=би^димымъ:
даю'щаго пи'щу а='лчущымъ.

\bukv{Г}д\сь рjьши'тъ w=кова^нныя: Гд\сь о_у=мудря'етъ слjьпцы`:

\bukv{Г}д\сь возво'дитъ низве'рж_eнныя: гд\сь лю'битъ прв\дники:

\bukv{Г}д\сь храни'тъ прише'льцы: си'ра и= вдову` прiи'метъ, и= пу'ть грjь'шныхъ погуби'тъ.

{\large\bukv{В}оцари'тся гд\сь во вjь'къ, бг~ъ тво'й сiw'не въ ро'дъ
  и= ро'дъ.\par}

\subsubsection{И= ны'нjь, гла'съ s~:}

%Пjь'снь гд\су i=и~су хр\сту`

\begin{large}
\bukv{Е=}диноро'дный сн~е и= сло'ве бж~iй, безсме'ртенъ сы'й,\* и= и=зво'ливый
сп~се'нiя на'шегw ра'ди\* воплоти'тися w\т ст~ы'я бц\ды и= приснодв~ы
мр~i'и,\* непрело'жнw вочеловjь'чивыйся,\* распны'йся же хр\сте` бж~е,
сме'ртiю сме'рть попра'вый,\* _e=ди'нъ сы'й ст~ы'я тр\оцы,\*
спрославля'емый _о=ц~у` и= ст~о'му дх~у, сп~си` на'съ.\par
\end{large}

\subsubsection{А=нтифw'нъ в~ повседне'вный:}

\bukv{Г}д\сь воц~ри'ся, въ лjь'поту w=блече'ся.

\refrain{\bukv{М}л~твами ст~ы'хъ твои'хъ, сп~се сп~си` на'съ.}

\bukv{Г}д\сь воц~ри'ся, въ лjь'поту w=блече'ся, w=блече'ся гд\сь въ
си'лу, и= препоя'сася.

\refrain{\bukv{М}л~твами ст~ы'хъ твои'хъ, сп~се сп~си` на'съ.}

\bukv{И='}бо о_у=тверди` вселе'нную, jа='же не подви'жится.

\refrain{\bukv{М}л~твами ст~ы'хъ твои'хъ, сп~се сп~си` на'съ.}

\bukv{С}видjь^нiя твоя^ о_у=вjь'ришася sjьлw`: до'му твоему`
подоба'етъ ст~ы'ня гд\си, въ долготу` днi'й.

\rem{Сла'ва, и= ны'нjь:} \bukv{Е=}диноро'дный сн~е:

\rem{Е=ктенiя` ма'лая:} \bukv{П}а'ки и= па'ки: [Гд\си поми'луй,
\rem{в~}]

\xpref{Во'згласъ} \bukv{JА='}кw бл~гъ и= чл~вjьколю'бецъ бг~ъ _е=си`,
и= тебjь` сла'ву возсыла'емъ, _о=ц~у`, и= сн~у, и= ст~о'му дх~у,
ны'нjь и= при'снw, и= во вjь'ки вjькw'въ.

\lik{\bukv{А=}ми'нь} \rem{[в~]}

\subsection[Бл~же'нны]{Та'же бл~же'нны:}

\begin{large}

\bukv{В}о цр\ствiи твое'мъ помяни` на'съ, гд\си,\* _e=гда` прiи'деши во
цр\ствiи твое'мъ.

\bukv{Б}л~же'ни ни'щiи ду'хомъ,\* ja='кw тjь'хъ _e='сть цр\ство нб\сное.

\bukv{Б}л~же'ни пла'чущiи,\* ja='кw тi'и о_у=тjь'шатся.

\bukv{Б}л~же'ни кро'тцыи,\* ja='кw тi'и наслjь'дятъ зе'млю.

\bukv{Б}л~же'ни а=лчущiи и= жа'ждущiи пра'вды,\* ja='кw тi'и насы'тятся.

\bukv{Б}л~же'ни ми'лостивiи,\* ja='кw тi'и поми'ловани бу'дутъ.

\bukv{Б}л~же'ни чи'стiи ср\дцемъ,\* ja='кw тi'и бг~а о_у='зрятъ.

\bukv{Б}л~же'ни миротво'рцы,\* ja='кw тi'и сн~ове бж~iи нареку'тся.

\bukv{Б}л~же'ни и=згна'ни пра'вды ра'ди,\* ja='кw тjь'хъ _e='сть
цр\ство нб\сное.

\bukv{Б}л~же'ни _e=сте`, _e=гда` поно'сятъ ва'мъ,\* и= и=ждену'тъ, и=
реку'тъ вся'къ sо'лъ глаго'лъ, на вы` лжу'ще мене` ра'ди.

\bukv{Р}а'дуйтеся и= весели'теся,\* ja='кw мзда` ва'ша мно'га на
нб~сjь'хъ.

\end{large}

\subsubsection{А=нтифw'нъ г~ повседне'вный:}

\bukv{П}рiиди'те возра'дуемся гд\севи, воскли'кнемъ бг~у сп~си'телю на'шему.

\refrain{\bukv{С}п~си' ны сн~е бж~iй, во ст~ы'хъ ди'венъ сы'й, пою'щiя ти`:
  а=ллилу'iа.}

\bukv{П}редвари'мъ лице` _е=гw` во и=сповjь'данiи, и= во _псалмjь'хъ
воскли'кнемъ _е=му`.

\refrain{\bukv{С}п~си' ны сн~е бж~iй:}

\bukv{JА='}кw бг~ъ ве'лiй гд\сь, и= ца'рь ве'лiй по все'й земли`.

\refrain{\bukv{С}п~си' ны сн~е бж~iй:}

\bukv{JА='}кw тогw` _е='сть мо'ре, и= то'й сотвори` _е=`, и= су'шу ру'цjь
_е=гw` созда'стjь.

\refrain{\bukv{С}п~си' ны сн~е бж~iй:}

%\rem{Сла'ва, и= ны'нjь:} 

\subsection{Вхо'дъ съ _е=v\глiемъ}

\deacon{\bukv{П}рему'дрость, про'сти.}

\lik{\begin{large}
\bukv{П}рiиди'те, поклони'мся и= припаде'мъ ко хр\сту`.\* Спаси' ны, сн~е
бж~iй, во ст~ы'хъ ди'венъ сы'й, пою'щыя ти`: А=ллилу'йа.\par
  \end{large}} \rem{Е=ди'ножды.}

\rem{А='ще недjь'ля:}\begin{large}
Воскресы'й и=зъ ме'ртвыхъ, пою'щыя ти`: А=ллилу'йа.\par
\end{large} \rem{Е=ди'ножды.}

\rem{А='ще пра'здникъ бг~оро'диченъ:} 
\begin{large}
  Моли'твами бц\ды, пою'щыя ти: А=ллилу'йа.\par
\end{large} \rem{Е=ди'ножды.}

\subsubsection{Та'же w=бы'чныя тропари` и= кондаки` по о_у=ста'ву.}

\rem{\normalsizeИ= ны'нjь:} \rem{Конда'къ хра'ма бг~оро'дицы, и=ли` гла'съ
  s~:}

\bukv{П}редста'тельство хр\стiа'нъ непосты'дное,\* хода'тайство ко творцу`
непрело'жное,\* не пре'зри грjь'шныхъ моле'нiй гла'сы,\* но предвари`, jа='кw
бл~га'я, на по'мощь на'съ, вjь'рнw зову'щихъ ти`:\* о_у=скори' на мл~тву,\* и=
потщи'ся на о_у=моле'нiе,\* предста'тельствующе при'снw бц\де, чту'щихъ тя`.

\iereios{\bukv{JА='}кw ст~ъ _е=си` бж~е на'шъ, и= тебjь` сла'ву возсыла'емъ,
  _о=ц~у`, и= сн~у, и= ст~о'му дх~у, ны'нjь и= при'снw:}

\deacon{\bukv{Г}д\си, сп~си` бл~гочести'выя, и= о_у=слы'ши ны`}. \rem{Ли'къ
  то'же.}

\deacon{\bukv{И=} во вjь'ки вjькw'въ.}

\lik{\bukv{А=}ми'нь.} \rem{И= пое'тъ трист~о'е:}

\bukv{С}т~ы'й бж~е, ст~ы'й крjь'пкiй, ст~ы'й безсме'ртный, поми'луй
на'съ. \rem{три'жды}.

\rem{Сла'ва и= ны'нjь:} \bukv{С}т~ы'й ст~ы'й безсме'ртный, поми'луй
на'съ.

\bukv{С}т~ы'й бж~е, ст~ы'й крjь'пкiй, ст~ы'й безсме'ртный, поми'луй
на'съ.

\rem{Въ вели'кiя пра'здники пое'мъ} \bukv{Е=}ли'цы: \rem{jа='коже о_у=ка'зано
  въ концjь` сiя` кни'ги.\par}

\deacon{\bukv{В}о'нмемъ.}

\iereios{\bukv{М}и'ръ всjь'мъ.}

\rem{И= чте'цъ проки'менъ [зри` въ концjь` се'й кни'ги].}

\rem{Посе'мъ} \deacon{\bukv{П}рему'дрость.}

\subsection[А=п\столъ и= _е=v\глiе]{И= чте'цъ надписа'нiе а=п\стола:}

\bukv{Д}jья'нiй ст~ы'хъ а=по'стwлъ чте'нiе. \rem{И=ли`} \bukv{С}обо'рнагw
посла'нiя i=а'кwвля, \rem{и=ли`} петро'ва чте'нiе. \rem{И=ли`} \bukv{К}
ри'млянwмъ, \rem{и=ли`} къ корi'нfянwмъ, \rem{и=ли`} къ гала'тwмъ посла'нiя
ст~а'гw а=по'стола па'вла чте'нiе.

\rem{Дiа'конъ па'ки:}} \bukv{В}о'нмемъ. \rem{Чте'цъ чте'тъ а=по'столъ.}

\rem{По сконча'нiи i=ере'й:} \bukv{М}и'ръ ти`.

\lik{А=ллилу'iа.} \rem{Три'жды.}

\deacon{\bukv{П}рему'дрость, про'сти, о_у=слы'шимъ ст~а'гw _е=vа'нгелiя.}

\iereios{\bukv{М}и'ръ всjь'мъ.}

\rem{Лю'дiе:} \bukv{И=} ду'хови твоему`.

\deacon{W\т и=м\ркъ ст~а'гw _е=vа'нгелiя чте'нiе.}

\lik{Сла'ва тебjь`, гд\си,\* сла'ва тебjь`.}

\iereios{Во'нмемъ.} \rem{Дiа'конъ чита'етъ _е=v\глiе. По сконча'нiи
  _е=v\глiя:} \bukv{С}ла'ва тебjь`, гд\си, сла'ва тебjь`.

\subsection[Сугу'бая _е=ктенiа`]{Та'же сугу'бая _е=ктенiа`:}

\bukv{Р}це'мъ вси` w\т всея` души`, и= w\т всегw` помышле'нiя на'шегw рце'мъ.

\lik{\bukv{Г}д\си, поми'луй.}

\bukv{Г}д\си, вседержи'телю, бж~е _о=т_е'цъ на'шихъ, мо'лимъ ти' ся,
о_у=слы'ши и= поми'луй.

\lik{\bukv{Г}д\си, поми'луй.}

\bukv{П}оми'луй на'съ, бж~е, по вели'цjьй ми'лости твое'й, мо'лимъ ти' ся,
о_у=слы'ши и= поми'луй.

\lik{\bukv{Г}д\си, поми'луй.} \rem{три'жды.}

\bukv{Е=}ще` мо'лимся w= ст~jь'шихъ правосла'вныхъ патрiа'рсjьхъ,
преwсвяще'ннjьйшемъ митрополи'тjь \rem{[и=ли`:} а=рхiепi'скопjь, \rem{и=ли`:}
_е=пи'скопjь\rem{]} на'шемъ \rem{[и=м\ркъ, _е=гw'же w='бласть]}, и= все'й во
хр\стjь` бра'тiи на'шей.

\lik{\bukv{Г}д\си, поми'луй.} \rem{три'жды.}

\bukv{Е=}ще` мо'лимся w= странjь` на'шей, прави'телjьхъ и= во'инствjь.

\lik{\bukv{Г}д\си, поми'луй.} \rem{три'жды.}

\bukv{Е=}ще` мо'лимся w= бра'тiяхъ на'шихъ, свяще'нницjьхъ,
священномона'сjьхъ, и= все'мъ во хр\стjь` бра'тствjь на'шемъ.

\lik{\bukv{Г}д\си, поми'луй.} \rem{три'жды.}

\bukv{Е=}ще` мо'лимся w= блаже'нныхъ и= приснопа'мятныхъ, святjь'йшихъ
патрiа'рсjьхъ правосла'вныхъ, \rem{[и= благочести'выхъ царjь'хъ, и=
  благовjь'рныхъ цари'цjьхъ,]} и= созда'телехъ ст~а'гw хра'ма сегw`
\rem{[а='ще во _о=би'тели:} ст~ы'я _о=би'тели сiя`\rem{]}, и= всjь'хъ пре'жде
почи'вшихъ _о=тцjь'хъ и= бра'тiяхъ, здjь` лежа'щихъ и= повсю'ду,
правосла'вныхъ.

\lik{\bukv{Г}д\си, поми'луй.} \rem{три'жды.}

\rem{Здjь` прилага'ются прош_е'нiя w= боля'щихъ и= про'чiя, а='ще потре'бно
  _е='сть.\par}

\bukv{Е=}ще` мо'лимся w= плодонося'щихъ и= добродjь'ющихъ во ст~jь'мъ и=
всечестнjь'мъ хра'мjь се'мъ, тружда'ющихся, пою'щихъ и= предстоя'щихъ лю'дехъ,
w=жида'ющихъ w\т тебе` вели'кiя и= бога'тыя ми'лости.

\lik{\bukv{Г}д\си, поми'луй.} \rem{три'жды.}

\subsubsection{Возглаше'нiе:}

\bukv{JА='}кw ми'лостивъ и= человjьколю'бецъ бг~ъ _е=си`, и= тебjь` сла'ву
возсыла'емъ, _о=ц~у` и= сн~у, и= ст~о'му дх~у, ны'нjь и= при'снw, и= во вjь'ки
вjькw'въ.

\lik{\bukv{А=}ми'нь.}

\subsubsection{А='ще же бу'детъ w= о_у=со'пшихъ приноше'нiе, дiа'конъ и=ли`
  свяще'нникъ глаго'летъ _е=ктенiю` сiю`:}

\bukv{П}оми'луй на'съ, бж~е, по вели'цjьй мл\сти твое'й, мо'лимъ ти' ся,
о_у=слы'ши и= поми'луй.

\lik{\bukv{Г}д\си, поми'луй.} \rem{три'жды.}

\bukv{Е=}ще мо'лимся w= о_у=покое'нiи ду'шъ о_у=со'пшихъ рабw'въ бж~iихъ,
\rem{и=м\ркъ}, и= w= _е=же прости'тися и='мъ вся'кому прегрjьше'нiю, во'льному
же и= нево'льному.

\lik{\bukv{Г}д\си, поми'луй.} \rem{три'жды.}

\bukv{JА=`}кw да гд\сь бг~ъ о_у=чини'тъ ду'шы и='хъ и=дjь'же пра'веднiи
о_у=покоя'ются.

\lik{\bukv{Г}д\си, поми'луй.} \rem{три'жды.}

\bukv{М}и'лости бж~iя, ца'рствiя небе'снагw, и= w=ставле'нiя грjьхw'въ и='хъ,
о_у= хр\ста` безсме'ртнагw цр~я` и= бг~а на'шегw про'симъ.

\lik{\bukv{П}ода'й, гд\си.}

\deacon{\bukv{Г}д\су помо'лимся.}

\lik{\bukv{Г}д\си, поми'луй.}

\rem{Во'згласъ:} \bukv{JА='}кw ты` _е=си` воскресе'нiе, и= живо'тъ, и= поко'й
о_у=со'пшихъ ра^бъ твои'хъ, \rem{и=м\ркъ}, хр\сте` бж~е на'шъ, и= тебjь`
сла'ву возсыла'емъ, со безнача'льнымъ твои'мъ _о=ц~е'мъ, и= прест~ы'мъ, и=
бл~ги'мъ, и= животворя'щимъ твои'мъ дх~омъ, ны'нjь и= при'снw и= во вjь'ки
вjькw'въ.

\lik{\bukv{А=}ми'нь.}


\subsection[W= w=глаше'нныхъ]{Та'же дiа'конъ:}

\bukv{П}омоли'теся, w=глаше'ннiи, гд\севи.

\lik{\bukv{Г}д\си, поми'луй.}

\bukv{В}jь'рнiи, w= w=глаше'нныхъ помо'лимся, да гд\сь поми'луетъ и='хъ.

\lik{\bukv{Г}д\си, поми'луй.}

\bukv{W=}гласи'тъ и='хъ сло'вомъ и='стины.

\lik{\bukv{Г}д\си, поми'луй.}

\bukv{W\т}кры'етъ и=`мъ _е=v\глiе пра'вды.

\lik{\bukv{Г}д\си, поми'луй.}

\bukv{С}оедини'тъ и='хъ ст~jь'й свое'й, собо'рнjьй и= а=по'стольстjьй це'ркви.

\lik{\bukv{Г}д\си, поми'луй.}

\bukv{С}паси`, поми'луй, заступи` и= сохрани` и='хъ, бж~е, твое'ю бл~года'тiю.

\lik{\bukv{Г}д\си, поми'луй.}

\bukv{W=}глаше'ннiи, главы^ ва'шя гд\севи приклони'те.

\lik{\bukv{Т}ебjь`, гд\си.}

\subsubsection{Возглаше'нiе:}

\bukv{Д}а и= тi'и съ на'ми сла'вятъ пречестно'е и= великолjь'пое и='мя твое`,
_о=ц~а`, и= сн~а, и= ст~а'гw дх~а, ны'нjь и= при'снw, и= во вjь'ки вjькw'въ.

\lik{\bukv{А=}ми'нь.}

\bukv{Е=}ли'цы w=глаше'ннiи, и=зыди'те: \bukv{w=}глаше'ннiи и=зыди'те,
\bukv{_е=}ли'цы w=глаше'ннiи и=зыди'те: \bukv{д}а никто` w\т w=глаше'нныхъ,
_е=ли'цы вjь'рнiи, па'ки и= па'ки ми'ромъ гд\су помо'лимся.

\lik{\bukv{Г}д\си, поми'луй.}

\bukv{З}аступи`, спаси`, поми'луй и= сохрани` на'съ бж~е, твое'ю бл~года'тiю.

\lik{\bukv{Г}д\си, поми'луй.}

\deacon{\bukv{П}рему'дрость.}

\subsubsection{Возглаше'нiе:}

\bukv{JА='}кw подоба'етъ тебjь` вся'кая сла'ва, че'сть, и= поклоне'нiе,
_о=ц~у`, и= сн~у, и= ст~о'му дх~у, ны'нjь и= при'снw и= во вjь'ки вjькw'въ.

\lik{\bukv{А=}ми'нь.}

\deacon{\bukv{П}а'ки и= па'ки ми'ромъ гд\су помо'лимся.}

\lik{\bukv{Г}д\си, поми'луй.}

\bukv{З}аступи`, спаси`, поми'луй и= сохрани` на'съ бж~е, твое'ю бл~года'тiю.

\lik{\bukv{Г}д\си, поми'луй.}

\deacon{\bukv{П}рему'дрость.}

\subsubsection{Возглаше'нiе:}

\bukv{JА='}кw да подъ держа'вою твое'ю всегда` храни'ми, тебjь` сла'ву
возсыла'емъ, _о=ц~у`, и= сн~у, и= ст~о'му дх~у, ны'нjь и= при'снw, и= во
вjь'ки вjькw'въ.

\lik{\bukv{А=}ми'нь.} \rem{[в~]}

\subsubsection{W\тверза'ются ст~ы'я две'ри.}

\subsection[Херувi'мская пjь'снь]{И= пое'тся херувi'мская пjь'снь.}

\begin{large}
\bukv{И='}же херувi'мы\* та'йнw w=бразу'юще,\* и= животворя'щей тр\оцjь\* трист~у'ю
пjь'снь припjьва'юще,\* вся'кое ны'нjь\* жите'йское\* w\тложи'мъ\* попече'нiе.
\end{large}

\rem{По вхо'дjь:} \bukv{А=}ми'нь.

\begin{large}
\bukv{JА='}кw да цр~я` всjь'хъ под\ъи'мемъ,\* а='гг~льскими неви'димw\*
дорv"носи'ма чи'нми. А=ллилу'iа, а=ллилу'iа, а=ллилу'iа.
\end{large}

\subsection{Е=ктенiа` проси'тельная}

\bukv{И=}спо'лнимъ моли'тву на'шу гд\сви.

\lik{\bukv{Г}д\си, поми'луй.}

\bukv{W=} ст~jь'мъ хра'мjь се'мъ, и= съ вjь'рою, благоговjь'нiемъ и= стра'хомъ
бж~iимъ входя'щихъ во'нь, гд\су помо'лимся.

\lik{\bukv{Г}д\си, поми'луй.}

\bukv{W=} и=зба'витися на'мъ w\т вся'кiя ско'рби, гнjь'ва и= ну'жды, гд\су
помо'лимся.

\lik{\bukv{Г}д\си, поми'луй.}

\bukv{З}аступи`, спаси`, поми'луй и= сохрани` на'съ, бж~е, твое'ю бл~года'тiю.

\lik{\bukv{Г}д\си, поми'луй.}

\bukv{Д}не` всегw` соверше'на, свя'та, ми'рна и= безгрjь'шна, о_у= гд\са
про'симъ.

\lik{\bukv{Г}д\си, поми'луй.}

\bukv{А='}гг~ла ми'рна, вjь'рна наста'вника, храни'теля ду'шъ и= тjьле'съ
на'шихъ, о_у= гд\са про'симъ.

\lik{\bukv{Г}д\си, поми'луй.}

\bukv{П}роще'нiя и= w=ставле'нiя грjьхw'въ и= прегрjьше'нiй на'шихъ, о_у=
гд\са про'симъ.

\lik{\bukv{П}ода'й, гд\си.}

\bukv{Д}о'брыхъ и= поле'зныхъ душа'мъ на'шимъ и= ми'ра мi'рови, о_у= гд\са
про'симъ.

\lik{\bukv{П}ода'й, гд\си.}

\bukv{П}ро'чее вре'мя живота` на'шегw въ ми'рjь и= покая'нiи сконча'ти о_у=
гд\са про'симъ.

\lik{\bukv{П}ода'й, гд\си.}

\bukv{Х}р\стiа'нскiя кончи'ны живота` на'шегw, безболjь'зненны, непосты'дны,
ми'рны, и= до'брагw w\твjь'та на стра'шнjьмъ суди'щи хр\сто'вjь, про'симъ.

\lik{\bukv{П}ода'й, гд\си.}

\bukv{П}рест~у'ю, пречи'стую, пребл~гослове'нную, сла'вную влч\дцу на'шу бц\ду
и= приснодв~у мр~i'ю, со всjь'ми ст~ы'ми помяну'вше, са'ми себе`, и= дру'гъ
дру'га, и= ве'сь живо'тъ на'шъ хр\сту бг~у предади'мъ.

\lik{\bukv{Т}ебjь`, гд\си.}

\subsubsection{Возглаше'нiе}

\bukv{Щ}едро'тами _е=диноро'днагw сн~а твоегw`, съ ни'мже бл~гослове'нъ
_е=си`, со прест~ы'мъ и= бл~ги'мъ и= животворя'щимъ твои'мъ дх~омъ, ны'нjь и=
при'снw, и= во вjь'ки вjькw'въ.

\lik{\bukv{А=}ми'нь.}

\iereios{\bukv{М}и'ръ всjь'мъ.}

\lik{\bukv{И=} ду'хови твоему`.}

\deacon{\bukv{В}озлю'бимъ дру'гъ дру'га, да _е=диномы'слiемъ и=сповjь'мы.}

\lik{{\large\bukv{_О=}ц~а`, и= сн~а, и= ст~а'го дх~а,\* тр\оцу _е=диносу'щную и=
    нераздjь'льную.}

\subsection{Сv"мво'лъ вjь'ры}

\deacon{\bukv{Д}ве'ри, две'ри, прему'дростiю во'нмемъ.}

\vskip 0.25\baselineskip

\num{а~}{\bukv{В}jь'рую} во _e=ди'наго бг~а _о=ца`, вседержи'теля, творца`
нб~у и= земли`, ви^димымъ же всjь^мъ и= неви^димымъ.

\num{в~}{\bukv{И=}} во _e=ди'наго гд\са i=и~са хр\ста`, сн~а бж~iя,
_e=диноро'днаго, и='же w\т _о=ц~а` рожде'ннаго пре'жде всjь'хъ
вjь^къ. Свjь'та w\т свjь'та, бг~а и='стинна w\т бг~а и='стинна,
рожде'нна, несотворе'нна, _e=диносу'щна _о=ц~у`, и='мже вся^
бы'ша.

\num{г~}{\bukv{Н}а'съ} ра'ди человjь^къ, и= на'шегw ра'ди сп~се'нiя
сше'дшаго съ нб~съ, и= воплоти'вшагося w\т дх~а ст~а и= мр~i'и
дв~ы и= вочеловjь'чшася.

\num{д~}{\bukv{Р}аспя'таго} же за ны` при понтi'йстjьмъ пiла'тjь, и=
страда'вша, и= погребе'на.

\num{_е~}{\bukv{И=}} воскре'сшаго въ тре'тiй де'нь по писа'нi_eмъ.

\num{s~}{\bukv{И=}} возше'дшаго на нб~са`, и= седя'ща w=десну'ю _о=ц~а`.

\num{з~}{\bukv{И=}} па'ки гряду'щаго со сла'вою, суди'ти живы^мъ и=
м_e'ртвымъ, _e=гw'же цр\ствiю не бу'детъ конца`.

\num{и~}{\bukv{И=}} въ дх~а ст~а'го, гд\са, животворя'щаго, и='же w\т
_о=ц~а` и=сходя'щаго, и='же со _о=ц~е'мъ и= сн~омъ спокланя'ема и=
ссла'вима, глаго'лавшаго пр\орw'ки.

\num{f~}{\bukv{В}о} _e=ди'ну святу'ю, собо'рную, и= а=п\сльскую цр~ковь.

\num{_i~}{\bukv{И=}сповjь'дую} _e=ди'но креще'нiе во w=ставле'нiе
грjьхw'въ.

\num{а~_i}{\bukv{Ч}а'ю} воскресе'нiя ме'ртвыхъ: 

\num{в~_i}{\bukv{И=}} жи'зни бу'дущагw вjь'ка. А=ми'нь.

\subsection{Ми'лость ми'ра}

\deacon{\bukv{С}та'немъ до'брjь, ста'немъ со стра'хомъ, во'нмемъ, ст~о'е
  возноше'нiе въ ми'рjь приноси'ти.}

\lik{{\large\bukv{М}и'лость ми'ра, же'ртву хвале'нiя.}}

\iereios{\bukv{Б}лагода'ть гд\са на'шегw i=и~са хр\ста`, и= любы` бг~а и=
  _о=ц~а`, и= прича'стiе ст~а'гw дх~а, бу'ди со всjь'ми ва'ми.}

\lik{{\large\bukv{И=} со ду'хомъ твои'мъ.}}

\iereios{\bukv{Г}орjь` и=мjь'имъ сердца`.}

\lik{{\large\bukv{И='}мамы ко гд\су.}}

\iereios{\bukv{Б}лагодари'мъ гд\са.}

\lik{{\large\bukv{Д}осто'йно и= пра'ведно _е='сть покланя'тися _о=ц~у`, и= сн~у, и=
  ст~о'му дх~у, тр\оцjь _е=диносу'щнjьй и= нераздjь'льнjьй.}}

\iereios{\bukv{П}обjь'дную пjь'снь пою'ще, вопiю'ще, взыва'юще и= глаго'люще.}

\lik{{\large\bukv{С}вя'тъ, свя'тъ, свя'тъ гд\сь саваw'fъ, и=спо'лнь не'бо и= земля`
сла'вы твоея`: w=са'нна въ вы'шнихъ, благослове'нъ гряды'й во и='мя
гд\сне, w=са'нна въ вы'шнихъ.}}

\iereios{\bukv{П}рiими'те, jа=ди'те: сiе` _е='сть тjь'ло мое`, _е='же за вы`
  ломи'мое во w=ставле'нiе грjьхw'въ.}

\lik{\bukv{А=}ми'нь.}

\iereios{\bukv{П}i'йте w\т нея` вси`, сiя` _е='сть кро'вь моя` но'вагw
  завjь'та, jа='же за вы` и= за мнw'гiя и=злива'емая во w=ставле'нiе грjьхw'въ.}

\lik{\bukv{А=}ми'нь.}

\iereios{\bukv{Т}воя^ w\т твои'хъ тебjь` принося'ще w= всjь'хъ и= за вся^.}

\lik{{\large\bukv{Т}ебе` пое'мъ, тебе` благослови'мъ, тебjь`
    благодари'мъ, гд\си, и= мо'лимъ ти ся, бж~е на'шъ.}}

\rem{Посе'мъ, прiи'мъ кади'ло, возглаша'етъ iере'й:}

\bukv{И=}зря'днw w= прест~jь'й, пречи'стjьй, пребл~гослове'ннjьй,
сла'внjьй вл\дчцjь на'шей бц\дjь и= приснwдв~jь мр~i'и.

\rem{Ли'къ пое'тъ:} {\large\bukv{Д}осто'йно _e='сть ja='кw вои'стинну
  блажи'ти тя` бц\ду, присноблаже'нную и= пренепоро'чную,
  и= мт~рь бг~а на'шегw: ч\стнjь'йшую херувi^мъ и сла'внjьйшую
  без\ъ сравне'нiя серафi^мъ, без\ъ и=стлjь'нiя бг~а сло'ва
  ро'ждшую, су'щую бг~оро'дицу тя` велича'емъ.\par}

\rem{Въ вели'кiя пра'здники пое'мъ задосто'йникъ. Зри` въ концjь`.}

\vskip 0.5\baselineskip

\rem{Свяще'нникъ мо'лится та'йнw, и= по пjь'нiи стiха`, возглаша'етъ:}

\bukv{В}ъ пе'рвыхъ помяни`, гд\си, правосла^вныя святjь^йшiя патрiа'рхи и=
господи'на на'шегw, высокопреwсвяще'ннjьйшаго митрополи'та \rem{[и='мярекъ]},
\rem{и=ли`} господи'на на'шегw, преwсвяще'ннjьйшаго _е=п\скопа
\rem{[и='мярекъ, _е=гw'же _о='бласть]}: и='хже да'руй ст~ы'мъ твои^мъ
це'рквамъ, въ ми'рjь цjь'лыхъ, че'стныхъ, здра'выхъ, долгоде'нствующихъ,
пра'во пра'вящихъ сло'во твоея` и='стины.

\lik{\bukv{И=} всjь'хъ и= вся^.}

\rem{Свяще'нникъ мо'лится та'йнw, и= по пjь'нiи стiха`, возглаша'етъ:}

\bukv{И=} да'ждь на'мъ _е=ди'нjьми о_у=сты` и= _е=ди'нjьмъ срц\демъ сла'вити
и= воспjьва'ти пречестно'е и= великолjь'пое и='мя твое`, _о=ц~а` и= сн~а, и=
ст~а'гw дх~а, ны'нjь и= пр\снw, и= во вjь'ки вjькw'въ.

\lik{\bukv{А=}ми'нь.}

\rem{Свяще'нникъ, w=бра'щься къ лю'демъ и= благословля'я, глаго'летъ:}

\bukv{И=} да бу'дутъ ми'лwсти вели'кагw бг~а и= сп~са на'шегw i=и~са хр\ста`
со всjь'ми ва'ми.

\lik{\bukv{И=} со дх~омъ твои'мъ.}

\subsection{Е=ктенiа`}

\deacon{\bukv{В}ся^ ст~ы^я помяну'вше, па'ки и= па'ки ми'ромъ гд\су помо'лимся.}

\lik{\bukv{Г}д\си, поми'луй.}

\bukv{W=} принесе'нныхъ и= w=свяще'нныхъ чт\сны'хъ дарjь'хъ гд\су помо'лимся.

\lik{\bukv{Г}д\си, поми'луй.}

\bukv{JА='}кw да чл~вjьколю'бецъ бг~ъ на'шъ, прiе'мъ я=` во ст~ы'й и=
пренб\сный и= мы'сленный сво'й же'ртвенникъ, въ воню` благоуха'нiя духо'внагw,
возниспо'слетъ на'мъ бж\ственную бл~года'ть и= да'ръ ст~а'гw дх~а, помо'лимся.

\lik{\bukv{Г}д\си, поми'луй.}

\bukv{W=} и=зба'витися на'мъ w\т вся'кiя ско'рби, гнjь'ва и= ну'жды, гд\су
помо'лимся.

\lik{\bukv{Г}д\си, поми'луй.}

\bukv{З}аступи`, спаси`, поми'луй и= сохрани` на'съ, бж~е, твое'ю бл~года'тiю.

\lik{\bukv{Г}д\си, поми'луй.}

\bukv{Д}не` всегw` соверше'на, свя'та, ми'рна и= безгрjь'шна, о_у= гд\са
про'симъ.

\lik{\bukv{Г}д\си, поми'луй.}

\bukv{А='}гг~ла ми'рна, вjь'рна наста'вника, храни'теля ду'шъ и= тjьле'съ
на'шихъ, о_у= гд\са про'симъ.

\lik{\bukv{Г}д\си, поми'луй.}

\bukv{П}роще'нiя и= w=ставле'нiя грjьхw'въ и= прегрjьше'нiй на'шихъ, о_у=
гд\са про'симъ.

\lik{\bukv{П}ода'й, гд\си.}

\bukv{Д}о'брыхъ и= поле'зныхъ душа'мъ на'шимъ и= ми'ра мi'рови, о_у= гд\са
про'симъ.

\lik{\bukv{П}ода'й, гд\си.}

\bukv{П}ро'чее вре'мя живота` на'шегw въ ми'рjь и= покая'нiи сконча'ти о_у=
гд\са про'симъ.

\lik{\bukv{П}ода'й, гд\си.}

\bukv{Х}р\стiа'нскiя кончи'ны живота` на'шегw, безболjь'зненны, непосты'дны,
ми'рны, и= до'брагw w\твjь'та на стра'шнjьмъ суди'щи хр\сто'вjь, про'симъ.

\lik{\bukv{П}ода'й, гд\си.}

\bukv{С}оедине'нiе вjь'ры и= прича'стiе ст~а'гw дх~а и=спроси'вше, са'ми
себе`, и= дру'гъ дру'га, и= ве'сь живо'тъ на'шъ хр\сту` бг~у предади'мъ.

\lik{\bukv{Т}ебjь`, гд\си.}

\subsection[О='ч~е на'шъ]{Свяще'нникъ, возглаше'нiе:}

\bukv{И=} сподо'би на'съ, вл\дко, со дерзнове'нiемъ, неwсужде'нно смjь'ти
призыва'ти тебе` нб\снаго бг~а _о=ц~а`, и= глаго'лати:

\rem{Лю'дiе:} {\large\bukv{О='}ч~е на'шъ, и='же _e=си` на нб~сjь'хъ! да святи'тся и='мя твое`:
да прiи'детъ цр\ствiе твое`: да бу'детъ во'ля твоя`, ja='кw на
нб~си`, и= на земли`. Хлjь'бъ на'шъ насу'щный да'ждь на'мъ дне'сь:
и= w=ста'ви на'мъ до'лги на'шя, ja='коже и= мы`
w=ставля'емъ должникw'мъ на'шымъ: и= не введи` на'съ
во и=скуше'нiе, но и=зба'ви на'съ w\т лука'вагw.\par}

\iereios{\bukv{JА='}кw твое` _е='сть ца'рство, и= си'ла, и= сла'ва, _о=ц~а` и=
  сн~а, и= ст~а'гw дх~а, ны'нjь и= пр\снw, и= во вjь'ки вjькw'въ.}

\lik{\bukv{А=}ми'нь.}

\iereios{\bukv{М}и'ръ всjь'мъ.}

\lik{\bukv{И=} дх~ови твоему`.}

\iereios{\bukv{Г}лавы^ ва'шя гд\севи приклони'те.}

\lik{\bukv{Т}ебjь`, гд\си.}

\rem{Свяще'нникъ мо'лится та'йнw. Возглаше'нiе.}

\bukv{Б}л~года'тiю и= щедро'тами и= чл~вjьколю'бiемъ _е=диноро'днагw сн~а
твоегw`, съ ни'мже бл~гослове'нъ _е=си`, со прест~ы'мъ и бл~ги'мъ и=
животворя'щимъ твои'мъ дх~омъ, нн~jь и= пр\снw, и= во вjь'ки вjькw'въ.

\lik{\bukv{А=}ми'нь.} \rem{[Два'жды.]}

\rem{Свяще'нникъ мо'лится та'йнw, посе'мъ вознося` ст~ы'й хле'бъ
  возглаша'етъ:}

\bukv{В}о'нмемъ, ст~а^я ст~ы^мъ.

\lik{{\large\bukv{Е=}ди'нъ ст~ъ, _е=ди'нъ гд\сь, i=и~съ хр\сто'съ, во сла'ву
    бг~а _о=ц~а`. А=ми'нь.}}

\subsection[Причаще'нiе мiря'нъ]{И= пою'тъ ли'цы кiно'нiкъ дне`, 
  и=ли` ст~а'гw. Зри` въ концjь`. По причаще'нiи сщ~е'нникъ w\тверза'етъ
  ца'рскiя врата` и= съ ча'шей возглаша'етъ:}

\bukv{С}о стра'хомъ бж~iимъ, и= вjь'рою приступи'те.

\lik{{\large\bukv{Б}л~гослове'нъ гряды'й во и='мя гд\сне, бг~ъ гд\сь и=
    jа=ви'ся на'мъ.\par}}

\subsubsection{Моли'тва ко причаще'нiю.}

{\large\bukv{В}jь'рую, гд\си, и= и=сповjь'дую, ja='кw ты` _e=си` вои'стинну
хр\сто'съ, сн~ъ бг~а жива'гw, прише'дый въ мi'ръ грjь^шныя
сп~сти`, w\т ни'хже пе'рвый _e='смь а='зъ. Е=ще` вjь'рую, ja='кw
сiе` _e='сть са'мое преч\стое тjь'ло твое`, и= сiя` _e='сть са'мая
ч\стна'я кро'вь твоя`. Молю'ся о_у=`бо тебjь`: поми'луй мя`, и=
прости' ми прегрjьш_e'нiя моя^, вw'льная и= невw'льная, ja=`же
сло'вомъ, ja=`же дjь'ломъ, ja=`же вjь'дjьнiемъ, и=
невjь'дjьнiемъ, и= сподо'би мя` неwсужде'ннw
причасти'тися преч\стыхъ твои'хъ та'инствъ, во w=ставле'нiе
грjьхw'въ и= въ жи'знь вjь'чную, а=ми'нь.

\bukv{В}е'чери твоея' та'йныя дне'сь, сн~е бж~iй, прича'стника мя`
прiими`: не бо` врагw'мъ твои^мъ та'йну повjь'мъ, ни лобза'нiя
ти` да'мъ ja='кw i=у'да, но ja='кw разбо'йникъ и=сповjь'даю
тя`: помяни' мя, гд\си, во цр\ствiи твое'мъ.

\bukv{Д}а не въ су'дъ, и=ли` во w=сужде'нiе бу'детъ мнjь` причаще'нiе ст~ы'хъ
твои'хъ та'инъ гд\си, но во и=сцjьле'нiе души` и= тjь'ла, а=ми'нь.\par}

\rem{Та'же приступа'ютъ хотя'щiи причасти'тися. И= и='дутъ _е=ди'нъ по
  _е=ди'ному, и= покланя'ются со вся'цjьмъ о_у=миле'нiемъ и= стра'хомъ,
  согбе'ннjь ру'цjь къ пе'рс_емъ и=му'ще: та'же прiи'метъ кi'йждо
  бж~е'ств_енныя та'йны. Ли'къ же и=ли` лю'дiе пою'тъ:\par}

{\large\bukv{Т}jь'ло хр\сто'во прiими'те, и=сто'чника безсме'ртнагw вкуси'те.
А=ллилу'iа.\par}

\subsection[По причаще'нiи]{Свяще'нникъ же благословля'етъ лю'ди, возглаша'я:}

\bukv{С}паси` бж~е, лю'ди твоя^ и= бл~гослови` достоя'нiе твое`.

\rem{Ли'къ же пое'тъ:} {\large\bukv{В}и'дjьхомъ свjь'тъ и='стинный, прiя'хомъ
  дх~а нб\снаго, w=брjьто'хомъ вjь'ру и='стинную, нераздjь'льнjьй тр\оцjь
  покланя'емся: та' бо на'съ спасла` _e='сть.\par}

\rem{Свяще'нникъ та'йнw:} \bukv{Б}л~гослове'нъ бг~ъ на'шъ:

\subsubsection{И= возгла'снw:}

\bukv{В}сегда`, ны'нjь и= пр\снw, и= во вjь'ки вjькw'въ.

\lik{\bukv{А=}ми'нь.}

{\large\bukv{Д}а и=спо'лнятся о_у=ста` на^ша хвале'нiя твоегw`, гд\си,
  ja='кw да пое'мъ сла'ву твою`, ja='кw сподо'билъ _e=си` на'съ причасти'тися
  ст~ы^мъ твои^мъ, бж~е'ств_eннымъ, безсм_e'ртнымъ и= животворя'щымъ та'йнамъ:
  соблюди` на'съ во твое'й ст~ы'ни, ве'сь де'нь поуча'тися пра'вдjь твое'й.
  А=ллилу'iа, а=ллилу'iа, а=ллилу'iа.\par}

\deacon{\bukv{П}ро'сти прiи'мше бж~е'ственныхъ ст~ы'хъ, преч\стыхъ,
  безсме'ртныхъ нб\сныхъ и= животворя'щихъ, стра'шныхъ хр\сто'выхъ та^инъ,
  досто'йнw благодари'мъ гд\са.}

\lik{\bukv{Г}д\си, поми'луй.}

\bukv{З}аступи`, спаси`, поми'луй и= сохрани` на'съ, бж~е, твое'ю бл~года'тiю.

\lik{\bukv{Г}д\си, поми'луй.}

\bukv{Д}е'нь ве'сь соверше'нъ, ст~ъ, ми'ренъ и= безгрjь'шенъ и=спроси'вше,
са'ми себе`, и= дру'гъ дру'га, и= ве'сь живо'тъ на'шъ хр\сту` бг~у предади'мъ.

\lik{\bukv{Т}ебjь`, гд\си.}

\subsubsection{Возглаше'нiе.}

\bukv{JА='}кw ты` _е=си` w=свяще'нiе на'ше, и= тебjь` сла'ву возсыла'емъ,
_о=ц~у` и= сн~у и= ст~о'му дх~у, ны'нjь и= пр\снw и= во вjь'ки вjькw'въ.

\lik{\bukv{А=}ми'нь.}

\iereios{\bukv{С}ъ ми'ромъ и=зы'демъ.}

\lik{\bukv{W=} и='мени гд\сни.}

\deacon{\bukv{Г}д\су помо'лимся.}

\lik{\bukv{Г}д\си, поми'луй.}

\subsubsection{Моли'тва заамвw'нная, возгла'снw:}

\bukv{Б}л~гословля'яй бл~гословя'щыя тя`, гд\си, и= w=сщ~а'яй на тя`
о_у=пова'ющыя, сп~си` лю'ди твоя^ и= бл~гослови` достоя'нiе твое`,
и=сполне'нiе цр~кве твоея^ сохрани`, w=ст~и` лю'бящыя бл~голjь'пiе до'му
твоегw`: ты` тjь'хъ воспросла'ви бж\ственною твое'ю си'лою, и= не w=ста'ви
на'съ, о_у=пова'ющихъ на тя`. Ми'ръ мi'рови твоему` да'руй, цр~квамъ твои^мъ,
сщ~е'нникwмъ, и= всjь'мъ лю'демъ твои'мъ. JА='кw вся'кое дая'нiе бл~го, и=
вся'къ да'ръ соверше'нъ свы'ше _е='сть, сходя'й, w\т тебjь` _о=ц~а` свjь'тwвъ:
и= тебjь` сла'ву, и= бл~годаре'нiе, и= поклоне'нiе возсыла'емъ, _о=ц~у` и=
сн~у и ст~о'му дх~у, ны'нjь и= пр\снw, и= во вjь'ки вjькw'въ.

\lik{\bukv{А=}ми'нь.}

\rem{Та'же:} {\large\bukv{Б}у'ди и='мя гд\сне бл~гослове'нно w\т ны'нjь и= до
  вjь'ка. \rem{Три'жды.\par}}

\subsubsection{И= _псалw'мъ л~г:}

\bukv{Б}л~гословлю` гд\са на вся'кое вре'мя,\* вы'ну хвала` _е=гw` во
о_у=стjь'хъ мои'хъ.

\bukv{W=} гд\сjь похва'лится душа` моя`:\* да о_у=слы'шатъ кро'тцыи и=
возвеселя'тся.

\bukv{В}озвели'чите гд\са со мно'ю,\* и= вознесе'мъ и='мя _е=гw` вку'пjь.

\bukv{В}зыска'хъ гд\са, и= о_у=слы'ша мя`,\* и= w\т всjь'хъ скорбе'й мои'хъ
и=зба'ви мя`.

\bukv{П}риступи'те къ нему`, и= просвjьти'теся,\* и= ли'ца ва^ша не
постыдя'тся.

\bukv{С}е'й ни'щiй воззва`, и= гд\сь о_у=слы'ша и=`,\* и= w\т всjь'хъ скорбе'й
_е=гw` сп~се` и=`.

\bukv{W=}полчи'тся а='гг~лъ гд\снь _о='крестъ боя'щихся _е=гw`,\* и=
и=зба'витъ и='хъ.

\bukv{В}куси'те и= ви'дите, jа='кw бл~гъ гд\сь:\* бл~же'нъ му'жъ, и='же
о_у=пова'етъ на'нь.

\bukv{Б}о'йтеся гд\са вси` ст~i'и _е=гw`,\* jа='кw нjь'сть лише'нiя боя'щымся
_е=гw`.

\bukv{Б}ога'тiи w=бнища'ша и= взалка'ша,\* взыска'ющiи же гд\са не лиша'тся
вся'кагw бла'га.

\iereios{\bukv{Б}л~гослове'нiе гд\сне на ва'съ, тогw` бл~года'тiю и= чл~колю'бiемъ,
  всегда`, ны'нjь, и= пр\снw и= во вjь'ки вjькw'въ.}

\lik{\bukv{А=}ми'нь.}

\iereios{\bukv{С}ла'ва тебjь`, хр\сте` бж~е, о_у=пова'нiе на'ше, сла'ва
  тебjь`.}

\lik{\bukv{С}ла'ва, и= ны'нjь: Гд\си, поми'луй, \rem{три'жды}. Бл~гослови`.}

\iereios{\bukv{Х}р\сто'съ и='стинный бг~ъ на'шъ, мл~твами преч\стыя своея`
  ма'тере \rem{[и= прw'чая]}, и=же во ст~ы'хъ _о=ц~а` на'шегw i=wа'нна,
  а=рхiепи'скопа кwнстанти'ня гра'да, златоу'стагw: и= ст~а'гw \rem{и=м\ркъ,
  _е=гw'же _е='сть хра'мъ и= _е=гw'же _е='сть де'нь:} и= всjь'хъ ст~ы'хъ,
  поми'луетъ и= спасе'тъ на'съ, jа='кw бл~гъ и= чл~колю'бецъ.}

\lik{\bukv{А=}ми'нь.}

\csendpict

\end{document}
